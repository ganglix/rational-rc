%% Generated by Sphinx.
\def\sphinxdocclass{report}
\documentclass[letterpaper,10pt,english]{sphinxmanual}
\ifdefined\pdfpxdimen
   \let\sphinxpxdimen\pdfpxdimen\else\newdimen\sphinxpxdimen
\fi \sphinxpxdimen=.75bp\relax

\PassOptionsToPackage{warn}{textcomp}
\usepackage[utf8]{inputenc}
\ifdefined\DeclareUnicodeCharacter
% support both utf8 and utf8x syntaxes
  \ifdefined\DeclareUnicodeCharacterAsOptional
    \def\sphinxDUC#1{\DeclareUnicodeCharacter{"#1}}
  \else
    \let\sphinxDUC\DeclareUnicodeCharacter
  \fi
  \sphinxDUC{00A0}{\nobreakspace}
  \sphinxDUC{2500}{\sphinxunichar{2500}}
  \sphinxDUC{2502}{\sphinxunichar{2502}}
  \sphinxDUC{2514}{\sphinxunichar{2514}}
  \sphinxDUC{251C}{\sphinxunichar{251C}}
  \sphinxDUC{2572}{\textbackslash}
\fi
\usepackage{cmap}
\usepackage[T1]{fontenc}
\usepackage{amsmath,amssymb,amstext}
\usepackage{babel}



\usepackage{times}
\expandafter\ifx\csname T@LGR\endcsname\relax
\else
% LGR was declared as font encoding
  \substitutefont{LGR}{\rmdefault}{cmr}
  \substitutefont{LGR}{\sfdefault}{cmss}
  \substitutefont{LGR}{\ttdefault}{cmtt}
\fi
\expandafter\ifx\csname T@X2\endcsname\relax
  \expandafter\ifx\csname T@T2A\endcsname\relax
  \else
  % T2A was declared as font encoding
    \substitutefont{T2A}{\rmdefault}{cmr}
    \substitutefont{T2A}{\sfdefault}{cmss}
    \substitutefont{T2A}{\ttdefault}{cmtt}
  \fi
\else
% X2 was declared as font encoding
  \substitutefont{X2}{\rmdefault}{cmr}
  \substitutefont{X2}{\sfdefault}{cmss}
  \substitutefont{X2}{\ttdefault}{cmtt}
\fi


\usepackage[Bjarne]{fncychap}
\usepackage{sphinx}

\fvset{fontsize=\small}
\usepackage{geometry}


% Include hyperref last.
\usepackage{hyperref}
% Fix anchor placement for figures with captions.
\usepackage{hypcap}% it must be loaded after hyperref.
% Set up styles of URL: it should be placed after hyperref.
\urlstyle{same}

\addto\captionsenglish{\renewcommand{\contentsname}{Contents:}}

\usepackage{sphinxmessages}
\setcounter{tocdepth}{1}


% Jupyter Notebook code cell colors
\definecolor{nbsphinxin}{HTML}{307FC1}
\definecolor{nbsphinxout}{HTML}{BF5B3D}
\definecolor{nbsphinx-code-bg}{HTML}{F5F5F5}
\definecolor{nbsphinx-code-border}{HTML}{E0E0E0}
\definecolor{nbsphinx-stderr}{HTML}{FFDDDD}
% ANSI colors for output streams and traceback highlighting
\definecolor{ansi-black}{HTML}{3E424D}
\definecolor{ansi-black-intense}{HTML}{282C36}
\definecolor{ansi-red}{HTML}{E75C58}
\definecolor{ansi-red-intense}{HTML}{B22B31}
\definecolor{ansi-green}{HTML}{00A250}
\definecolor{ansi-green-intense}{HTML}{007427}
\definecolor{ansi-yellow}{HTML}{DDB62B}
\definecolor{ansi-yellow-intense}{HTML}{B27D12}
\definecolor{ansi-blue}{HTML}{208FFB}
\definecolor{ansi-blue-intense}{HTML}{0065CA}
\definecolor{ansi-magenta}{HTML}{D160C4}
\definecolor{ansi-magenta-intense}{HTML}{A03196}
\definecolor{ansi-cyan}{HTML}{60C6C8}
\definecolor{ansi-cyan-intense}{HTML}{258F8F}
\definecolor{ansi-white}{HTML}{C5C1B4}
\definecolor{ansi-white-intense}{HTML}{A1A6B2}
\definecolor{ansi-default-inverse-fg}{HTML}{FFFFFF}
\definecolor{ansi-default-inverse-bg}{HTML}{000000}

% Define an environment for non-plain-text code cell outputs (e.g. images)
\makeatletter
\newenvironment{nbsphinxfancyoutput}{%
    % Avoid fatal error with framed.sty if graphics too long to fit on one page
    \let\sphinxincludegraphics\nbsphinxincludegraphics
    \nbsphinx@image@maxheight\textheight
    \advance\nbsphinx@image@maxheight -2\fboxsep   % default \fboxsep 3pt
    \advance\nbsphinx@image@maxheight -2\fboxrule  % default \fboxrule 0.4pt
    \advance\nbsphinx@image@maxheight -\baselineskip
\def\nbsphinxfcolorbox{\spx@fcolorbox{nbsphinx-code-border}{white}}%
\def\FrameCommand{\nbsphinxfcolorbox\nbsphinxfancyaddprompt\@empty}%
\def\FirstFrameCommand{\nbsphinxfcolorbox\nbsphinxfancyaddprompt\sphinxVerbatim@Continues}%
\def\MidFrameCommand{\nbsphinxfcolorbox\sphinxVerbatim@Continued\sphinxVerbatim@Continues}%
\def\LastFrameCommand{\nbsphinxfcolorbox\sphinxVerbatim@Continued\@empty}%
\MakeFramed{\advance\hsize-\width\@totalleftmargin\z@\linewidth\hsize\@setminipage}%
\lineskip=1ex\lineskiplimit=1ex\raggedright%
}{\par\unskip\@minipagefalse\endMakeFramed}
\makeatother
\newbox\nbsphinxpromptbox
\def\nbsphinxfancyaddprompt{\ifvoid\nbsphinxpromptbox\else
    \kern\fboxrule\kern\fboxsep
    \copy\nbsphinxpromptbox
    \kern-\ht\nbsphinxpromptbox\kern-\dp\nbsphinxpromptbox
    \kern-\fboxsep\kern-\fboxrule\nointerlineskip
    \fi}
\newlength\nbsphinxcodecellspacing
\setlength{\nbsphinxcodecellspacing}{0pt}

% Define support macros for attaching opening and closing lines to notebooks
\newsavebox\nbsphinxbox
\makeatletter
\newcommand{\nbsphinxstartnotebook}[1]{%
    \par
    % measure needed space
    \setbox\nbsphinxbox\vtop{{#1\par}}
    % reserve some space at bottom of page, else start new page
    \needspace{\dimexpr2.5\baselineskip+\ht\nbsphinxbox+\dp\nbsphinxbox}
    % mimick vertical spacing from \section command
      \addpenalty\@secpenalty
      \@tempskipa 3.5ex \@plus 1ex \@minus .2ex\relax
      \addvspace\@tempskipa
      {\Large\@tempskipa\baselineskip
             \advance\@tempskipa-\prevdepth
             \advance\@tempskipa-\ht\nbsphinxbox
             \ifdim\@tempskipa>\z@
               \vskip \@tempskipa
             \fi}
    \unvbox\nbsphinxbox
    % if notebook starts with a \section, prevent it from adding extra space
    \@nobreaktrue\everypar{\@nobreakfalse\everypar{}}%
    % compensate the parskip which will get inserted by next paragraph
    \nobreak\vskip-\parskip
    % do not break here
    \nobreak
}% end of \nbsphinxstartnotebook

\newcommand{\nbsphinxstopnotebook}[1]{%
    \par
    % measure needed space
    \setbox\nbsphinxbox\vbox{{#1\par}}
    \nobreak % it updates page totals
    \dimen@\pagegoal
    \advance\dimen@-\pagetotal \advance\dimen@-\pagedepth
    \advance\dimen@-\ht\nbsphinxbox \advance\dimen@-\dp\nbsphinxbox
    \ifdim\dimen@<\z@
      % little space left
      \unvbox\nbsphinxbox
      \kern-.8\baselineskip
      \nobreak\vskip\z@\@plus1fil
      \penalty100
      \vskip\z@\@plus-1fil
      \kern.8\baselineskip
    \else
      \unvbox\nbsphinxbox
    \fi
}% end of \nbsphinxstopnotebook

% Ensure height of an included graphics fits in nbsphinxfancyoutput frame
\newdimen\nbsphinx@image@maxheight % set in nbsphinxfancyoutput environment
\newcommand*{\nbsphinxincludegraphics}[2][]{%
    \gdef\spx@includegraphics@options{#1}%
    \setbox\spx@image@box\hbox{\includegraphics[#1,draft]{#2}}%
    \in@false
    \ifdim \wd\spx@image@box>\linewidth
      \g@addto@macro\spx@includegraphics@options{,width=\linewidth}%
      \in@true
    \fi
    % no rotation, no need to worry about depth
    \ifdim \ht\spx@image@box>\nbsphinx@image@maxheight
      \g@addto@macro\spx@includegraphics@options{,height=\nbsphinx@image@maxheight}%
      \in@true
    \fi
    \ifin@
      \g@addto@macro\spx@includegraphics@options{,keepaspectratio}%
    \fi
    \setbox\spx@image@box\box\voidb@x % clear memory
    \expandafter\includegraphics\expandafter[\spx@includegraphics@options]{#2}%
}% end of "\MakeFrame"-safe variant of \sphinxincludegraphics
\makeatother

\makeatletter
\renewcommand*\sphinx@verbatim@nolig@list{\do\'\do\`}
\begingroup
\catcode`'=\active
\let\nbsphinx@noligs\@noligs
\g@addto@macro\nbsphinx@noligs{\let'\PYGZsq}
\endgroup
\makeatother
\renewcommand*\sphinxbreaksbeforeactivelist{\do\<\do\"\do\'}
\renewcommand*\sphinxbreaksafteractivelist{\do\.\do\,\do\:\do\;\do\?\do\!\do\/\do\>\do\-}
\makeatletter
\fvset{codes*=\sphinxbreaksattexescapedchars\do\^\^\let\@noligs\nbsphinx@noligs}
\makeatother



\title{Tinkrete}
\date{Apr 23, 2021}
\release{alpha}
\author{Gang Li}
\newcommand{\sphinxlogo}{\vbox{}}
\renewcommand{\releasename}{Release}
\makeindex
\begin{document}

\pagestyle{empty}
\sphinxmaketitle
\pagestyle{plain}
\sphinxtableofcontents
\pagestyle{normal}
\phantomsection\label{\detokenize{index::doc}}


\sphinxAtStartPar
Version:
0.2.2
alpha


\chapter{Introduction}
\label{\detokenize{intro:introduction}}\label{\detokenize{intro::doc}}
\sphinxAtStartPar
Tinkrete is a practical life cycle deterioration modelling framework.
It utilizes the field survey data and provides probabilistic predictions of the RC structure deterioration through different stages of the service life cycle.
It covers various deterioration mechanisms such as membrane deterioration, concrete carbonation and chloride penetration, corrosion and cracking.


\chapter{Modules}
\label{\detokenize{modules:modules}}\label{\detokenize{modules::doc}}

\section{membrane module}
\label{\detokenize{membrane:module-membrane}}\label{\detokenize{membrane:membrane-module}}\label{\detokenize{membrane::doc}}\index{module@\spxentry{module}!membrane@\spxentry{membrane}}\index{membrane@\spxentry{membrane}!module@\spxentry{module}}
\sphinxAtStartPar
\sphinxstylestrong{Summary}

\sphinxAtStartPar
A statistical model is used to predict the probability of failure for the membrane.
\begin{itemize}
\item {} 
\sphinxAtStartPar
\sphinxstylestrong{Resistance}: membrane service life

\item {} 
\sphinxAtStartPar
\sphinxstylestrong{Load}: age

\item {} 
\sphinxAtStartPar
\sphinxstylestrong{limit\sphinxhyphen{}state}: age \textgreater{}= service life.

\end{itemize}
\index{Membrane\_Model (class in membrane)@\spxentry{Membrane\_Model}\spxextra{class in membrane}}

\begin{fulllineitems}
\phantomsection\label{\detokenize{membrane:membrane.Membrane_Model}}\pysiglinewithargsret{\sphinxbfcode{\sphinxupquote{class }}\sphinxcode{\sphinxupquote{membrane.}}\sphinxbfcode{\sphinxupquote{Membrane\_Model}}}{\emph{\DUrole{n}{pars}}}{}
\sphinxAtStartPar
Bases: \sphinxcode{\sphinxupquote{object}}
\index{calibrate() (membrane.Membrane\_Model method)@\spxentry{calibrate()}\spxextra{membrane.Membrane\_Model method}}

\begin{fulllineitems}
\phantomsection\label{\detokenize{membrane:membrane.Membrane_Model.calibrate}}\pysiglinewithargsret{\sphinxbfcode{\sphinxupquote{calibrate}}}{\emph{\DUrole{n}{membrane\_age\_field}}, \emph{\DUrole{n}{membrane\_failure\_ratio\_field}}}{}
\sphinxAtStartPar
calibrate membrane model to field condition
\begin{quote}\begin{description}
\item[{Parameters}] \leavevmode\begin{itemize}
\item {} 
\sphinxAtStartPar
\sphinxstyleliteralstrong{\sphinxupquote{membrane\_age\_field}} (\sphinxstyleliteralemphasis{\sphinxupquote{float}}\sphinxstyleliteralemphasis{\sphinxupquote{, }}\sphinxstyleliteralemphasis{\sphinxupquote{int}}) \textendash{} membrane age when membrane failure rate is surveyed

\item {} 
\sphinxAtStartPar
\sphinxstyleliteralstrong{\sphinxupquote{membrane\_failure\_ratio\_field}} (\sphinxstyleliteralemphasis{\sphinxupquote{float}}) \textendash{} failure rate e.g. 0.1 for 10\%

\end{itemize}

\item[{Returns}] \leavevmode
\sphinxAtStartPar
calibrated model

\item[{Return type}] \leavevmode
\sphinxAtStartPar
membrane model object instance

\end{description}\end{quote}

\end{fulllineitems}

\index{copy() (membrane.Membrane\_Model method)@\spxentry{copy()}\spxextra{membrane.Membrane\_Model method}}

\begin{fulllineitems}
\phantomsection\label{\detokenize{membrane:membrane.Membrane_Model.copy}}\pysiglinewithargsret{\sphinxbfcode{\sphinxupquote{copy}}}{}{}
\sphinxAtStartPar
create a deepcopy

\end{fulllineitems}

\index{membrane\_failure\_with\_year() (membrane.Membrane\_Model method)@\spxentry{membrane\_failure\_with\_year()}\spxextra{membrane.Membrane\_Model method}}

\begin{fulllineitems}
\phantomsection\label{\detokenize{membrane:membrane.Membrane_Model.membrane_failure_with_year}}\pysiglinewithargsret{\sphinxbfcode{\sphinxupquote{membrane\_failure\_with\_year}}}{\emph{\DUrole{n}{year\_lis}}, \emph{\DUrole{n}{plot}\DUrole{o}{=}\DUrole{default_value}{True}}, \emph{\DUrole{n}{amplify}\DUrole{o}{=}\DUrole{default_value}{80}}}{}
\sphinxAtStartPar
solve pf, beta at a list of time steps with plot option

\end{fulllineitems}

\index{postproc() (membrane.Membrane\_Model method)@\spxentry{postproc()}\spxextra{membrane.Membrane\_Model method}}

\begin{fulllineitems}
\phantomsection\label{\detokenize{membrane:membrane.Membrane_Model.postproc}}\pysiglinewithargsret{\sphinxbfcode{\sphinxupquote{postproc}}}{\emph{\DUrole{n}{plot}\DUrole{o}{=}\DUrole{default_value}{False}}}{}
\sphinxAtStartPar
solve pf, beta, attach R distribution with plot option

\end{fulllineitems}

\index{run() (membrane.Membrane\_Model method)@\spxentry{run()}\spxextra{membrane.Membrane\_Model method}}

\begin{fulllineitems}
\phantomsection\label{\detokenize{membrane:membrane.Membrane_Model.run}}\pysiglinewithargsret{\sphinxbfcode{\sphinxupquote{run}}}{\emph{\DUrole{n}{t}}}{}
\sphinxAtStartPar
attach the resistance: membrane age

\end{fulllineitems}


\end{fulllineitems}

\index{Pf\_RS\_special() (in module membrane)@\spxentry{Pf\_RS\_special()}\spxextra{in module membrane}}

\begin{fulllineitems}
\phantomsection\label{\detokenize{membrane:membrane.Pf_RS_special}}\pysiglinewithargsret{\sphinxcode{\sphinxupquote{membrane.}}\sphinxbfcode{\sphinxupquote{Pf\_RS\_special}}}{\emph{\DUrole{n}{R\_info}}, \emph{\DUrole{n}{S}}, \emph{\DUrole{n}{R\_distrib\_type}\DUrole{o}{=}\DUrole{default_value}{\textquotesingle{}normal\textquotesingle{}}}, \emph{\DUrole{n}{plot}\DUrole{o}{=}\DUrole{default_value}{False}}}{}
\sphinxAtStartPar
special case of helper\_fuc.Pf\_RS, here the “load” S is a number and it calculates the probability of failure  Pf = P(R\sphinxhyphen{}S\textless{}0), given the R(resistance) and S(load)
with three three methods and use method 3 if it is checked “OK” with the other two
\begin{enumerate}
\sphinxsetlistlabels{\arabic}{enumi}{enumii}{}{.}%
\item {} 
\sphinxAtStartPar
crude monte carlo

\item {} 
\sphinxAtStartPar
numerical integral of g kernel fit

\item {} 
\sphinxAtStartPar
R S integral: \(F_R(S)\), reliability index(beta factor) is calculated with simple 1st order g.mean()/g.std()

\end{enumerate}
\begin{quote}\begin{description}
\item[{Parameters}] \leavevmode\begin{itemize}
\item {} 
\sphinxAtStartPar
\sphinxstyleliteralstrong{\sphinxupquote{R\_info}} (\sphinxstyleliteralemphasis{\sphinxupquote{tuple}}) \textendash{} distribution of Resistance, for this specicial case, the membrane service life.
R\_distrib\_type=’normal’ \sphinxhyphen{}\textgreater{} tuple(m,s) for normal m: mean s: standard deviation
other distribution form will be ignored.

\item {} 
\sphinxAtStartPar
\sphinxstyleliteralstrong{\sphinxupquote{S}} (\sphinxstyleliteralemphasis{\sphinxupquote{numpy array}}) \textendash{} distribution of load, for this special case, the membrane age.

\item {} 
\sphinxAtStartPar
\sphinxstyleliteralstrong{\sphinxupquote{R\_distrib\_type}} (\sphinxstyleliteralemphasis{\sphinxupquote{str}}\sphinxstyleliteralemphasis{\sphinxupquote{, }}\sphinxstyleliteralemphasis{\sphinxupquote{optional}}) \textendash{} by default ‘normal’

\item {} 
\sphinxAtStartPar
\sphinxstyleliteralstrong{\sphinxupquote{plot}} (\sphinxstyleliteralemphasis{\sphinxupquote{bool}}\sphinxstyleliteralemphasis{\sphinxupquote{, }}\sphinxstyleliteralemphasis{\sphinxupquote{optional}}) \textendash{} plot distribution, by default False

\end{itemize}

\item[{Returns}] \leavevmode
\sphinxAtStartPar
(probability of failure, beta factor)

\item[{Return type}] \leavevmode
\sphinxAtStartPar
tuple

\end{description}\end{quote}

\begin{sphinxadmonition}{note}{Note:}
\sphinxAtStartPar
R\_info only supports two\sphinxhyphen{}parameter normal distribution.
\end{sphinxadmonition}

\end{fulllineitems}

\index{RS\_plot\_special() (in module membrane)@\spxentry{RS\_plot\_special()}\spxextra{in module membrane}}

\begin{fulllineitems}
\phantomsection\label{\detokenize{membrane:membrane.RS_plot_special}}\pysiglinewithargsret{\sphinxcode{\sphinxupquote{membrane.}}\sphinxbfcode{\sphinxupquote{RS\_plot\_special}}}{\emph{\DUrole{n}{model}}, \emph{\DUrole{n}{ax}\DUrole{o}{=}\DUrole{default_value}{None}}, \emph{\DUrole{n}{t\_offset}\DUrole{o}{=}\DUrole{default_value}{0}}, \emph{\DUrole{n}{amplify}\DUrole{o}{=}\DUrole{default_value}{1}}}{}
\sphinxAtStartPar
plot R S distribution vertically at a time to an axis special case: S is a number.
\begin{quote}\begin{description}
\item[{Parameters}] \leavevmode\begin{itemize}
\item {} 
\sphinxAtStartPar
\sphinxstyleliteralstrong{\sphinxupquote{model}} (\sphinxstyleliteralemphasis{\sphinxupquote{model object instance}}) \textendash{} \begin{itemize}
\item {} 
\sphinxAtStartPar
model.R\_distrib : scipy.stats.\_continuous\_distns, normal or beta {[}calculated in Pf\_RS() through model.postproc(){]}

\item {} 
\sphinxAtStartPar
model.S : single number for this special case

\end{itemize}


\item {} 
\sphinxAtStartPar
\sphinxstyleliteralstrong{\sphinxupquote{ax}} (\sphinxstyleliteralemphasis{\sphinxupquote{axes}}) \textendash{} subplot axis

\item {} 
\sphinxAtStartPar
\sphinxstyleliteralstrong{\sphinxupquote{t\_offset}} (\sphinxstyleliteralemphasis{\sphinxupquote{int}}\sphinxstyleliteralemphasis{\sphinxupquote{, }}\sphinxstyleliteralemphasis{\sphinxupquote{float}}) \textendash{} time offset to move the plot along the t\sphinxhyphen{}axis. default is zero

\item {} 
\sphinxAtStartPar
\sphinxstyleliteralstrong{\sphinxupquote{amplify}} (\sphinxstyleliteralemphasis{\sphinxupquote{int}}) \textendash{} scale the height of the pdf plot

\end{itemize}

\end{description}\end{quote}

\end{fulllineitems}

\index{calibrate\_f() (in module membrane)@\spxentry{calibrate\_f()}\spxextra{in module membrane}}

\begin{fulllineitems}
\phantomsection\label{\detokenize{membrane:membrane.calibrate_f}}\pysiglinewithargsret{\sphinxcode{\sphinxupquote{membrane.}}\sphinxbfcode{\sphinxupquote{calibrate\_f}}}{\emph{\DUrole{n}{model\_raw}}, \emph{\DUrole{n}{t}}, \emph{\DUrole{n}{membrane\_failure\_ratio\_field}}, \emph{\DUrole{n}{tol}\DUrole{o}{=}\DUrole{default_value}{1e\sphinxhyphen{}06}}, \emph{\DUrole{n}{max\_count}\DUrole{o}{=}\DUrole{default_value}{100}}, \emph{\DUrole{n}{print\_out}\DUrole{o}{=}\DUrole{default_value}{True}}}{}
\sphinxAtStartPar
calibrate membrane model to field condition by finding the corresponding membrane service life std that matches the failure ratio in the field
\begin{quote}\begin{description}
\item[{Parameters}] \leavevmode\begin{itemize}
\item {} 
\sphinxAtStartPar
\sphinxstyleliteralstrong{\sphinxupquote{model\_raw}} (\sphinxstyleliteralemphasis{\sphinxupquote{model instance}}) \textendash{} model to be calibrated

\item {} 
\sphinxAtStartPar
\sphinxstyleliteralstrong{\sphinxupquote{t}} (\sphinxstyleliteralemphasis{\sphinxupquote{int}}\sphinxstyleliteralemphasis{\sphinxupquote{, }}\sphinxstyleliteralemphasis{\sphinxupquote{float}}) \textendash{} membrane age when membrane failure rate is surveyed {[}year{]}

\item {} 
\sphinxAtStartPar
\sphinxstyleliteralstrong{\sphinxupquote{membrane\_failure\_ratio\_field}} (\sphinxstyleliteralemphasis{\sphinxupquote{float}}) \textendash{} failure rate e.g. 0.1 for 10\%

\item {} 
\sphinxAtStartPar
\sphinxstyleliteralstrong{\sphinxupquote{tol}} (\sphinxstyleliteralemphasis{\sphinxupquote{float}}\sphinxstyleliteralemphasis{\sphinxupquote{, }}\sphinxstyleliteralemphasis{\sphinxupquote{optional}}) \textendash{} optimization tolerance, by default 1e\sphinxhyphen{}6

\item {} 
\sphinxAtStartPar
\sphinxstyleliteralstrong{\sphinxupquote{max\_count}} (\sphinxstyleliteralemphasis{\sphinxupquote{int}}\sphinxstyleliteralemphasis{\sphinxupquote{, }}\sphinxstyleliteralemphasis{\sphinxupquote{optional}}) \textendash{} optimization max iteration number, by default 100

\item {} 
\sphinxAtStartPar
\sphinxstyleliteralstrong{\sphinxupquote{print\_out}} (\sphinxstyleliteralemphasis{\sphinxupquote{bool}}\sphinxstyleliteralemphasis{\sphinxupquote{, }}\sphinxstyleliteralemphasis{\sphinxupquote{optional}}) \textendash{} if True print out model vs field compare, by default True

\end{itemize}

\item[{Returns}] \leavevmode
\sphinxAtStartPar
calibrated model

\item[{Return type}] \leavevmode
\sphinxAtStartPar
membrane model object instance

\end{description}\end{quote}

\end{fulllineitems}

\index{membrane\_age() (in module membrane)@\spxentry{membrane\_age()}\spxextra{in module membrane}}

\begin{fulllineitems}
\phantomsection\label{\detokenize{membrane:membrane.membrane_age}}\pysiglinewithargsret{\sphinxcode{\sphinxupquote{membrane.}}\sphinxbfcode{\sphinxupquote{membrane\_age}}}{\emph{\DUrole{n}{t}}}{}
\sphinxAtStartPar
return the membrane age as the “resistance”
\begin{quote}\begin{description}
\item[{Parameters}] \leavevmode
\sphinxAtStartPar
\sphinxstyleliteralstrong{\sphinxupquote{t}} (\sphinxstyleliteralemphasis{\sphinxupquote{int}}\sphinxstyleliteralemphasis{\sphinxupquote{, }}\sphinxstyleliteralemphasis{\sphinxupquote{float}}) \textendash{} membrane age

\item[{Returns}] \leavevmode
\sphinxAtStartPar
membrane age

\item[{Return type}] \leavevmode
\sphinxAtStartPar
int, float

\end{description}\end{quote}

\begin{sphinxadmonition}{note}{Note:}
\sphinxAtStartPar
This function is a placeholder for more complex age input
\end{sphinxadmonition}

\end{fulllineitems}

\index{membrane\_failure\_year() (in module membrane)@\spxentry{membrane\_failure\_year()}\spxextra{in module membrane}}

\begin{fulllineitems}
\phantomsection\label{\detokenize{membrane:membrane.membrane_failure_year}}\pysiglinewithargsret{\sphinxcode{\sphinxupquote{membrane.}}\sphinxbfcode{\sphinxupquote{membrane\_failure\_year}}}{\emph{\DUrole{n}{model}}, \emph{\DUrole{n}{year\_lis}}, \emph{\DUrole{n}{plot}\DUrole{o}{=}\DUrole{default_value}{True}}, \emph{\DUrole{n}{amplify}\DUrole{o}{=}\DUrole{default_value}{30}}}{}
\sphinxAtStartPar
membrane\_failure\_year: run model over a list of time steps
\begin{quote}\begin{description}
\item[{Parameters}] \leavevmode\begin{itemize}
\item {} 
\sphinxAtStartPar
\sphinxstyleliteralstrong{\sphinxupquote{model}} (\sphinxstyleliteralemphasis{\sphinxupquote{class instance}}) \textendash{} Membrane\_model class instance

\item {} 
\sphinxAtStartPar
\sphinxstyleliteralstrong{\sphinxupquote{year\_lis}} (\sphinxstyleliteralemphasis{\sphinxupquote{list}}\sphinxstyleliteralemphasis{\sphinxupquote{, }}\sphinxstyleliteralemphasis{\sphinxupquote{array\sphinxhyphen{}like}}) \textendash{} a list of time steps

\item {} 
\sphinxAtStartPar
\sphinxstyleliteralstrong{\sphinxupquote{plot}} (\sphinxstyleliteralemphasis{\sphinxupquote{bool}}\sphinxstyleliteralemphasis{\sphinxupquote{, }}\sphinxstyleliteralemphasis{\sphinxupquote{optional}}) \textendash{} if True, plot the Pf, beta, R S distribution, by default True

\item {} 
\sphinxAtStartPar
\sphinxstyleliteralstrong{\sphinxupquote{amplify}} (\sphinxstyleliteralemphasis{\sphinxupquote{int}}\sphinxstyleliteralemphasis{\sphinxupquote{, }}\sphinxstyleliteralemphasis{\sphinxupquote{optional}}) \textendash{} the arbitrary comparable size of the distribution curve, by default 80

\end{itemize}

\item[{Returns}] \leavevmode
\sphinxAtStartPar
(pf list , beta list)

\item[{Return type}] \leavevmode
\sphinxAtStartPar
tuple

\end{description}\end{quote}

\end{fulllineitems}

\index{membrane\_life() (in module membrane)@\spxentry{membrane\_life()}\spxextra{in module membrane}}

\begin{fulllineitems}
\phantomsection\label{\detokenize{membrane:membrane.membrane_life}}\pysiglinewithargsret{\sphinxcode{\sphinxupquote{membrane.}}\sphinxbfcode{\sphinxupquote{membrane\_life}}}{\emph{\DUrole{n}{pars}}}{}
\sphinxAtStartPar
calculate the mean value of the service life from the manufacture’s service life label(eg. 30 years with 95\% confidence)
with the given standard deviation
\begin{quote}\begin{description}
\item[{Parameters}] \leavevmode
\sphinxAtStartPar
\sphinxstyleliteralstrong{\sphinxupquote{pars}} (\sphinxstyleliteralemphasis{\sphinxupquote{parameter object instance}}) \textendash{} raw parameters
pars.life\_product\_label\_life
pars.life\_confidence
pars.life\_std

\item[{Returns}] \leavevmode
\sphinxAtStartPar
service life mean value

\item[{Return type}] \leavevmode
\sphinxAtStartPar
float

\end{description}\end{quote}

\end{fulllineitems}



\section{carbonation module}
\label{\detokenize{carbonation:module-carbonation}}\label{\detokenize{carbonation:carbonation-module}}\label{\detokenize{carbonation::doc}}\index{module@\spxentry{module}!carbonation@\spxentry{carbonation}}\index{carbonation@\spxentry{carbonation}!module@\spxentry{module}}
\sphinxAtStartPar
\sphinxstylestrong{Summary}

\sphinxAtStartPar
Modified analytical solution of Fick’s law (square root of time)

\sphinxAtStartPar
Proportional constant is modified by material properties and exposure environments
\begin{itemize}
\item {} 
\sphinxAtStartPar
\sphinxstylestrong{Resistance}:       cover depth

\item {} 
\sphinxAtStartPar
\sphinxstylestrong{Load}:             carbonation depth

\item {} 
\sphinxAtStartPar
\sphinxstylestrong{limit\sphinxhyphen{}state}:      carbonation depth \textgreater{}= cover depth

\item {} 
\sphinxAtStartPar
\sphinxstylestrong{Field data}:       carbonation depths (repeated measurements)

\end{itemize}
\index{C\_S() (in module carbonation)@\spxentry{C\_S()}\spxextra{in module carbonation}}

\begin{fulllineitems}
\phantomsection\label{\detokenize{carbonation:carbonation.C_S}}\pysiglinewithargsret{\sphinxcode{\sphinxupquote{carbonation.}}\sphinxbfcode{\sphinxupquote{C\_S}}}{\emph{\DUrole{n}{C\_S\_emi}\DUrole{o}{=}\DUrole{default_value}{0}}}{}
\sphinxAtStartPar
Calculate CO2 density{[}kg/m\textasciicircum{}3{]} in the environment, it is about 350\sphinxhyphen{}380 ppm in the atm plus other source or sink
\begin{quote}\begin{description}
\item[{Parameters}] \leavevmode
\sphinxAtStartPar
\sphinxstyleliteralstrong{\sphinxupquote{C\_S\_emi}} (\sphinxstyleliteralemphasis{\sphinxupquote{additional emission}}\sphinxstyleliteralemphasis{\sphinxupquote{, }}\sphinxstyleliteralemphasis{\sphinxupquote{positive}}\sphinxstyleliteralemphasis{\sphinxupquote{ or }}\sphinxstyleliteralemphasis{\sphinxupquote{negative}}\sphinxstyleliteralemphasis{\sphinxupquote{(}}\sphinxstyleliteralemphasis{\sphinxupquote{sink}}\sphinxstyleliteralemphasis{\sphinxupquote{)}}\sphinxstyleliteralemphasis{\sphinxupquote{, }}\sphinxstyleliteralemphasis{\sphinxupquote{default is 0}}) \textendash{} 

\end{description}\end{quote}

\end{fulllineitems}

\index{Carb\_depth() (in module carbonation)@\spxentry{Carb\_depth()}\spxextra{in module carbonation}}

\begin{fulllineitems}
\phantomsection\label{\detokenize{carbonation:carbonation.Carb_depth}}\pysiglinewithargsret{\sphinxcode{\sphinxupquote{carbonation.}}\sphinxbfcode{\sphinxupquote{Carb\_depth}}}{\emph{\DUrole{n}{t}}, \emph{\DUrole{n}{pars}}}{}
\sphinxAtStartPar
Master model function, calculate carbonation depth and the k constant of sqrt of time from all the
parameters. The derived parameters is also calculated within this funcion. Caution: The pars instance is mutable,
so a deepcopy of the original instance should be used if the calculation is not intended for “inplace”.
\begin{quote}\begin{description}
\item[{Parameters}] \leavevmode\begin{itemize}
\item {} 
\sphinxAtStartPar
\sphinxstyleliteralstrong{\sphinxupquote{t}} (\sphinxstyleliteralemphasis{\sphinxupquote{time}}\sphinxstyleliteralemphasis{\sphinxupquote{ {[}}}\sphinxstyleliteralemphasis{\sphinxupquote{year}}\sphinxstyleliteralemphasis{\sphinxupquote{{]}}}) \textendash{} 

\item {} 
\sphinxAtStartPar
\sphinxstyleliteralstrong{\sphinxupquote{pars}} (\sphinxstyleliteralemphasis{\sphinxupquote{object/instance of wrapper class}}\sphinxstyleliteralemphasis{\sphinxupquote{(}}\sphinxstyleliteralemphasis{\sphinxupquote{empty class}}\sphinxstyleliteralemphasis{\sphinxupquote{)}}) \textendash{} a wrapper of all material and environmental parameters deep\sphinxhyphen{}copied from the raw data

\end{itemize}

\item[{Returns}] \leavevmode
\sphinxAtStartPar
\sphinxstylestrong{out}

\item[{Return type}] \leavevmode
\sphinxAtStartPar
carbionation depth at the time t {[}mm{]}

\end{description}\end{quote}

\begin{sphinxadmonition}{note}{Note:}
\sphinxAtStartPar
intermediate parameters calculated and attached to

\sphinxAtStartPar
pars k\_e : environmental function {[}\sphinxhyphen{}{]}

\sphinxAtStartPar
k\_c : execution transfer parameter {[}\sphinxhyphen{}{]}

\sphinxAtStartPar
account for curing measures

\sphinxAtStartPar
k\_t : regression parameter {[}\sphinxhyphen{}{]}

\sphinxAtStartPar
R\_ACC\_0\_inv: inverse effective carbonation resistance of concrete(accelerated) {[}(mm\textasciicircum{}2/year)/(kg/m\textasciicircum{}3){]} eps\_t  : error term {[}\sphinxhyphen{}{]}

\sphinxAtStartPar
C\_S    : CO2 concentration {[}\(kg/m^3\){]}

\sphinxAtStartPar
W\_t    : weather function {[}\sphinxhyphen{}{]}

\sphinxAtStartPar
k      : constant before the sqrt of time(time{[}year{]}, carbonation depth{[}mm{]}) {[}mm/year\textasciicircum{}0.5{]}
typical value of k =3\textasciitilde{}4 for unit mm,year {[}\sphinxurl{https://www.researchgate.net/publication/272174090\_Carbonation\_Coefficient\_of\_Concrete\_in\_Dhaka\_City}{]}
\end{sphinxadmonition}

\end{fulllineitems}

\index{Carbonation\_Model (class in carbonation)@\spxentry{Carbonation\_Model}\spxextra{class in carbonation}}

\begin{fulllineitems}
\phantomsection\label{\detokenize{carbonation:carbonation.Carbonation_Model}}\pysiglinewithargsret{\sphinxbfcode{\sphinxupquote{class }}\sphinxcode{\sphinxupquote{carbonation.}}\sphinxbfcode{\sphinxupquote{Carbonation\_Model}}}{\emph{\DUrole{n}{pars}}}{}
\sphinxAtStartPar
Bases: \sphinxcode{\sphinxupquote{object}}
\index{calibrate() (carbonation.Carbonation\_Model method)@\spxentry{calibrate()}\spxextra{carbonation.Carbonation\_Model method}}

\begin{fulllineitems}
\phantomsection\label{\detokenize{carbonation:carbonation.Carbonation_Model.calibrate}}\pysiglinewithargsret{\sphinxbfcode{\sphinxupquote{calibrate}}}{\emph{\DUrole{n}{t}}, \emph{\DUrole{n}{carb\_depth\_field}}, \emph{\DUrole{n}{print\_out}\DUrole{o}{=}\DUrole{default_value}{False}}}{}
\sphinxAtStartPar
return a new model instance with calibrated param

\end{fulllineitems}

\index{carb\_with\_year() (carbonation.Carbonation\_Model method)@\spxentry{carb\_with\_year()}\spxextra{carbonation.Carbonation\_Model method}}

\begin{fulllineitems}
\phantomsection\label{\detokenize{carbonation:carbonation.Carbonation_Model.carb_with_year}}\pysiglinewithargsret{\sphinxbfcode{\sphinxupquote{carb\_with\_year}}}{\emph{\DUrole{n}{year\_lis}}, \emph{\DUrole{n}{plot}\DUrole{o}{=}\DUrole{default_value}{True}}, \emph{\DUrole{n}{amplify}\DUrole{o}{=}\DUrole{default_value}{80}}}{}
\end{fulllineitems}

\index{copy() (carbonation.Carbonation\_Model method)@\spxentry{copy()}\spxextra{carbonation.Carbonation\_Model method}}

\begin{fulllineitems}
\phantomsection\label{\detokenize{carbonation:carbonation.Carbonation_Model.copy}}\pysiglinewithargsret{\sphinxbfcode{\sphinxupquote{copy}}}{}{}
\end{fulllineitems}

\index{postproc() (carbonation.Carbonation\_Model method)@\spxentry{postproc()}\spxextra{carbonation.Carbonation\_Model method}}

\begin{fulllineitems}
\phantomsection\label{\detokenize{carbonation:carbonation.Carbonation_Model.postproc}}\pysiglinewithargsret{\sphinxbfcode{\sphinxupquote{postproc}}}{\emph{\DUrole{n}{plot}\DUrole{o}{=}\DUrole{default_value}{False}}}{}
\end{fulllineitems}

\index{run() (carbonation.Carbonation\_Model method)@\spxentry{run()}\spxextra{carbonation.Carbonation\_Model method}}

\begin{fulllineitems}
\phantomsection\label{\detokenize{carbonation:carbonation.Carbonation_Model.run}}\pysiglinewithargsret{\sphinxbfcode{\sphinxupquote{run}}}{\emph{\DUrole{n}{t}}}{}
\sphinxAtStartPar
t{[}year{]}

\end{fulllineitems}


\end{fulllineitems}

\index{R\_ACC\_0\_inv() (in module carbonation)@\spxentry{R\_ACC\_0\_inv()}\spxextra{in module carbonation}}

\begin{fulllineitems}
\phantomsection\label{\detokenize{carbonation:carbonation.R_ACC_0_inv}}\pysiglinewithargsret{\sphinxcode{\sphinxupquote{carbonation.}}\sphinxbfcode{\sphinxupquote{R\_ACC\_0\_inv}}}{\emph{\DUrole{n}{pars}}}{}~\begin{description}
\item[{Calculate R\_ACC\_0\_inv{[}(mm\textasciicircum{}2/year)/(kg/m\textasciicircum{}3){]}, the inverse effective carbonation resistance of concrete(accelerated)}] \leavevmode
\sphinxAtStartPar
From ACC test or from existion empirical data interpolation for orientation purpose
test condition: duration time = 56 days CO2 = 2.0 vol\%, T =25 degC RH\_ref =65

\end{description}
\begin{quote}\begin{description}
\item[{Parameters}] \leavevmode\begin{itemize}
\item {} 
\sphinxAtStartPar
\sphinxstyleliteralstrong{\sphinxupquote{pars.x\_c}} (\sphinxstyleliteralemphasis{\sphinxupquote{float}}) \textendash{} measured carbonation depth in the accelerated test{[}m{]}

\item {} 
\sphinxAtStartPar
\sphinxstyleliteralstrong{\sphinxupquote{pars.option.choose}} (\sphinxstyleliteralemphasis{\sphinxupquote{bool}}) \textendash{} if true \sphinxhyphen{}\textgreater{} choose to use interpolation method

\item {} 
\sphinxAtStartPar
\sphinxstyleliteralstrong{\sphinxupquote{pars.option.df\_R\_ACC}} (\sphinxstyleliteralemphasis{\sphinxupquote{pd.dataframe}}) \textendash{} data table for interpolate, loaded by function load\_df\_R\_ACC, interpolated by function interp\_extrap\_f

\end{itemize}

\item[{Returns}] \leavevmode
\sphinxAtStartPar
\sphinxstylestrong{out} \textendash{} parameter value with sample number = N\_SAMPLE(defined globally)

\item[{Return type}] \leavevmode
\sphinxAtStartPar
numpy arrays

\end{description}\end{quote}
\subsubsection*{Notes}

\sphinxAtStartPar
Pay special attention to the units in the source code

\end{fulllineitems}

\index{W\_t() (in module carbonation)@\spxentry{W\_t()}\spxextra{in module carbonation}}

\begin{fulllineitems}
\phantomsection\label{\detokenize{carbonation:carbonation.W_t}}\pysiglinewithargsret{\sphinxcode{\sphinxupquote{carbonation.}}\sphinxbfcode{\sphinxupquote{W\_t}}}{\emph{\DUrole{n}{t}}, \emph{\DUrole{n}{pars}}}{}
\sphinxAtStartPar
Calculate weather function W\_t, a parameter considering the meso\sphinxhyphen{}climatic conditions due to wetting events of concrete surface
\begin{quote}\begin{description}
\item[{Parameters}] \leavevmode\begin{itemize}
\item {} 
\sphinxAtStartPar
\sphinxstyleliteralstrong{\sphinxupquote{pars.ToW}} (\sphinxstyleliteralemphasis{\sphinxupquote{time of wetness}}\sphinxstyleliteralemphasis{\sphinxupquote{ {[}}}\sphinxstyleliteralemphasis{\sphinxupquote{\sphinxhyphen{}}}\sphinxstyleliteralemphasis{\sphinxupquote{{]}}}) \textendash{} ToW = (days with rainfall h\_Nd \textgreater{}= 2.5 mm per day)/365

\item {} 
\sphinxAtStartPar
\sphinxstyleliteralstrong{\sphinxupquote{pars.p\_SR}} (\sphinxstyleliteralemphasis{\sphinxupquote{probability of driving rain}}\sphinxstyleliteralemphasis{\sphinxupquote{ {[}}}\sphinxstyleliteralemphasis{\sphinxupquote{\sphinxhyphen{}}}\sphinxstyleliteralemphasis{\sphinxupquote{{]}}}) \textendash{} Vertical \sphinxhyphen{}\textgreater{} weather station
Horizontal 1.0
Interior 0.0

\item {} 
\sphinxAtStartPar
\sphinxstyleliteralstrong{\sphinxupquote{exponent of regression}}\sphinxstyleliteralstrong{\sphinxupquote{ {[}}}\sphinxstyleliteralstrong{\sphinxupquote{\sphinxhyphen{}}}\sphinxstyleliteralstrong{\sphinxupquote{{]} }}\sphinxstyleliteralstrong{\sphinxupquote{ND}}\sphinxstyleliteralstrong{\sphinxupquote{(}}\sphinxstyleliteralstrong{\sphinxupquote{0.446}} (\sphinxstyleliteralemphasis{\sphinxupquote{pars.b\_w;}}) \textendash{} 

\item {} 
\sphinxAtStartPar
\sphinxstyleliteralstrong{\sphinxupquote{0.163}}\sphinxstyleliteralstrong{\sphinxupquote{)}} \textendash{} 

\item {} 
\sphinxAtStartPar
\sphinxstyleliteralstrong{\sphinxupquote{param t\_0}} (\sphinxstyleliteralemphasis{\sphinxupquote{built\sphinxhyphen{}in}}) \textendash{} 

\end{itemize}

\item[{Returns}] \leavevmode
\sphinxAtStartPar
\sphinxstylestrong{out}

\item[{Return type}] \leavevmode
\sphinxAtStartPar
numpy array

\end{description}\end{quote}

\end{fulllineitems}

\index{calibrate\_f() (in module carbonation)@\spxentry{calibrate\_f()}\spxextra{in module carbonation}}

\begin{fulllineitems}
\phantomsection\label{\detokenize{carbonation:carbonation.calibrate_f}}\pysiglinewithargsret{\sphinxcode{\sphinxupquote{carbonation.}}\sphinxbfcode{\sphinxupquote{calibrate\_f}}}{\emph{\DUrole{n}{model\_raw}}, \emph{\DUrole{n}{t}}, \emph{\DUrole{n}{carb\_depth\_field}}, \emph{\DUrole{n}{tol}\DUrole{o}{=}\DUrole{default_value}{1e\sphinxhyphen{}06}}, \emph{\DUrole{n}{max\_count}\DUrole{o}{=}\DUrole{default_value}{50}}, \emph{\DUrole{n}{print\_out}\DUrole{o}{=}\DUrole{default_value}{True}}}{}
\sphinxAtStartPar
carb\_depth\_field{[}mm{]}\sphinxhyphen{}\textgreater{} find corresponding x\_c(accelerated test carb depth{[}m{]})
Calibrate the carbonation model with field carbonation test data and return the new calibrated model object/instance
Optimization method: searching for the best accelerated test carbonation depth x\_c{[}m{]} so the model matches field data
on the mean value of the carbonation depth)
\begin{quote}\begin{description}
\item[{Parameters}] \leavevmode\begin{itemize}
\item {} 
\sphinxAtStartPar
\sphinxstyleliteralstrong{\sphinxupquote{model\_raw}} (\sphinxstyleliteralemphasis{\sphinxupquote{object/instance of Carbonation\_Model class}}\sphinxstyleliteralemphasis{\sphinxupquote{, }}\sphinxstyleliteralemphasis{\sphinxupquote{mutable}}\sphinxstyleliteralemphasis{\sphinxupquote{, }}\sphinxstyleliteralemphasis{\sphinxupquote{so a deepcopy will be used in this function}}) \textendash{} 

\item {} 
\sphinxAtStartPar
\sphinxstyleliteralstrong{\sphinxupquote{t}} (\sphinxstyleliteralemphasis{\sphinxupquote{float}}\sphinxstyleliteralemphasis{\sphinxupquote{ or }}\sphinxstyleliteralemphasis{\sphinxupquote{int}}) \textendash{} survey time, age of the concrete{[}year{]}

\item {} 
\sphinxAtStartPar
\sphinxstyleliteralstrong{\sphinxupquote{carb\_depth\_field}} (\sphinxstyleliteralemphasis{\sphinxupquote{numpy array}}) \textendash{} at time t, field carbonation depths{[}mm{]}

\item {} 
\sphinxAtStartPar
\sphinxstyleliteralstrong{\sphinxupquote{tol}} (\sphinxstyleliteralemphasis{\sphinxupquote{float}}) \textendash{} accelerated carbonation depth x\_c optimization tolerance, default is 1e\sphinxhyphen{}5 {[}mm{]}

\item {} 
\sphinxAtStartPar
\sphinxstyleliteralstrong{\sphinxupquote{max\_count}} (\sphinxstyleliteralemphasis{\sphinxupquote{int}}) \textendash{} maximun number of searching iteration, default is 50

\end{itemize}

\item[{Returns}] \leavevmode
\sphinxAtStartPar
\sphinxstylestrong{out} \textendash{} new calibrated model

\item[{Return type}] \leavevmode
\sphinxAtStartPar
object/instance of Carbonation\_Model class

\end{description}\end{quote}

\end{fulllineitems}

\index{carb\_year() (in module carbonation)@\spxentry{carb\_year()}\spxextra{in module carbonation}}

\begin{fulllineitems}
\phantomsection\label{\detokenize{carbonation:carbonation.carb_year}}\pysiglinewithargsret{\sphinxcode{\sphinxupquote{carbonation.}}\sphinxbfcode{\sphinxupquote{carb\_year}}}{\emph{\DUrole{n}{model}}, \emph{\DUrole{n}{year\_lis}}, \emph{\DUrole{n}{plot}\DUrole{o}{=}\DUrole{default_value}{True}}, \emph{\DUrole{n}{amplify}\DUrole{o}{=}\DUrole{default_value}{80}}}{}
\sphinxAtStartPar
run model over time

\end{fulllineitems}

\index{eps\_t() (in module carbonation)@\spxentry{eps\_t()}\spxextra{in module carbonation}}

\begin{fulllineitems}
\phantomsection\label{\detokenize{carbonation:carbonation.eps_t}}\pysiglinewithargsret{\sphinxcode{\sphinxupquote{carbonation.}}\sphinxbfcode{\sphinxupquote{eps\_t}}}{}{}
\sphinxAtStartPar
Calculate error term, eps\_t{[}(mm\textasciicircum{}2/years)/(kg/m\textasciicircum{}3){]},
considering inaccuracies which occur conditionally when using the ACC test method  k\_t{[}\sphinxhyphen{}{]}
\subsubsection*{Notes}

\sphinxAtStartPar
for R\_ACC\_0\_inv{[}(mm\textasciicircum{}2/years)/(kg/m\textasciicircum{}3){]}

\end{fulllineitems}

\index{k\_c() (in module carbonation)@\spxentry{k\_c()}\spxextra{in module carbonation}}

\begin{fulllineitems}
\phantomsection\label{\detokenize{carbonation:carbonation.k_c}}\pysiglinewithargsret{\sphinxcode{\sphinxupquote{carbonation.}}\sphinxbfcode{\sphinxupquote{k\_c}}}{\emph{\DUrole{n}{pars}}}{}
\sphinxAtStartPar
calculate k\_c: execution transfer parameter {[}\sphinxhyphen{}{]}, effect of period of curing for the accelerated carbonation test
\begin{quote}\begin{description}
\item[{Parameters}] \leavevmode\begin{itemize}
\item {} 
\sphinxAtStartPar
\sphinxstyleliteralstrong{\sphinxupquote{pars.t\_c}} (\sphinxstyleliteralemphasis{\sphinxupquote{period of curing}}\sphinxstyleliteralemphasis{\sphinxupquote{ {[}}}\sphinxstyleliteralemphasis{\sphinxupquote{d}}\sphinxstyleliteralemphasis{\sphinxupquote{{]}}}) \textendash{} constant

\item {} 
\sphinxAtStartPar
\sphinxstyleliteralstrong{\sphinxupquote{b\_c}} (\sphinxstyleliteralemphasis{\sphinxupquote{exponent of regression}}\sphinxstyleliteralemphasis{\sphinxupquote{ {[}}}\sphinxstyleliteralemphasis{\sphinxupquote{\sphinxhyphen{}}}\sphinxstyleliteralemphasis{\sphinxupquote{{]}}}) \textendash{} \begin{description}
\item[{normal distribution, m: \sphinxhyphen{}0.567}] \leavevmode
\sphinxAtStartPar
s: 0.024

\end{description}


\end{itemize}

\end{description}\end{quote}

\end{fulllineitems}

\index{k\_e() (in module carbonation)@\spxentry{k\_e()}\spxextra{in module carbonation}}

\begin{fulllineitems}
\phantomsection\label{\detokenize{carbonation:carbonation.k_e}}\pysiglinewithargsret{\sphinxcode{\sphinxupquote{carbonation.}}\sphinxbfcode{\sphinxupquote{k\_e}}}{\emph{\DUrole{n}{pars}}}{}
\sphinxAtStartPar
Calculate k\_e{[}\sphinxhyphen{}{]}, environmental factor, effect of relative humidity
\begin{quote}\begin{description}
\item[{Parameters}] \leavevmode\begin{itemize}
\item {} 
\sphinxAtStartPar
\sphinxstyleliteralstrong{\sphinxupquote{pars.RH\_ref}} (\sphinxstyleliteralemphasis{\sphinxupquote{65}}\sphinxstyleliteralemphasis{\sphinxupquote{ {[}}}\sphinxstyleliteralemphasis{\sphinxupquote{\%}}\sphinxstyleliteralemphasis{\sphinxupquote{{]}}}) \textendash{} 

\item {} 
\sphinxAtStartPar
\sphinxstyleliteralstrong{\sphinxupquote{g\_e}} (\sphinxstyleliteralemphasis{\sphinxupquote{2.5}}\sphinxstyleliteralemphasis{\sphinxupquote{ {[}}}\sphinxstyleliteralemphasis{\sphinxupquote{\sphinxhyphen{}}}\sphinxstyleliteralemphasis{\sphinxupquote{{]}}}) \textendash{} 

\item {} 
\sphinxAtStartPar
\sphinxstyleliteralstrong{\sphinxupquote{f\_e}} (\sphinxstyleliteralemphasis{\sphinxupquote{5.0}}\sphinxstyleliteralemphasis{\sphinxupquote{ {[}}}\sphinxstyleliteralemphasis{\sphinxupquote{\sphinxhyphen{}}}\sphinxstyleliteralemphasis{\sphinxupquote{{]}}}) \textendash{} 

\end{itemize}

\end{description}\end{quote}

\end{fulllineitems}

\index{k\_t() (in module carbonation)@\spxentry{k\_t()}\spxextra{in module carbonation}}

\begin{fulllineitems}
\phantomsection\label{\detokenize{carbonation:carbonation.k_t}}\pysiglinewithargsret{\sphinxcode{\sphinxupquote{carbonation.}}\sphinxbfcode{\sphinxupquote{k\_t}}}{}{}
\sphinxAtStartPar
Calculate test method regression parameter k\_t{[}\sphinxhyphen{}{]}
\subsubsection*{Notes}

\sphinxAtStartPar
for R\_ACC\_0\_inv{[}(mm\textasciicircum{}2/years)/(kg/m\textasciicircum{}3){]}

\end{fulllineitems}

\index{load\_df\_R\_ACC() (in module carbonation)@\spxentry{load\_df\_R\_ACC()}\spxextra{in module carbonation}}

\begin{fulllineitems}
\phantomsection\label{\detokenize{carbonation:carbonation.load_df_R_ACC}}\pysiglinewithargsret{\sphinxcode{\sphinxupquote{carbonation.}}\sphinxbfcode{\sphinxupquote{load\_df\_R\_ACC}}}{}{}
\sphinxAtStartPar
load the data table of the accelerated carbonation test
for R\_ACC interpolation.
\begin{quote}\begin{description}
\item[{Returns}] \leavevmode
\sphinxAtStartPar


\item[{Return type}] \leavevmode
\sphinxAtStartPar
Pandas Dataframe

\end{description}\end{quote}
\subsubsection*{Notes}

\sphinxAtStartPar
w/c 0.45 cemI is comparable to ACC of 3 mm.

\end{fulllineitems}



\section{chloride module}
\label{\detokenize{chloride:module-chloride}}\label{\detokenize{chloride:chloride-module}}\label{\detokenize{chloride::doc}}\index{module@\spxentry{module}!chloride@\spxentry{chloride}}\index{chloride@\spxentry{chloride}!module@\spxentry{module}}
\sphinxAtStartPar
\sphinxstylestrong{Summary}

\sphinxAtStartPar
Analytical solution of Fick’s second law under advection zone

\sphinxAtStartPar
Modified with material property and exposure environment
\begin{itemize}
\item {} 
\sphinxAtStartPar
\sphinxstylestrong{Resistance}:       critical chloride content

\item {} 
\sphinxAtStartPar
\sphinxstylestrong{Load}:             chloride content at rebar depth

\item {} 
\sphinxAtStartPar
\sphinxstylestrong{limit\sphinxhyphen{}state}:      chloride content at rebar depth \textgreater{}= critical chloride content

\item {} 
\sphinxAtStartPar
\sphinxstylestrong{Field data}:       chloride content profile

\end{itemize}

\sphinxAtStartPar
TODO: make t input vectorized
\index{A\_t() (in module chloride)@\spxentry{A\_t()}\spxextra{in module chloride}}

\begin{fulllineitems}
\phantomsection\label{\detokenize{chloride:chloride.A_t}}\pysiglinewithargsret{\sphinxcode{\sphinxupquote{chloride.}}\sphinxbfcode{\sphinxupquote{A\_t}}}{\emph{\DUrole{n}{t}}, \emph{\DUrole{n}{pars}}}{}
\sphinxAtStartPar
calculate A\_t considering the ageing effect
\begin{quote}\begin{description}
\item[{Parameters}] \leavevmode\begin{itemize}
\item {} 
\sphinxAtStartPar
\sphinxstyleliteralstrong{\sphinxupquote{t}} (\sphinxstyleliteralemphasis{\sphinxupquote{int}}\sphinxstyleliteralemphasis{\sphinxupquote{, }}\sphinxstyleliteralemphasis{\sphinxupquote{float}}) \textendash{} time {[}year{]}

\item {} 
\sphinxAtStartPar
\sphinxstyleliteralstrong{\sphinxupquote{pars}} (\sphinxstyleliteralemphasis{\sphinxupquote{instance of param object}}) \textendash{} 
\sphinxAtStartPar
a wrapper of all material and environmental parameters deep\sphinxhyphen{}copied from the raw data
\begin{itemize}
\item {} \begin{description}
\item[{pars.concrete\_type}] \leavevmode{[}string{]}
\sphinxAtStartPar
Option:

\sphinxAtStartPar
’Portland cement concrete’,

\sphinxAtStartPar
’Portland fly ash cement concrete’,

\sphinxAtStartPar
’Blast furnace slag cement concrete’

\end{description}

\end{itemize}


\end{itemize}

\item[{Returns}] \leavevmode
\sphinxAtStartPar
\sphinxstylestrong{out} \textendash{} subfunction considering the ‘ageing’{[}\sphinxhyphen{}{]}

\item[{Return type}] \leavevmode
\sphinxAtStartPar
numpy array

\end{description}\end{quote}

\begin{sphinxadmonition}{note}{Note:}
\sphinxAtStartPar
built\sphinxhyphen{}in parameters
\begin{itemize}
\item {} 
\sphinxAtStartPar
pars.k\_t : transfer parameter, k\_t =1 was set for experiment {[}\sphinxhyphen{}{]}

\item {} 
\sphinxAtStartPar
pars.t\_0 : reference point of time, 0.0767 {[}year{]}

\end{itemize}
\end{sphinxadmonition}

\end{fulllineitems}

\index{C\_S\_0() (in module chloride)@\spxentry{C\_S\_0()}\spxextra{in module chloride}}

\begin{fulllineitems}
\phantomsection\label{\detokenize{chloride:chloride.C_S_0}}\pysiglinewithargsret{\sphinxcode{\sphinxupquote{chloride.}}\sphinxbfcode{\sphinxupquote{C\_S\_0}}}{\emph{\DUrole{n}{pars}}}{}
\sphinxAtStartPar
Return (surface) chloride saturation concentration C\_S\_0 {[}wt.\sphinxhyphen{}\%/cement{]} caused by  C\_eqv {[}g/l{]}
\begin{quote}\begin{description}
\item[{Parameters}] \leavevmode\begin{itemize}
\item {} 
\sphinxAtStartPar
\sphinxstyleliteralstrong{\sphinxupquote{pars.C\_eqv}} (\sphinxstyleliteralemphasis{\sphinxupquote{float}}) \textendash{} calculated with by C\_eqv(pars) {[}g/L{]}

\item {} 
\sphinxAtStartPar
\sphinxstyleliteralstrong{\sphinxupquote{pars.C\_eqv\_to\_C\_S\_0}} (\sphinxstyleliteralemphasis{\sphinxupquote{global function}}) \textendash{} 
\sphinxAtStartPar
This function is based experiment with the info of
\begin{itemize}
\item {} 
\sphinxAtStartPar
binder\sphinxhyphen{}specific chloride\sphinxhyphen{}adsorption\sphinxhyphen{}isotherms

\item {} 
\sphinxAtStartPar
the concrete composition(cement/concrete ratio)

\item {} 
\sphinxAtStartPar
potential chloride impact C\_eqv {[}g/L{]}

\end{itemize}


\end{itemize}

\item[{Returns}] \leavevmode
\sphinxAtStartPar
chloride saturation concentration C\_S\_0 {[}wt.\sphinxhyphen{}\%/cement{]}

\item[{Return type}] \leavevmode
\sphinxAtStartPar
float

\end{description}\end{quote}

\begin{sphinxadmonition}{note}{Note:}
\sphinxAtStartPar
The conversion function C\_eqv\_to\_C\_S\_0(pars.C\_eqv) is derived from experiment data of 300kg cement w/c=0.5 OPC.
TODO: update to a conversion function dependent on the proportioning and cementitious material
\end{sphinxadmonition}

\end{fulllineitems}

\index{C\_S\_dx() (in module chloride)@\spxentry{C\_S\_dx()}\spxextra{in module chloride}}

\begin{fulllineitems}
\phantomsection\label{\detokenize{chloride:chloride.C_S_dx}}\pysiglinewithargsret{\sphinxcode{\sphinxupquote{chloride.}}\sphinxbfcode{\sphinxupquote{C\_S\_dx}}}{\emph{\DUrole{n}{pars}}}{}
\sphinxAtStartPar
return the substitute chloride surface concentration, i.e. chloride content just below the advection zone.

\sphinxAtStartPar
Fick’s 2nd law applies below the advection zone(depth=dx). No advection effect when dx = 0
condition considered: continuous/intermittent expsure \sphinxhyphen{} ‘submerged’,’leakage’, ‘spray’, ‘splash’ where C\_S\_dx = C\_S\_0.
The advection depth dx is calculated in the dx() function externally.

\sphinxAtStartPar
if exposure\_condition\_geom\_sensitive is True: the observed/empirical highest chloride content in concrete C\_max is used, C\_max is calculated by C\_max()
\begin{quote}\begin{description}
\item[{Parameters}] \leavevmode\begin{itemize}
\item {} 
\sphinxAtStartPar
\sphinxstyleliteralstrong{\sphinxupquote{pars}} (\sphinxstyleliteralemphasis{\sphinxupquote{object/instance of param class}}) \textendash{} contains material and environment parameters

\item {} 
\sphinxAtStartPar
\sphinxstyleliteralstrong{\sphinxupquote{pars.C\_S\_0}} (\sphinxstyleliteralemphasis{\sphinxupquote{float}}\sphinxstyleliteralemphasis{\sphinxupquote{ or }}\sphinxstyleliteralemphasis{\sphinxupquote{numpy array}}) \textendash{} chloride saturation concentration C\_S\_0 {[}wt.\sphinxhyphen{}\%/cement{]}
built\sphinxhyphen{}in calculation with C\_S\_0(pars)

\item {} 
\sphinxAtStartPar
\sphinxstyleliteralstrong{\sphinxupquote{pars.C\_max}} (\sphinxstyleliteralemphasis{\sphinxupquote{float}}) \textendash{} maximum content of chlorides within the chloride profile, {[}wt.\sphinxhyphen{}\%/cement{]}
built\sphinxhyphen{}in calculation with C\_max(pars)

\item {} 
\sphinxAtStartPar
\sphinxstyleliteralstrong{\sphinxupquote{pars.exposure\_condition}} (\sphinxstyleliteralemphasis{\sphinxupquote{string}}) \textendash{} continuous/intermittent expsure \sphinxhyphen{} ‘submerged’,’leakage’, ‘spray’, ‘splash’

\item {} 
\sphinxAtStartPar
\sphinxstyleliteralstrong{\sphinxupquote{pars.exposure\_condition\_geom\_sensitive}} (\sphinxstyleliteralemphasis{\sphinxupquote{bool}}) \textendash{} if True, the C\_max is used instead of C\_S\_0

\end{itemize}

\item[{Returns}] \leavevmode
\sphinxAtStartPar
C\_S\_dx, the substitute chloride surface concentration {[}wt.\sphinxhyphen{}\%/cement{]}

\item[{Return type}] \leavevmode
\sphinxAtStartPar
float or numpy arrays

\end{description}\end{quote}

\end{fulllineitems}

\index{C\_crit\_param() (in module chloride)@\spxentry{C\_crit\_param()}\spxextra{in module chloride}}

\begin{fulllineitems}
\phantomsection\label{\detokenize{chloride:chloride.C_crit_param}}\pysiglinewithargsret{\sphinxcode{\sphinxupquote{chloride.}}\sphinxbfcode{\sphinxupquote{C\_crit\_param}}}{}{}
\sphinxAtStartPar
return the beta distribution parameters for the critical chloride content(total chloride), C\_crit {[}wt.\sphinxhyphen{}\%/cement{]}
\begin{quote}\begin{description}
\item[{Returns}] \leavevmode
\sphinxAtStartPar
parameters of general beta distribution (mean, std, lower\_bound, upper\_bound)

\item[{Return type}] \leavevmode
\sphinxAtStartPar
tuple

\end{description}\end{quote}

\end{fulllineitems}

\index{C\_eqv() (in module chloride)@\spxentry{C\_eqv()}\spxextra{in module chloride}}

\begin{fulllineitems}
\phantomsection\label{\detokenize{chloride:chloride.C_eqv}}\pysiglinewithargsret{\sphinxcode{\sphinxupquote{chloride.}}\sphinxbfcode{\sphinxupquote{C\_eqv}}}{\emph{\DUrole{n}{pars}}}{}
\sphinxAtStartPar
Evaluate the Potential chloride impact \sphinxhyphen{}\textgreater{} equivalent chloride solution concentration, C\_eqv{[}g/L{]}
from the source of
\begin{enumerate}
\sphinxsetlistlabels{\arabic}{enumi}{enumii}{}{.}%
\item {} 
\sphinxAtStartPar
marine or coastal and/or

\item {} 
\sphinxAtStartPar
de icing salt

\end{enumerate}

\sphinxAtStartPar
It is later used to estimate the boundary condition C\_S\_dx of contineous exposure or NON\sphinxhyphen{}geometry\sphinxhyphen{}sensitive intermittent exposure
\begin{quote}\begin{description}
\item[{Parameters}] \leavevmode
\sphinxAtStartPar
\sphinxstyleliteralstrong{\sphinxupquote{pars}} (\sphinxstyleliteralemphasis{\sphinxupquote{instance of the param object}}) \textendash{} a wrapper of all material and environmental parameters deep\sphinxhyphen{}copied from the raw data
See Note for details

\item[{Returns}] \leavevmode
\sphinxAtStartPar
C\_eqv, potential chloride impact {[}g/L{]}

\item[{Return type}] \leavevmode
\sphinxAtStartPar
float

\end{description}\end{quote}

\begin{sphinxadmonition}{note}{Note:}\begin{enumerate}
\sphinxsetlistlabels{\arabic}{enumi}{enumii}{}{.}%
\item {} 
\sphinxAtStartPar
marine or coastal

\end{enumerate}
\begin{itemize}
\item {} 
\sphinxAtStartPar
pars.C\_0\_M : natural chloride content of sea water {[}g/l{]}

\end{itemize}
\begin{enumerate}
\sphinxsetlistlabels{\arabic}{enumi}{enumii}{}{.}%
\setcounter{enumi}{1}
\item {} 
\sphinxAtStartPar
de\sphinxhyphen{}icing salt (hard to quantify)

\end{enumerate}
\begin{itemize}
\item {} 
\sphinxAtStartPar
pars.C\_0\_R : average chloride content of the chloride contaminated water {[}g/l{]}

\item {} 
\sphinxAtStartPar
pars.n     : average number of salting events per year {[}\sphinxhyphen{}{]}

\item {} 
\sphinxAtStartPar
pars.C\_R\_i : average amount of chloride spread within one spreading event {[}g/m2{]}

\item {} 
\sphinxAtStartPar
pars.h\_S\_i : amount of water from rain and melted snow per spreading period {[}l/m2{]}

\end{itemize}

\sphinxAtStartPar
C\_eqv is used for contineous exposure or NON\sphinxhyphen{}geometry\sphinxhyphen{}sensitive intermittent exposure.
For geometry\sphinxhyphen{}sensitive condition(road side splash) the tested C\_max() should be used.
\end{sphinxadmonition}

\end{fulllineitems}

\index{C\_eqv\_to\_C\_S\_0() (in module chloride)@\spxentry{C\_eqv\_to\_C\_S\_0()}\spxextra{in module chloride}}

\begin{fulllineitems}
\phantomsection\label{\detokenize{chloride:chloride.C_eqv_to_C_S_0}}\pysiglinewithargsret{\sphinxcode{\sphinxupquote{chloride.}}\sphinxbfcode{\sphinxupquote{C\_eqv\_to\_C\_S\_0}}}{\emph{\DUrole{n}{C\_eqv}}}{}
\sphinxAtStartPar
Convert solution chloride content to saturated chloride content in concrete
interpolate function for 300kg cement w/c=0.5 OPC. Other empirical function should be used if available
\begin{quote}\begin{description}
\item[{Parameters}] \leavevmode
\sphinxAtStartPar
\sphinxstyleliteralstrong{\sphinxupquote{C\_eqv}} (\sphinxstyleliteralemphasis{\sphinxupquote{float}}) \textendash{} chloride content of the solution at the surface{[}g/L{]}

\item[{Returns}] \leavevmode
\sphinxAtStartPar
saturated chloride content in concrete{[}wt\sphinxhyphen{}\%/cement{]}

\item[{Return type}] \leavevmode
\sphinxAtStartPar
float

\end{description}\end{quote}

\end{fulllineitems}

\index{C\_max() (in module chloride)@\spxentry{C\_max()}\spxextra{in module chloride}}

\begin{fulllineitems}
\phantomsection\label{\detokenize{chloride:chloride.C_max}}\pysiglinewithargsret{\sphinxcode{\sphinxupquote{chloride.}}\sphinxbfcode{\sphinxupquote{C\_max}}}{\emph{\DUrole{n}{pars}}}{}
\sphinxAtStartPar
C\_max: maximum content of chlorides within the chloride profile {[}wt.\sphinxhyphen{}\%/cement{]}
calculate from empirical equations or from test data {[}wt.\sphinxhyphen{}\%/concrete{]}
\begin{quote}\begin{description}
\item[{Parameters}] \leavevmode\begin{itemize}
\item {} 
\sphinxAtStartPar
\sphinxstyleliteralstrong{\sphinxupquote{pars.cement\_concrete\_ratio}} (\sphinxstyleliteralemphasis{\sphinxupquote{float}}) \textendash{} cement/concrete weight ratio, used to convert {[}wt.\sphinxhyphen{}\%/concrete{]} \sphinxhyphen{}\textgreater{} {[}wt.\sphinxhyphen{}\%/cement{]}

\item {} 
\sphinxAtStartPar
\sphinxstyleliteralstrong{\sphinxupquote{pars.C\_max\_option}} (\sphinxstyleliteralemphasis{\sphinxupquote{string}}) \textendash{} “empirical” \sphinxhyphen{} use empirical equation
“user\_input” \sphinxhyphen{} use user input, from test

\item {} 
\sphinxAtStartPar
\sphinxstyleliteralstrong{\sphinxupquote{pars.x\_a}} \textendash{} “empirical” option: horizontal distance from the roadside {[}cm{]}

\item {} 
\sphinxAtStartPar
\sphinxstyleliteralstrong{\sphinxupquote{pars.x\_h}} \textendash{} “empirical” option: height above road surface {[}cm{]}

\item {} 
\sphinxAtStartPar
\sphinxstyleliteralstrong{\sphinxupquote{pars.C\_max\_user\_input}} \textendash{} “user\_input” option: Experiment\sphinxhyphen{}tested maximum chloride content {[}wt.\sphinxhyphen{}\%/concrete{]}

\end{itemize}

\item[{Returns}] \leavevmode
\sphinxAtStartPar
\sphinxstylestrong{C\_max} \textendash{} maximum content of chlorides within the chloride profile, {[}wt.\sphinxhyphen{}\%/cement{]}

\item[{Return type}] \leavevmode
\sphinxAtStartPar
float

\end{description}\end{quote}

\begin{sphinxadmonition}{note}{Note:}
\sphinxAtStartPar
The empirical expression should be determined for structures of different exposure or concrete mixe.
A typical C\_max used by default in this function is from
\begin{itemize}
\item {} 
\sphinxAtStartPar
location: urban and rural areas in Germany

\item {} 
\sphinxAtStartPar
time of exposure of the considered structure: 5\sphinxhyphen{}40 years

\item {} 
\sphinxAtStartPar
concrete: CEM I, w/c = 0.45 up to w/c = 0.60,

\end{itemize}
\end{sphinxadmonition}

\end{fulllineitems}

\index{Chloride\_Model (class in chloride)@\spxentry{Chloride\_Model}\spxextra{class in chloride}}

\begin{fulllineitems}
\phantomsection\label{\detokenize{chloride:chloride.Chloride_Model}}\pysiglinewithargsret{\sphinxbfcode{\sphinxupquote{class }}\sphinxcode{\sphinxupquote{chloride.}}\sphinxbfcode{\sphinxupquote{Chloride\_Model}}}{\emph{\DUrole{n}{pars\_raw}}}{}
\sphinxAtStartPar
Bases: \sphinxcode{\sphinxupquote{object}}
\index{calibrate() (chloride.Chloride\_Model method)@\spxentry{calibrate()}\spxextra{chloride.Chloride\_Model method}}

\begin{fulllineitems}
\phantomsection\label{\detokenize{chloride:chloride.Chloride_Model.calibrate}}\pysiglinewithargsret{\sphinxbfcode{\sphinxupquote{calibrate}}}{\emph{\DUrole{n}{t}}, \emph{\DUrole{n}{chloride\_content\_field}}, \emph{\DUrole{n}{print\_proc}\DUrole{o}{=}\DUrole{default_value}{False}}, \emph{\DUrole{n}{plot}\DUrole{o}{=}\DUrole{default_value}{True}}}{}
\sphinxAtStartPar
return a calibrated model with calibrate\_chloride\_f\_group() function
\begin{quote}\begin{description}
\item[{Parameters}] \leavevmode\begin{itemize}
\item {} 
\sphinxAtStartPar
\sphinxstyleliteralstrong{\sphinxupquote{t}} (\sphinxstyleliteralemphasis{\sphinxupquote{int}}\sphinxstyleliteralemphasis{\sphinxupquote{, }}\sphinxstyleliteralemphasis{\sphinxupquote{float}}) \textendash{} time {[}year{]}

\item {} 
\sphinxAtStartPar
\sphinxstyleliteralstrong{\sphinxupquote{chloride\_content\_field}} (\sphinxstyleliteralemphasis{\sphinxupquote{pandas dataframe}}) \textendash{} containts field chloride contents at various depths {[}wt.\sphinxhyphen{}\%/cement{]}

\item {} 
\sphinxAtStartPar
\sphinxstyleliteralstrong{\sphinxupquote{print\_proc}} (\sphinxstyleliteralemphasis{\sphinxupquote{bool}}\sphinxstyleliteralemphasis{\sphinxupquote{, }}\sphinxstyleliteralemphasis{\sphinxupquote{optional}}) \textendash{} if true, print the optimization process, by default False

\item {} 
\sphinxAtStartPar
\sphinxstyleliteralstrong{\sphinxupquote{plot}} (\sphinxstyleliteralemphasis{\sphinxupquote{bool}}\sphinxstyleliteralemphasis{\sphinxupquote{, }}\sphinxstyleliteralemphasis{\sphinxupquote{optional}}) \textendash{} if true, plot the field vs model comparison, by default True

\end{itemize}

\item[{Returns}] \leavevmode
\sphinxAtStartPar
a new calibrated model with the averaged calibrated D\_RCM\_0

\item[{Return type}] \leavevmode
\sphinxAtStartPar
instance of Chloride\_Model object

\end{description}\end{quote}

\end{fulllineitems}

\index{chloride\_with\_year() (chloride.Chloride\_Model method)@\spxentry{chloride\_with\_year()}\spxextra{chloride.Chloride\_Model method}}

\begin{fulllineitems}
\phantomsection\label{\detokenize{chloride:chloride.Chloride_Model.chloride_with_year}}\pysiglinewithargsret{\sphinxbfcode{\sphinxupquote{chloride\_with\_year}}}{\emph{\DUrole{n}{depth}}, \emph{\DUrole{n}{year\_lis}}, \emph{\DUrole{n}{plot}\DUrole{o}{=}\DUrole{default_value}{True}}, \emph{\DUrole{n}{amplify}\DUrole{o}{=}\DUrole{default_value}{1}}}{}
\sphinxAtStartPar
chloride\_with\_year runs the model for a list of time steps
\begin{quote}\begin{description}
\item[{Parameters}] \leavevmode\begin{itemize}
\item {} 
\sphinxAtStartPar
\sphinxstyleliteralstrong{\sphinxupquote{depth}} (\sphinxstyleliteralemphasis{\sphinxupquote{float}}) \textendash{} depth at which the chloride concrete is calculated, x{[}mm{]}

\item {} 
\sphinxAtStartPar
\sphinxstyleliteralstrong{\sphinxupquote{year\_lis}} (\sphinxstyleliteralemphasis{\sphinxupquote{list}}) \textendash{} a list of time steps {[}year{]}

\item {} 
\sphinxAtStartPar
\sphinxstyleliteralstrong{\sphinxupquote{plot}} (\sphinxstyleliteralemphasis{\sphinxupquote{bool}}\sphinxstyleliteralemphasis{\sphinxupquote{, }}\sphinxstyleliteralemphasis{\sphinxupquote{optional}}) \textendash{} if true, plot the R S curve, pf, beta with time axis, by default True

\item {} 
\sphinxAtStartPar
\sphinxstyleliteralstrong{\sphinxupquote{amplify}} (\sphinxstyleliteralemphasis{\sphinxupquote{int}}\sphinxstyleliteralemphasis{\sphinxupquote{, }}\sphinxstyleliteralemphasis{\sphinxupquote{optional}}) \textendash{} a scale parameter adjusting the hight of the distribution curve, by default 80

\end{itemize}

\item[{Returns}] \leavevmode
\sphinxAtStartPar
(pf list, beta list)

\item[{Return type}] \leavevmode
\sphinxAtStartPar
tuple

\end{description}\end{quote}

\end{fulllineitems}

\index{copy() (chloride.Chloride\_Model method)@\spxentry{copy()}\spxextra{chloride.Chloride\_Model method}}

\begin{fulllineitems}
\phantomsection\label{\detokenize{chloride:chloride.Chloride_Model.copy}}\pysiglinewithargsret{\sphinxbfcode{\sphinxupquote{copy}}}{}{}
\sphinxAtStartPar
create a deepcopy of the instance, to preserve the mutable object

\end{fulllineitems}

\index{postproc() (chloride.Chloride\_Model method)@\spxentry{postproc()}\spxextra{chloride.Chloride\_Model method}}

\begin{fulllineitems}
\phantomsection\label{\detokenize{chloride:chloride.Chloride_Model.postproc}}\pysiglinewithargsret{\sphinxbfcode{\sphinxupquote{postproc}}}{\emph{\DUrole{n}{plot}\DUrole{o}{=}\DUrole{default_value}{False}}}{}
\sphinxAtStartPar
postproc the solved model and attach the Pf and beta to the model object
\begin{quote}\begin{description}
\item[{Parameters}] \leavevmode
\sphinxAtStartPar
\sphinxstyleliteralstrong{\sphinxupquote{plot}} (\sphinxstyleliteralemphasis{\sphinxupquote{bool}}\sphinxstyleliteralemphasis{\sphinxupquote{, }}\sphinxstyleliteralemphasis{\sphinxupquote{optional}}) \textendash{} if true, plot the R S curve, by default False

\end{description}\end{quote}

\end{fulllineitems}

\index{run() (chloride.Chloride\_Model method)@\spxentry{run()}\spxextra{chloride.Chloride\_Model method}}

\begin{fulllineitems}
\phantomsection\label{\detokenize{chloride:chloride.Chloride_Model.run}}\pysiglinewithargsret{\sphinxbfcode{\sphinxupquote{run}}}{\emph{\DUrole{n}{x}}, \emph{\DUrole{n}{t}}}{}
\sphinxAtStartPar
solve the chloride content at depth x and time t
:param x: depth x{[}mm{]}
:type x: int, float
:param t: time{[}year{]}
:type t: float

\end{fulllineitems}


\end{fulllineitems}

\index{Chloride\_content() (in module chloride)@\spxentry{Chloride\_content()}\spxextra{in module chloride}}

\begin{fulllineitems}
\phantomsection\label{\detokenize{chloride:chloride.Chloride_content}}\pysiglinewithargsret{\sphinxcode{\sphinxupquote{chloride.}}\sphinxbfcode{\sphinxupquote{Chloride\_content}}}{\emph{\DUrole{n}{x}}, \emph{\DUrole{n}{t}}, \emph{\DUrole{n}{pars}}}{}
\sphinxAtStartPar
Chloride\_content is the master model function, calculate chloride content at depth x and time t with Fick’s 2nd law below the convection zone (x \textgreater{} dx)
The derived parameters is also calculated within this funcion.
\begin{itemize}
\item {} 
\sphinxAtStartPar
Caution: The pars instance is mutable, so a deepcopy of the original instance should be used if the calculation is not intended for “inplace”.

\end{itemize}
\begin{quote}\begin{description}
\item[{Parameters}] \leavevmode\begin{itemize}
\item {} 
\sphinxAtStartPar
\sphinxstyleliteralstrong{\sphinxupquote{x}} (\sphinxstyleliteralemphasis{\sphinxupquote{float}}\sphinxstyleliteralemphasis{\sphinxupquote{, }}\sphinxstyleliteralemphasis{\sphinxupquote{int}}) \textendash{} depth at which chloride content C\_x\_t is reported {[}mm{]}

\item {} 
\sphinxAtStartPar
\sphinxstyleliteralstrong{\sphinxupquote{t}} (\sphinxstyleliteralemphasis{\sphinxupquote{float}}\sphinxstyleliteralemphasis{\sphinxupquote{, }}\sphinxstyleliteralemphasis{\sphinxupquote{int}}) \textendash{} time {[}year{]}

\item {} 
\sphinxAtStartPar
\sphinxstyleliteralstrong{\sphinxupquote{pars}} (\sphinxstyleliteralemphasis{\sphinxupquote{instance of param object}}) \textendash{} a wrapper of all material and environmental parameters deep\sphinxhyphen{}copied from the raw data

\end{itemize}

\item[{Returns}] \leavevmode
\sphinxAtStartPar
sample of the distribution of the chloride content in concrete at a depth x (suface x=0) at time t {[}wt\sphinxhyphen{}.\%/c{]}

\item[{Return type}] \leavevmode
\sphinxAtStartPar
numpy array

\end{description}\end{quote}

\begin{sphinxadmonition}{note}{Note:}
\sphinxAtStartPar
intermediate parameters were calculated and attached to pars
\begin{itemize}
\item {} 
\sphinxAtStartPar
C\_0    : initial chloride content of the concrete {[}wt\sphinxhyphen{}.\%/cement{]}

\item {} 
\sphinxAtStartPar
C\_S\_dx : chloride content at a depth dx and a certain point of time t {[}wt\sphinxhyphen{}.\%/cement{]}

\item {} 
\sphinxAtStartPar
dx     : depth of the convection zone (concrete layer, up to which the process of chloride penetration differs from Fick’s 2nd law of diffusion) {[}mm{]}

\item {} 
\sphinxAtStartPar
D\_app  : apparent coefficient of chloride diffusion through concrete {[}mm\textasciicircum{}2/year{]}

\item {} 
\sphinxAtStartPar
erf    : imported error function

\end{itemize}
\end{sphinxadmonition}

\end{fulllineitems}

\index{D\_RCM\_0() (in module chloride)@\spxentry{D\_RCM\_0()}\spxextra{in module chloride}}

\begin{fulllineitems}
\phantomsection\label{\detokenize{chloride:chloride.D_RCM_0}}\pysiglinewithargsret{\sphinxcode{\sphinxupquote{chloride.}}\sphinxbfcode{\sphinxupquote{D\_RCM\_0}}}{\emph{\DUrole{n}{pars}}}{}
\sphinxAtStartPar
Return the chloride migration coefficient from Rapid chloride migration test {[}m\textasciicircum{}2/s{]} see NT Build 492
if the test data is not available from pars, use interpolation of existion empirical data for orientation purpose
Pay attention to the units output {[}mm\textasciicircum{}2/year{]}, used for the model
\begin{quote}\begin{description}
\item[{Parameters}] \leavevmode\begin{itemize}
\item {} 
\sphinxAtStartPar
\sphinxstyleliteralstrong{\sphinxupquote{pars}} (\sphinxstyleliteralemphasis{\sphinxupquote{instance of param object}}) \textendash{} a wrapper of all material and environmental parameters deep\sphinxhyphen{}copied from the raw data

\item {} 
\sphinxAtStartPar
\sphinxstyleliteralstrong{\sphinxupquote{pars.D\_RCM\_test}} (\sphinxstyleliteralemphasis{\sphinxupquote{int}}\sphinxstyleliteralemphasis{\sphinxupquote{ or }}\sphinxstyleliteralemphasis{\sphinxupquote{float}}) \textendash{} RCM test results{[}m\textasciicircum{}2/s{]}, the mean value from the test is used, and standard deviation is estimated based on mean

\item {} 
\sphinxAtStartPar
\sphinxstyleliteralstrong{\sphinxupquote{pars.option.choose}} (\sphinxstyleliteralemphasis{\sphinxupquote{bool}}) \textendash{} if true interpolation from existing data table is used

\item {} 
\sphinxAtStartPar
\sphinxstyleliteralstrong{\sphinxupquote{pars.option.df\_D\_RCM\_0}} (\sphinxstyleliteralemphasis{\sphinxupquote{pandas dataframe}}) \textendash{} experimental data table(cement type, and w/c eqv) for interpolation

\item {} 
\sphinxAtStartPar
\sphinxstyleliteralstrong{\sphinxupquote{pars.option.cement\_type}} (\sphinxstyleliteralemphasis{\sphinxupquote{string}}) \textendash{} 
\sphinxAtStartPar
select cement type for data interpolation of the df\_D\_RCM\_0,
Options:
‘CEM\_I\_42.5\_R’

\sphinxAtStartPar
’CEM\_I\_42.5\_R+FA’

\sphinxAtStartPar
’CEM\_I\_42.5\_R+SF’

\sphinxAtStartPar
’CEM\_III/B\_42.5’


\item {} 
\sphinxAtStartPar
\sphinxstyleliteralstrong{\sphinxupquote{pars.option.wc\_eqv}} (\sphinxstyleliteralemphasis{\sphinxupquote{float}}) \textendash{} equivalent water cement ratio considering supplementary cementitious materials

\end{itemize}

\item[{Returns}] \leavevmode
\sphinxAtStartPar
D\_RCM\_0\_final {[}mm\textasciicircum{}2/year{]}

\item[{Return type}] \leavevmode
\sphinxAtStartPar
numpy array

\end{description}\end{quote}

\end{fulllineitems}

\index{D\_app() (in module chloride)@\spxentry{D\_app()}\spxextra{in module chloride}}

\begin{fulllineitems}
\phantomsection\label{\detokenize{chloride:chloride.D_app}}\pysiglinewithargsret{\sphinxcode{\sphinxupquote{chloride.}}\sphinxbfcode{\sphinxupquote{D\_app}}}{\emph{\DUrole{n}{t}}, \emph{\DUrole{n}{pars}}}{}
\sphinxAtStartPar
Calculate the apparent coefficient of chloride diffusion through concrete D\_app{[}mm\textasciicircum{}2/year{]}
\begin{quote}\begin{description}
\item[{Parameters}] \leavevmode\begin{itemize}
\item {} 
\sphinxAtStartPar
\sphinxstyleliteralstrong{\sphinxupquote{t}} (\sphinxstyleliteralemphasis{\sphinxupquote{float}}\sphinxstyleliteralemphasis{\sphinxupquote{, }}\sphinxstyleliteralemphasis{\sphinxupquote{int}}) \textendash{} time {[}year{]}

\item {} 
\sphinxAtStartPar
\sphinxstyleliteralstrong{\sphinxupquote{pars}} (\sphinxstyleliteralemphasis{\sphinxupquote{instance of param object}}) \textendash{} a wrapper of all material and environmental parameters deep\sphinxhyphen{}copied from the raw data

\end{itemize}

\item[{Returns}] \leavevmode
\sphinxAtStartPar
sample of the distribution of the apparent coefficient of chloride diffusion through concrete {[}mm\textasciicircum{}2/year{]}

\item[{Return type}] \leavevmode
\sphinxAtStartPar
numpy array

\end{description}\end{quote}

\begin{sphinxadmonition}{note}{Note:}
\sphinxAtStartPar
intermediate parameters calculated and attached to pars
\begin{itemize}
\item {} 
\sphinxAtStartPar
k\_e     : environmental transfer variable {[}\sphinxhyphen{}{]}

\item {} 
\sphinxAtStartPar
D\_RCM\_0 : chloride migration coefficient {[}mm\textasciicircum{}2/year{]}

\item {} 
\sphinxAtStartPar
k\_t     : transfer parameter, k\_t =1 was set in A\_t(){[}\sphinxhyphen{}{]}

\item {} 
\sphinxAtStartPar
A\_t     : subfunction considering the ‘ageing’ {[}\sphinxhyphen{}{]}

\end{itemize}
\end{sphinxadmonition}

\end{fulllineitems}

\index{b\_e() (in module chloride)@\spxentry{b\_e()}\spxextra{in module chloride}}

\begin{fulllineitems}
\phantomsection\label{\detokenize{chloride:chloride.b_e}}\pysiglinewithargsret{\sphinxcode{\sphinxupquote{chloride.}}\sphinxbfcode{\sphinxupquote{b\_e}}}{}{}
\sphinxAtStartPar
provide the large sample array of b\_e : regression variable {[}K{]}

\end{fulllineitems}

\index{calibrate\_chloride\_f() (in module chloride)@\spxentry{calibrate\_chloride\_f()}\spxextra{in module chloride}}

\begin{fulllineitems}
\phantomsection\label{\detokenize{chloride:chloride.calibrate_chloride_f}}\pysiglinewithargsret{\sphinxcode{\sphinxupquote{chloride.}}\sphinxbfcode{\sphinxupquote{calibrate\_chloride\_f}}}{\emph{\DUrole{n}{model\_raw}}, \emph{\DUrole{n}{x}}, \emph{\DUrole{n}{t}}, \emph{\DUrole{n}{chloride\_content}}, \emph{\DUrole{n}{tol}\DUrole{o}{=}\DUrole{default_value}{1e\sphinxhyphen{}15}}, \emph{\DUrole{n}{max\_count}\DUrole{o}{=}\DUrole{default_value}{50}}, \emph{\DUrole{n}{print\_out}\DUrole{o}{=}\DUrole{default_value}{True}}, \emph{\DUrole{n}{print\_proc}\DUrole{o}{=}\DUrole{default_value}{False}}}{}
\sphinxAtStartPar
calibrate chloride model to field data at one depth at one time.
Calibrate the chloride model with field chloride test data and return the new calibrated model object/instance
Optimization method:  Field chloride content at depth x and time t \sphinxhyphen{}\textgreater{} find corresponding D\_RCM\_0(repaid chloride migration diffusivity{[}m\textasciicircum{}2/s{]})
\begin{quote}\begin{description}
\item[{Parameters}] \leavevmode\begin{itemize}
\item {} 
\sphinxAtStartPar
\sphinxstyleliteralstrong{\sphinxupquote{model\_raw}} (\sphinxstyleliteralemphasis{\sphinxupquote{object/instance of Chloride\_model class}}\sphinxstyleliteralemphasis{\sphinxupquote{ (}}\sphinxstyleliteralemphasis{\sphinxupquote{to be calibrated}}\sphinxstyleliteralemphasis{\sphinxupquote{)}}) \textendash{} 

\item {} 
\sphinxAtStartPar
\sphinxstyleliteralstrong{\sphinxupquote{x}} (\sphinxstyleliteralemphasis{\sphinxupquote{float}}) \textendash{} depth {[}mm{]}

\item {} 
\sphinxAtStartPar
\sphinxstyleliteralstrong{\sphinxupquote{t}} (\sphinxstyleliteralemphasis{\sphinxupquote{: int}}\sphinxstyleliteralemphasis{\sphinxupquote{ or }}\sphinxstyleliteralemphasis{\sphinxupquote{float}}) \textendash{} time {[}year{]}

\item {} 
\sphinxAtStartPar
\sphinxstyleliteralstrong{\sphinxupquote{chloride\_content}} (\sphinxstyleliteralemphasis{\sphinxupquote{float}}\sphinxstyleliteralemphasis{\sphinxupquote{ or }}\sphinxstyleliteralemphasis{\sphinxupquote{int}}) \textendash{} field chloride\_content{[}wt.\sphinxhyphen{}\%/cement{]} at time t, depth x,

\item {} 
\sphinxAtStartPar
\sphinxstyleliteralstrong{\sphinxupquote{tol}} (\sphinxstyleliteralemphasis{\sphinxupquote{float}}) \textendash{} D\_RCM\_0 optimization absolute tolerance 1e\sphinxhyphen{}15 {[}m\textasciicircum{}2/s{]}

\item {} 
\sphinxAtStartPar
\sphinxstyleliteralstrong{\sphinxupquote{max\_count}} (\sphinxstyleliteralemphasis{\sphinxupquote{int}}) \textendash{} maximun number of searching iteration, default is 50

\item {} 
\sphinxAtStartPar
\sphinxstyleliteralstrong{\sphinxupquote{print\_out}} (\sphinxstyleliteralemphasis{\sphinxupquote{bool}}) \textendash{} if true, print model and field chloride content

\item {} 
\sphinxAtStartPar
\sphinxstyleliteralstrong{\sphinxupquote{print\_proc}} (\sphinxstyleliteralemphasis{\sphinxupquote{bool}}) \textendash{} if turn, print optimization process. (debug message in the logger)

\end{itemize}

\item[{Returns}] \leavevmode
\sphinxAtStartPar
new calibrated model

\item[{Return type}] \leavevmode
\sphinxAtStartPar
instance of Chloride\_Model object

\end{description}\end{quote}

\begin{sphinxadmonition}{note}{Note:}
\sphinxAtStartPar
calibrate model to field data at three depths in calibrate\_chloride\_f\_group()
chloride\_content\_field{[}wt.\sphinxhyphen{}\%/cement{]} at time t
\begin{itemize}
\item {} 
\sphinxAtStartPar
optimizing corresponding D\_RCM\_0,

\item {} 
\sphinxAtStartPar
fixed C\_S\_dx (exposure type dependent)

\item {} 
\sphinxAtStartPar
fixed dx (determined by the original model)

\end{itemize}
\end{sphinxadmonition}

\end{fulllineitems}

\index{calibrate\_chloride\_f\_group() (in module chloride)@\spxentry{calibrate\_chloride\_f\_group()}\spxextra{in module chloride}}

\begin{fulllineitems}
\phantomsection\label{\detokenize{chloride:chloride.calibrate_chloride_f_group}}\pysiglinewithargsret{\sphinxcode{\sphinxupquote{chloride.}}\sphinxbfcode{\sphinxupquote{calibrate\_chloride\_f\_group}}}{\emph{\DUrole{n}{model\_raw}}, \emph{\DUrole{n}{t}}, \emph{\DUrole{n}{chloride\_content\_field}}, \emph{\DUrole{n}{plot}\DUrole{o}{=}\DUrole{default_value}{True}}, \emph{\DUrole{n}{print\_proc}\DUrole{o}{=}\DUrole{default_value}{False}}}{}
\sphinxAtStartPar
use calibrate\_chloride\_f() to calibrate model to field chloride content at three or more depths, and return the new calibrated model with the averaged D\_RCM\_0
\begin{quote}\begin{description}
\item[{Parameters}] \leavevmode\begin{itemize}
\item {} 
\sphinxAtStartPar
\sphinxstyleliteralstrong{\sphinxupquote{model\_raw}} (\sphinxstyleliteralemphasis{\sphinxupquote{object/instance of Chloride\_Model class}}) \textendash{} model object to be calibrated), model\_raw.copy() will be used

\item {} 
\sphinxAtStartPar
\sphinxstyleliteralstrong{\sphinxupquote{chloride\_content\_field}} (\sphinxstyleliteralemphasis{\sphinxupquote{pandas dataframe}}) \textendash{} containts field chloride contents at various depths {[}wt.\sphinxhyphen{}\%/cement{]}

\item {} 
\sphinxAtStartPar
\sphinxstyleliteralstrong{\sphinxupquote{t}} (\sphinxstyleliteralemphasis{\sphinxupquote{int}}\sphinxstyleliteralemphasis{\sphinxupquote{ or }}\sphinxstyleliteralemphasis{\sphinxupquote{float}}) \textendash{} time {[}year{]}

\end{itemize}

\item[{Returns}] \leavevmode
\sphinxAtStartPar
a new calibrated model with the averaged calibrated D\_RCM\_0

\item[{Return type}] \leavevmode
\sphinxAtStartPar
object/instance of Chloride\_model class

\end{description}\end{quote}

\end{fulllineitems}

\index{chloride\_year() (in module chloride)@\spxentry{chloride\_year()}\spxextra{in module chloride}}

\begin{fulllineitems}
\phantomsection\label{\detokenize{chloride:chloride.chloride_year}}\pysiglinewithargsret{\sphinxcode{\sphinxupquote{chloride.}}\sphinxbfcode{\sphinxupquote{chloride\_year}}}{\emph{\DUrole{n}{model}}, \emph{\DUrole{n}{depth}}, \emph{\DUrole{n}{year\_lis}}, \emph{\DUrole{n}{plot}\DUrole{o}{=}\DUrole{default_value}{True}}, \emph{\DUrole{n}{amplify}\DUrole{o}{=}\DUrole{default_value}{80}}}{}
\sphinxAtStartPar
run model over a list of time steps

\end{fulllineitems}

\index{dx() (in module chloride)@\spxentry{dx()}\spxextra{in module chloride}}

\begin{fulllineitems}
\phantomsection\label{\detokenize{chloride:chloride.dx}}\pysiglinewithargsret{\sphinxcode{\sphinxupquote{chloride.}}\sphinxbfcode{\sphinxupquote{dx}}}{\emph{\DUrole{n}{pars}}}{}
\sphinxAtStartPar
return dx : advection depth {[}mm{]} dependent on the exposure conditions

\end{fulllineitems}

\index{k\_e() (in module chloride)@\spxentry{k\_e()}\spxextra{in module chloride}}

\begin{fulllineitems}
\phantomsection\label{\detokenize{chloride:chloride.k_e}}\pysiglinewithargsret{\sphinxcode{\sphinxupquote{chloride.}}\sphinxbfcode{\sphinxupquote{k\_e}}}{\emph{\DUrole{n}{pars}}}{}
\sphinxAtStartPar
Calculate k\_e: environmental transfer variable {[}\sphinxhyphen{}{]}
\begin{quote}\begin{description}
\item[{Parameters}] \leavevmode
\sphinxAtStartPar
\sphinxstyleliteralstrong{\sphinxupquote{pars}} (\sphinxstyleliteralemphasis{\sphinxupquote{instance of param object}}) \textendash{} 
\sphinxAtStartPar
a wrapper of all material and environmental parameters deep\sphinxhyphen{}copied from the raw data
\begin{itemize}
\item {} 
\sphinxAtStartPar
pars.T\_ref  : standard test temperatrue 293 {[}K{]}

\item {} 
\sphinxAtStartPar
pars.T\_real : temperature of the structural element {[}K{]}

\item {} 
\sphinxAtStartPar
pars.b\_e    : regression variable {[}K{]}

\end{itemize}


\item[{Returns}] \leavevmode
\sphinxAtStartPar
large sample of the distribution of k\_e

\item[{Return type}] \leavevmode
\sphinxAtStartPar
numpy array

\end{description}\end{quote}

\end{fulllineitems}

\index{load\_df\_D\_RCM() (in module chloride)@\spxentry{load\_df\_D\_RCM()}\spxextra{in module chloride}}

\begin{fulllineitems}
\phantomsection\label{\detokenize{chloride:chloride.load_df_D_RCM}}\pysiglinewithargsret{\sphinxcode{\sphinxupquote{chloride.}}\sphinxbfcode{\sphinxupquote{load\_df\_D\_RCM}}}{}{}
\sphinxAtStartPar
load the data table of the Rapid Chloride Migration(RCM) test
for D\_RCM interpolation.
\begin{quote}\begin{description}
\item[{Returns}] \leavevmode
\sphinxAtStartPar
Data table from experiment

\item[{Return type}] \leavevmode
\sphinxAtStartPar
Pandas Dataframe

\end{description}\end{quote}

\end{fulllineitems}



\section{corrosion module}
\label{\detokenize{corrosion:module-corrosion}}\label{\detokenize{corrosion:corrosion-module}}\label{\detokenize{corrosion::doc}}\index{module@\spxentry{module}!corrosion@\spxentry{corrosion}}\index{corrosion@\spxentry{corrosion}!module@\spxentry{module}}
\sphinxAtStartPar
\sphinxstylestrong{Summary}

\sphinxAtStartPar
2D electrochemical model and its regressed solution

\sphinxAtStartPar
icorr = f(moisture, temperature, oxygen availability)

\sphinxAtStartPar
\sphinxstylestrong{Field data}
\begin{itemize}
\item {} 
\sphinxAtStartPar
Volumetric water content (TDR moisture sensor)

\item {} 
\sphinxAtStartPar
corrosion rate (LPR, corrosion sensor) to validate the model

\end{itemize}
\index{C\_f() (in module corrosion)@\spxentry{C\_f()}\spxextra{in module corrosion}}

\begin{fulllineitems}
\phantomsection\label{\detokenize{corrosion:corrosion.C_f}}\pysiglinewithargsret{\sphinxcode{\sphinxupquote{corrosion.}}\sphinxbfcode{\sphinxupquote{C\_f}}}{\emph{\DUrole{n}{T}}}{}
\sphinxAtStartPar
C\_f returns BET model parameter C sampled from a normal distribution
\begin{quote}\begin{description}
\item[{Parameters}] \leavevmode
\sphinxAtStartPar
\sphinxstyleliteralstrong{\sphinxupquote{T}} (\sphinxstyleliteralemphasis{\sphinxupquote{float}}) \textendash{} temperature {[}K{]}

\end{description}\end{quote}

\begin{sphinxadmonition}{note}{Note:}
\sphinxAtStartPar
C varies from 10 to 50. This function is not applicable for elevated temperatures
\end{sphinxadmonition}

\end{fulllineitems}

\index{Corrosion\_Model (class in corrosion)@\spxentry{Corrosion\_Model}\spxextra{class in corrosion}}

\begin{fulllineitems}
\phantomsection\label{\detokenize{corrosion:corrosion.Corrosion_Model}}\pysiglinewithargsret{\sphinxbfcode{\sphinxupquote{class }}\sphinxcode{\sphinxupquote{corrosion.}}\sphinxbfcode{\sphinxupquote{Corrosion\_Model}}}{\emph{\DUrole{n}{pars}}}{}
\sphinxAtStartPar
Bases: \sphinxcode{\sphinxupquote{object}}
\index{calibrate() (corrosion.Corrosion\_Model method)@\spxentry{calibrate()}\spxextra{corrosion.Corrosion\_Model method}}

\begin{fulllineitems}
\phantomsection\label{\detokenize{corrosion:corrosion.Corrosion_Model.calibrate}}\pysiglinewithargsret{\sphinxbfcode{\sphinxupquote{calibrate}}}{\emph{\DUrole{n}{field\_data}}}{}
\end{fulllineitems}

\index{copy() (corrosion.Corrosion\_Model method)@\spxentry{copy()}\spxextra{corrosion.Corrosion\_Model method}}

\begin{fulllineitems}
\phantomsection\label{\detokenize{corrosion:corrosion.Corrosion_Model.copy}}\pysiglinewithargsret{\sphinxbfcode{\sphinxupquote{copy}}}{}{}
\end{fulllineitems}

\index{run() (corrosion.Corrosion\_Model method)@\spxentry{run()}\spxextra{corrosion.Corrosion\_Model method}}

\begin{fulllineitems}
\phantomsection\label{\detokenize{corrosion:corrosion.Corrosion_Model.run}}\pysiglinewithargsret{\sphinxbfcode{\sphinxupquote{run}}}{}{}
\sphinxAtStartPar
solve for icorr and the corresponding section loss rate

\end{fulllineitems}


\end{fulllineitems}

\index{Cs\_g\_f() (in module corrosion)@\spxentry{Cs\_g\_f()}\spxextra{in module corrosion}}

\begin{fulllineitems}
\phantomsection\label{\detokenize{corrosion:corrosion.Cs_g_f}}\pysiglinewithargsret{\sphinxcode{\sphinxupquote{corrosion.}}\sphinxbfcode{\sphinxupquote{Cs\_g\_f}}}{}{}
\sphinxAtStartPar
atmospheric O2 concentration in gas phase on the boundary {[}mol/m\textasciicircum{}3{]}, converted from 20.95\% by volume

\end{fulllineitems}

\index{De\_O2\_f() (in module corrosion)@\spxentry{De\_O2\_f()}\spxextra{in module corrosion}}

\begin{fulllineitems}
\phantomsection\label{\detokenize{corrosion:corrosion.De_O2_f}}\pysiglinewithargsret{\sphinxcode{\sphinxupquote{corrosion.}}\sphinxbfcode{\sphinxupquote{De\_O2\_f}}}{\emph{\DUrole{n}{pars}}}{}
\sphinxAtStartPar
calculate the O2 effective diffusivity of concrete
:param pars:
:type pars: instance of Param object
\begin{quote}\begin{description}
\item[{Returns}] \leavevmode
\sphinxAtStartPar
O2 effective diffusivity of concrete

\item[{Return type}] \leavevmode
\sphinxAtStartPar
float, numpy array

\end{description}\end{quote}
\subsubsection*{Notes}

\sphinxAtStartPar
important intermediate Parameters
\begin{itemize}
\item {} 
\sphinxAtStartPar
epsilon\_p : porosity of hardened cement paste,

\item {} 
\sphinxAtStartPar
RH : relative humidity {[}\sphinxhyphen{}\%{]}

\end{itemize}

\sphinxAtStartPar
Gas diffusion along the aggregate\sphinxhyphen{}paste interface makes up for the lack of diffusion through the aggregate particles themselves.
Therefore, the value of effective diffusivity is considered herein as a function of the porosity of hardened cement paste.
{[}TODO: add temperature dependence{]}

\end{fulllineitems}

\index{RH\_to\_WaterbyMassHCP() (in module corrosion)@\spxentry{RH\_to\_WaterbyMassHCP()}\spxextra{in module corrosion}}

\begin{fulllineitems}
\phantomsection\label{\detokenize{corrosion:corrosion.RH_to_WaterbyMassHCP}}\pysiglinewithargsret{\sphinxcode{\sphinxupquote{corrosion.}}\sphinxbfcode{\sphinxupquote{RH\_to\_WaterbyMassHCP}}}{\emph{\DUrole{n}{pars}}}{}
\sphinxAtStartPar
return water content(g/g hardended cement paste) from RH in pores/environment based on w\_c, cement\_type, Temperature by using modified BET model

\begin{sphinxadmonition}{note}{Note:}
\sphinxAtStartPar
Refernce: Xi, Y., Bazant, Z. P., \& Jennings, H. M. (1993). Moisture Diffusion in Cementitious Materials Adsorption Isotherms.
\end{sphinxadmonition}

\end{fulllineitems}

\index{Section\_loss\_Model (class in corrosion)@\spxentry{Section\_loss\_Model}\spxextra{class in corrosion}}

\begin{fulllineitems}
\phantomsection\label{\detokenize{corrosion:corrosion.Section_loss_Model}}\pysiglinewithargsret{\sphinxbfcode{\sphinxupquote{class }}\sphinxcode{\sphinxupquote{corrosion.}}\sphinxbfcode{\sphinxupquote{Section\_loss\_Model}}}{\emph{\DUrole{n}{pars}}}{}
\sphinxAtStartPar
Bases: \sphinxcode{\sphinxupquote{object}}
\index{copy() (corrosion.Section\_loss\_Model method)@\spxentry{copy()}\spxextra{corrosion.Section\_loss\_Model method}}

\begin{fulllineitems}
\phantomsection\label{\detokenize{corrosion:corrosion.Section_loss_Model.copy}}\pysiglinewithargsret{\sphinxbfcode{\sphinxupquote{copy}}}{}{}
\sphinxAtStartPar
copy return a deep copy

\end{fulllineitems}

\index{postproc() (corrosion.Section\_loss\_Model method)@\spxentry{postproc()}\spxextra{corrosion.Section\_loss\_Model method}}

\begin{fulllineitems}
\phantomsection\label{\detokenize{corrosion:corrosion.Section_loss_Model.postproc}}\pysiglinewithargsret{\sphinxbfcode{\sphinxupquote{postproc}}}{\emph{\DUrole{n}{plot}\DUrole{o}{=}\DUrole{default_value}{False}}}{}
\sphinxAtStartPar
calculate the Pf and beta from accumulated section loss and section loss limit
\begin{quote}\begin{description}
\item[{Parameters}] \leavevmode
\sphinxAtStartPar
\sphinxstyleliteralstrong{\sphinxupquote{plot}} (\sphinxstyleliteralemphasis{\sphinxupquote{bool}}\sphinxstyleliteralemphasis{\sphinxupquote{, }}\sphinxstyleliteralemphasis{\sphinxupquote{optional}}) \textendash{} if true plot the R S curve, by default False

\end{description}\end{quote}

\end{fulllineitems}

\index{run() (corrosion.Section\_loss\_Model method)@\spxentry{run()}\spxextra{corrosion.Section\_loss\_Model method}}

\begin{fulllineitems}
\phantomsection\label{\detokenize{corrosion:corrosion.Section_loss_Model.run}}\pysiglinewithargsret{\sphinxbfcode{\sphinxupquote{run}}}{\emph{\DUrole{n}{t\_end}}}{}
\sphinxAtStartPar
run model to solve the accumulated section loss at t\_end by using x\_loss\_t\_fun()
\begin{quote}\begin{description}
\item[{Parameters}] \leavevmode
\sphinxAtStartPar
\sphinxstyleliteralstrong{\sphinxupquote{t\_end}} (\sphinxstyleliteralemphasis{\sphinxupquote{int}}\sphinxstyleliteralemphasis{\sphinxupquote{, }}\sphinxstyleliteralemphasis{\sphinxupquote{float}}) \textendash{} year

\end{description}\end{quote}

\end{fulllineitems}

\index{section\_loss\_with\_year() (corrosion.Section\_loss\_Model method)@\spxentry{section\_loss\_with\_year()}\spxextra{corrosion.Section\_loss\_Model method}}

\begin{fulllineitems}
\phantomsection\label{\detokenize{corrosion:corrosion.Section_loss_Model.section_loss_with_year}}\pysiglinewithargsret{\sphinxbfcode{\sphinxupquote{section\_loss\_with\_year}}}{\emph{\DUrole{n}{year\_lis}}, \emph{\DUrole{n}{plot}\DUrole{o}{=}\DUrole{default_value}{True}}, \emph{\DUrole{n}{amplify}\DUrole{o}{=}\DUrole{default_value}{1}}}{}
\sphinxAtStartPar
use x\_loss\_year() to report the accumulated section loss at each time step and
the corresponding Pf and beta.
\begin{quote}\begin{description}
\item[{Parameters}] \leavevmode\begin{itemize}
\item {} 
\sphinxAtStartPar
\sphinxstyleliteralstrong{\sphinxupquote{year\_lis}} (\sphinxstyleliteralemphasis{\sphinxupquote{list}}) \textendash{} a list of time step {[}year{]}

\item {} 
\sphinxAtStartPar
\sphinxstyleliteralstrong{\sphinxupquote{plot}} (\sphinxstyleliteralemphasis{\sphinxupquote{bool}}\sphinxstyleliteralemphasis{\sphinxupquote{, }}\sphinxstyleliteralemphasis{\sphinxupquote{optional}}) \textendash{} if true plot the RS pf beta with time, by default True

\item {} 
\sphinxAtStartPar
\sphinxstyleliteralstrong{\sphinxupquote{amplify}} (\sphinxstyleliteralemphasis{\sphinxupquote{int}}\sphinxstyleliteralemphasis{\sphinxupquote{, }}\sphinxstyleliteralemphasis{\sphinxupquote{optional}}) \textendash{} scale factor to adjust the height of the distribution curve, by default 1

\end{itemize}

\item[{Returns}] \leavevmode
\sphinxAtStartPar
(pf\_list, beta\_list)

\item[{Return type}] \leavevmode
\sphinxAtStartPar
tuple

\end{description}\end{quote}

\end{fulllineitems}


\end{fulllineitems}

\index{V\_m\_f() (in module corrosion)@\spxentry{V\_m\_f()}\spxextra{in module corrosion}}

\begin{fulllineitems}
\phantomsection\label{\detokenize{corrosion:corrosion.V_m_f}}\pysiglinewithargsret{\sphinxcode{\sphinxupquote{corrosion.}}\sphinxbfcode{\sphinxupquote{V\_m\_f}}}{\emph{\DUrole{n}{t}}, \emph{\DUrole{n}{w\_c}}, \emph{\DUrole{n}{cement\_type}}}{}
\sphinxAtStartPar
Calculate V\_m, a BET model parameter
\begin{quote}\begin{description}
\item[{Parameters}] \leavevmode\begin{itemize}
\item {} 
\sphinxAtStartPar
\sphinxstyleliteralstrong{\sphinxupquote{t}} (\sphinxstyleliteralemphasis{\sphinxupquote{curing time/concrete age}}\sphinxstyleliteralemphasis{\sphinxupquote{ {[}}}\sphinxstyleliteralemphasis{\sphinxupquote{day}}\sphinxstyleliteralemphasis{\sphinxupquote{{]}}}) \textendash{} 

\item {} 
\sphinxAtStartPar
\sphinxstyleliteralstrong{\sphinxupquote{w\_c}} (\sphinxstyleliteralemphasis{\sphinxupquote{water\sphinxhyphen{}cement ratio}}) \textendash{} 

\end{itemize}

\item[{Returns}] \leavevmode
\sphinxAtStartPar
V\_m : BET model parameter

\item[{Return type}] \leavevmode
\sphinxAtStartPar
numpy array

\end{description}\end{quote}

\begin{sphinxadmonition}{note}{Note:}
\sphinxAtStartPar
ASTM C150 cement type:

\sphinxAtStartPar
Cement Type          Description

\sphinxAtStartPar
Type I            :   Normal

\sphinxAtStartPar
Type II           :   Moderate Sulfate Resistance

\sphinxAtStartPar
Type II (MH)      :   Moderate Heat of Hydration (and Moderate Sulfate Resistance)

\sphinxAtStartPar
Type III          :   High Early Strength

\sphinxAtStartPar
Type IV           :   Low Heat Hydration

\sphinxAtStartPar
Type V            :   High Sulfate Resistance
\end{sphinxadmonition}

\end{fulllineitems}

\index{WaterbyMassHCP\_to\_RH() (in module corrosion)@\spxentry{WaterbyMassHCP\_to\_RH()}\spxextra{in module corrosion}}

\begin{fulllineitems}
\phantomsection\label{\detokenize{corrosion:corrosion.WaterbyMassHCP_to_RH}}\pysiglinewithargsret{\sphinxcode{\sphinxupquote{corrosion.}}\sphinxbfcode{\sphinxupquote{WaterbyMassHCP\_to\_RH}}}{\emph{\DUrole{n}{pars}}}{}
\sphinxAtStartPar
return RH in pores/environment from water content(g/g hardened cement paste) based on w\_c, cement\_type, Temperature
a reverse function of RH\_to\_WaterbyMassHCP()

\end{fulllineitems}

\index{WaterbyMassHCP\_to\_theta\_water() (in module corrosion)@\spxentry{WaterbyMassHCP\_to\_theta\_water()}\spxextra{in module corrosion}}

\begin{fulllineitems}
\phantomsection\label{\detokenize{corrosion:corrosion.WaterbyMassHCP_to_theta_water}}\pysiglinewithargsret{\sphinxcode{\sphinxupquote{corrosion.}}\sphinxbfcode{\sphinxupquote{WaterbyMassHCP\_to\_theta\_water}}}{\emph{\DUrole{n}{pars}}}{}
\sphinxAtStartPar
convert water content from g/g hardened cement paste(HCP)
to volumetric in HCP to volumetric in concrete

\end{fulllineitems}

\index{calibrate\_f() (in module corrosion)@\spxentry{calibrate\_f()}\spxextra{in module corrosion}}

\begin{fulllineitems}
\phantomsection\label{\detokenize{corrosion:corrosion.calibrate_f}}\pysiglinewithargsret{\sphinxcode{\sphinxupquote{corrosion.}}\sphinxbfcode{\sphinxupquote{calibrate\_f}}}{\emph{\DUrole{n}{raw\_model}}, \emph{\DUrole{n}{field\_data}}}{}
\sphinxAtStartPar
A placeholder function for future use. field\_data: temperature, theta\_water, icorr\_list

\end{fulllineitems}

\index{epsilon\_p\_f() (in module corrosion)@\spxentry{epsilon\_p\_f()}\spxextra{in module corrosion}}

\begin{fulllineitems}
\phantomsection\label{\detokenize{corrosion:corrosion.epsilon_p_f}}\pysiglinewithargsret{\sphinxcode{\sphinxupquote{corrosion.}}\sphinxbfcode{\sphinxupquote{epsilon\_p\_f}}}{\emph{\DUrole{n}{pars}}}{}
\sphinxAtStartPar
calculate the porosity of the hardened cement paste from the concrete porosity

\sphinxAtStartPar
pars : instance of Param object
\begin{quote}\begin{description}
\item[{Returns}] \leavevmode
\sphinxAtStartPar


\item[{Return type}] \leavevmode
\sphinxAtStartPar
float, numpy array

\end{description}\end{quote}

\begin{sphinxadmonition}{note}{Note:}
\sphinxAtStartPar
{[}TODO: when the concrete porosity is not known, the calculated porosity is time dependent at young age, a function of concrete mix and t{]}
\end{sphinxadmonition}

\end{fulllineitems}

\index{iL\_f() (in module corrosion)@\spxentry{iL\_f()}\spxextra{in module corrosion}}

\begin{fulllineitems}
\phantomsection\label{\detokenize{corrosion:corrosion.iL_f}}\pysiglinewithargsret{\sphinxcode{\sphinxupquote{corrosion.}}\sphinxbfcode{\sphinxupquote{iL\_f}}}{\emph{\DUrole{n}{pars}}}{}
\sphinxAtStartPar
calculate O2 limiting current density
:param pars: parameter object that contains the material properties
:type pars: instance of Param object
\begin{quote}\begin{description}
\item[{Returns}] \leavevmode
\sphinxAtStartPar
O2 limiting current density {[}A/m\textasciicircum{}2{]}

\item[{Return type}] \leavevmode
\sphinxAtStartPar
float, numpy array

\end{description}\end{quote}

\begin{sphinxadmonition}{note}{Note:}
\sphinxAtStartPar
intermidiate parameters
\begin{itemize}
\item {} 
\sphinxAtStartPar
z : number of charge, 4 for oxygen

\item {} 
\sphinxAtStartPar
delta : thickness of diffusion layer {[}m{]}

\item {} 
\sphinxAtStartPar
pars.De\_O2 : diffusivity {[}m\textasciicircum{}2/s{]}

\item {} 
\sphinxAtStartPar
pars.Cs\_g : bulk concentration {[}mol/m\textasciicircum{}3{]}

\item {} 
\sphinxAtStartPar
pars.epsilon\_g : gas phase fraction

\end{itemize}
\end{sphinxadmonition}
\begin{quote}\begin{description}
\item[{Returns}] \leavevmode
\sphinxAtStartPar
iL : current density over the steel concrete interface {[}A/m\textasciicircum{}2{]}

\item[{Return type}] \leavevmode
\sphinxAtStartPar
float, numpy array

\end{description}\end{quote}

\end{fulllineitems}

\index{icorr\_base() (in module corrosion)@\spxentry{icorr\_base()}\spxextra{in module corrosion}}

\begin{fulllineitems}
\phantomsection\label{\detokenize{corrosion:corrosion.icorr_base}}\pysiglinewithargsret{\sphinxcode{\sphinxupquote{corrosion.}}\sphinxbfcode{\sphinxupquote{icorr\_base}}}{\emph{\DUrole{n}{rho}}, \emph{\DUrole{n}{T}}, \emph{\DUrole{n}{iL}}, \emph{\DUrole{n}{d}}}{}
\sphinxAtStartPar
calculate averaged corrosion current density over the rebar\sphinxhyphen{}concrete surface from resistivity, temperature, limiting current and cover thickness
:param rho: resistivity {[}ohm.m{]}
:type rho: float, numpy array
:param T: temperature {[}K{]}
:type T: float, numpy array
:param iL: limiting current, oxygen diffusion {[} A/m\textasciicircum{}2{]}
:type iL: float, numpy array
:param d: concrete cover depth {[}m{]}
:type d: float, numpy array
\begin{quote}\begin{description}
\item[{Returns}] \leavevmode
\sphinxAtStartPar
icorr : corrosion current density, treated as uniform corrosion {[}A/m\textasciicircum{}2{]}

\item[{Return type}] \leavevmode
\sphinxAtStartPar
float array

\end{description}\end{quote}

\begin{sphinxadmonition}{note}{Note:}
\sphinxAtStartPar
reference: Pour\sphinxhyphen{}Ghaz, M., Isgor, O. B., \& Ghods, P. (2009)The effect of temperature on the corrosion of steel in concrete. Part 1: Simulated polarization resistance tests and model development. Corrosion Science, 51(2), 415\textendash{}425. \sphinxurl{https://doi.org/10.1016/j.corsci.2008.10.034}
parameters from ref
SI units
\end{sphinxadmonition}

\end{fulllineitems}

\index{icorr\_f() (in module corrosion)@\spxentry{icorr\_f()}\spxextra{in module corrosion}}

\begin{fulllineitems}
\phantomsection\label{\detokenize{corrosion:corrosion.icorr_f}}\pysiglinewithargsret{\sphinxcode{\sphinxupquote{corrosion.}}\sphinxbfcode{\sphinxupquote{icorr\_f}}}{\emph{\DUrole{n}{pars}}}{}
\sphinxAtStartPar
A wrapper of the icorr\_base() with modified parameters (resistivity rho \sphinxhyphen{}\textgreater{} volumetric water content, theta\_water)
by theta2rho\_fun().
\begin{quote}\begin{description}
\item[{Parameters}] \leavevmode
\sphinxAtStartPar
\sphinxstyleliteralstrong{\sphinxupquote{pars}} (\sphinxstyleliteralemphasis{\sphinxupquote{instance of Param class}}) \textendash{} \begin{itemize}
\item {} 
\sphinxAtStartPar
pars.theta\_water,

\item {} 
\sphinxAtStartPar
pars.T,

\item {} 
\sphinxAtStartPar
pars.iL,

\item {} 
\sphinxAtStartPar
pars.d,

\item {} 
\sphinxAtStartPar
pars.a,

\item {} 
\sphinxAtStartPar
pars.b

\end{itemize}


\item[{Returns}] \leavevmode
\sphinxAtStartPar
icorr : corrosion current density {[}A/m\textasciicircum{}2{]}

\item[{Return type}] \leavevmode
\sphinxAtStartPar
float, numpy array

\end{description}\end{quote}

\end{fulllineitems}

\index{icorr\_to\_mmpy() (in module corrosion)@\spxentry{icorr\_to\_mmpy()}\spxextra{in module corrosion}}

\begin{fulllineitems}
\phantomsection\label{\detokenize{corrosion:corrosion.icorr_to_mmpy}}\pysiglinewithargsret{\sphinxcode{\sphinxupquote{corrosion.}}\sphinxbfcode{\sphinxupquote{icorr\_to\_mmpy}}}{\emph{\DUrole{n}{icorr}}}{}
\sphinxAtStartPar
icorr\_to\_mmpy converts icorr {[}A/m\textasciicircum{}2{]} to corrosion rate{[}mm/year{]} using Faraday’s laws
\begin{quote}\begin{description}
\item[{Parameters}] \leavevmode
\sphinxAtStartPar
\sphinxstyleliteralstrong{\sphinxupquote{icorr}} (\sphinxstyleliteralemphasis{\sphinxupquote{float}}) \textendash{} corrosion current density {[}A/m\textasciicircum{}2{]}

\item[{Returns}] \leavevmode
\sphinxAtStartPar
corrosion rate, section loss {[}mm/year{]}

\item[{Return type}] \leavevmode
\sphinxAtStartPar
float

\end{description}\end{quote}

\end{fulllineitems}

\index{k\_f() (in module corrosion)@\spxentry{k\_f()}\spxextra{in module corrosion}}

\begin{fulllineitems}
\phantomsection\label{\detokenize{corrosion:corrosion.k_f}}\pysiglinewithargsret{\sphinxcode{\sphinxupquote{corrosion.}}\sphinxbfcode{\sphinxupquote{k\_f}}}{\emph{\DUrole{n}{C\_mean}}, \emph{\DUrole{n}{w\_c}}, \emph{\DUrole{n}{t}}, \emph{\DUrole{n}{cement\_type}}}{}
\sphinxAtStartPar
returns BET model parameter k

\end{fulllineitems}

\index{mmpy\_to\_icorr() (in module corrosion)@\spxentry{mmpy\_to\_icorr()}\spxextra{in module corrosion}}

\begin{fulllineitems}
\phantomsection\label{\detokenize{corrosion:corrosion.mmpy_to_icorr}}\pysiglinewithargsret{\sphinxcode{\sphinxupquote{corrosion.}}\sphinxbfcode{\sphinxupquote{mmpy\_to\_icorr}}}{\emph{\DUrole{n}{rate}}}{}
\sphinxAtStartPar
mmpy\_to\_icorr converts corrosion rate{[}mm/year{]} to icorr {[}A/m\textasciicircum{}2{]} using Faraday’s laws
\begin{quote}\begin{description}
\item[{Parameters}] \leavevmode
\sphinxAtStartPar
\sphinxstyleliteralstrong{\sphinxupquote{rate}} (\sphinxstyleliteralemphasis{\sphinxupquote{float}}) \textendash{} corrosion rate, section loss {[}mm/year{]}

\item[{Returns}] \leavevmode
\sphinxAtStartPar
corrosion current density {[}A/m\textasciicircum{}2{]}

\item[{Return type}] \leavevmode
\sphinxAtStartPar
float

\end{description}\end{quote}

\end{fulllineitems}

\index{theta2rho\_fun() (in module corrosion)@\spxentry{theta2rho\_fun()}\spxextra{in module corrosion}}

\begin{fulllineitems}
\phantomsection\label{\detokenize{corrosion:corrosion.theta2rho_fun}}\pysiglinewithargsret{\sphinxcode{\sphinxupquote{corrosion.}}\sphinxbfcode{\sphinxupquote{theta2rho\_fun}}}{\emph{\DUrole{n}{theta\_water}}, \emph{\DUrole{n}{a}}, \emph{\DUrole{n}{b}}}{}
\sphinxAtStartPar
volumetric water content to resistivity, index regression function used

\end{fulllineitems}

\index{theta\_water\_to\_WaterbyMassHCP() (in module corrosion)@\spxentry{theta\_water\_to\_WaterbyMassHCP()}\spxextra{in module corrosion}}

\begin{fulllineitems}
\phantomsection\label{\detokenize{corrosion:corrosion.theta_water_to_WaterbyMassHCP}}\pysiglinewithargsret{\sphinxcode{\sphinxupquote{corrosion.}}\sphinxbfcode{\sphinxupquote{theta\_water\_to\_WaterbyMassHCP}}}{\emph{\DUrole{n}{pars}}}{}
\sphinxAtStartPar
convert water content from volumetric by concrete to volumetric in HCP  to g/g  inHCP
a reverse function of WaterbyMassHCP\_to\_theta\_water()

\end{fulllineitems}

\index{x\_loss\_t\_fun() (in module corrosion)@\spxentry{x\_loss\_t\_fun()}\spxextra{in module corrosion}}

\begin{fulllineitems}
\phantomsection\label{\detokenize{corrosion:corrosion.x_loss_t_fun}}\pysiglinewithargsret{\sphinxcode{\sphinxupquote{corrosion.}}\sphinxbfcode{\sphinxupquote{x\_loss\_t\_fun}}}{\emph{\DUrole{n}{t\_end}}, \emph{\DUrole{n}{n\_step}}, \emph{\DUrole{n}{x\_loss\_rate}}, \emph{\DUrole{n}{p\_active\_t\_curve}}}{}
\sphinxAtStartPar
x\_loss\_t\_fun returns x\_loss samples at a SINGLE given time t\_end. the samples represents distribution of all possible x\_loss with
different corrosion history
\begin{quote}\begin{description}
\item[{Parameters}] \leavevmode\begin{itemize}
\item {} 
\sphinxAtStartPar
\sphinxstyleliteralstrong{\sphinxupquote{t\_end}} (\sphinxstyleliteralemphasis{\sphinxupquote{int}}\sphinxstyleliteralemphasis{\sphinxupquote{, }}\sphinxstyleliteralemphasis{\sphinxupquote{float}}) \textendash{} year in which the x\_loss is reported

\item {} 
\sphinxAtStartPar
\sphinxstyleliteralstrong{\sphinxupquote{n\_step}} (\sphinxstyleliteralemphasis{\sphinxupquote{int}}) \textendash{} number of time steps

\item {} 
\sphinxAtStartPar
\sphinxstyleliteralstrong{\sphinxupquote{r\_corr\_mean}} (\sphinxstyleliteralemphasis{\sphinxupquote{float}}) \textendash{} averaged corrosion rate i.e. x\sphinxhyphen{}loss rate

\item {} 
\sphinxAtStartPar
\sphinxstyleliteralstrong{\sphinxupquote{p\_active\_t\_curve}} (\sphinxstyleliteralemphasis{\sphinxupquote{tuple}}) \textendash{} (t\_lis\_curve, pf\_lis\_curve)

\end{itemize}

\item[{Returns}] \leavevmode
\sphinxAtStartPar
section loss at t\_end year, a large sample from the distribution

\item[{Return type}] \leavevmode
\sphinxAtStartPar
numpy array

\end{description}\end{quote}

\end{fulllineitems}

\index{x\_loss\_year() (in module corrosion)@\spxentry{x\_loss\_year()}\spxextra{in module corrosion}}

\begin{fulllineitems}
\phantomsection\label{\detokenize{corrosion:corrosion.x_loss_year}}\pysiglinewithargsret{\sphinxcode{\sphinxupquote{corrosion.}}\sphinxbfcode{\sphinxupquote{x\_loss\_year}}}{\emph{\DUrole{n}{model}}, \emph{\DUrole{n}{year\_lis}}, \emph{\DUrole{n}{plot}\DUrole{o}{=}\DUrole{default_value}{True}}, \emph{\DUrole{n}{amplify}\DUrole{o}{=}\DUrole{default_value}{80}}}{}
\sphinxAtStartPar
run x\_loss\_t\_fun() function over time

\end{fulllineitems}



\section{cracking module}
\label{\detokenize{cracking:module-cracking}}\label{\detokenize{cracking:cracking-module}}\label{\detokenize{cracking::doc}}\index{module@\spxentry{module}!cracking@\spxentry{cracking}}\index{cracking@\spxentry{cracking}!module@\spxentry{module}}
\sphinxAtStartPar
\sphinxstylestrong{Summary}

\sphinxAtStartPar
Thick\sphinxhyphen{}walled expansive cylinder model to calculate internal
stress strain through the concrete cover and the location of the crack tip
\begin{itemize}
\item {} 
\sphinxAtStartPar
\sphinxstylestrong{Resistance}:       cover depth

\item {} 
\sphinxAtStartPar
\sphinxstylestrong{Load}:             crack length

\item {} 
\sphinxAtStartPar
\sphinxstylestrong{limit\sphinxhyphen{}state}:      crack length = cover depth

\item {} \begin{description}
\item[{\sphinxstylestrong{Field data}:       concrete mechanical properties}] \leavevmode
\sphinxAtStartPar
(compressive, tensile strength Young’s modulus)
delamination, visible crack ratio

\end{description}

\end{itemize}
\index{Cracking\_Model (class in cracking)@\spxentry{Cracking\_Model}\spxextra{class in cracking}}

\begin{fulllineitems}
\phantomsection\label{\detokenize{cracking:cracking.Cracking_Model}}\pysiglinewithargsret{\sphinxbfcode{\sphinxupquote{class }}\sphinxcode{\sphinxupquote{cracking.}}\sphinxbfcode{\sphinxupquote{Cracking\_Model}}}{\emph{\DUrole{n}{pars}}}{}
\sphinxAtStartPar
Bases: \sphinxcode{\sphinxupquote{object}}
\index{copy() (cracking.Cracking\_Model method)@\spxentry{copy()}\spxextra{cracking.Cracking\_Model method}}

\begin{fulllineitems}
\phantomsection\label{\detokenize{cracking:cracking.Cracking_Model.copy}}\pysiglinewithargsret{\sphinxbfcode{\sphinxupquote{copy}}}{}{}
\sphinxAtStartPar
create a deepcopy

\end{fulllineitems}

\index{postproc() (cracking.Cracking\_Model method)@\spxentry{postproc()}\spxextra{cracking.Cracking\_Model method}}

\begin{fulllineitems}
\phantomsection\label{\detokenize{cracking:cracking.Cracking_Model.postproc}}\pysiglinewithargsret{\sphinxbfcode{\sphinxupquote{postproc}}}{}{}
\sphinxAtStartPar
calculate the crack length and surface crack rate

\end{fulllineitems}

\index{run() (cracking.Cracking\_Model method)@\spxentry{run()}\spxextra{cracking.Cracking\_Model method}}

\begin{fulllineitems}
\phantomsection\label{\detokenize{cracking:cracking.Cracking_Model.run}}\pysiglinewithargsret{\sphinxbfcode{\sphinxupquote{run}}}{\emph{\DUrole{n}{stochastic}\DUrole{o}{=}\DUrole{default_value}{True}}}{}
\sphinxAtStartPar
Solve stress and strain and crack tip location in concrete cover

\end{fulllineitems}


\end{fulllineitems}

\index{bilinear\_stress\_strain() (in module cracking)@\spxentry{bilinear\_stress\_strain()}\spxextra{in module cracking}}

\begin{fulllineitems}
\phantomsection\label{\detokenize{cracking:cracking.bilinear_stress_strain}}\pysiglinewithargsret{\sphinxcode{\sphinxupquote{cracking.}}\sphinxbfcode{\sphinxupquote{bilinear\_stress\_strain}}}{\emph{\DUrole{n}{epsilon\_theta}}, \emph{\DUrole{n}{f\_t}}, \emph{\DUrole{n}{E\_0}}}{}
\sphinxAtStartPar
returns the stress in concrete from strain using the bilinear stress\sphinxhyphen{}strain curve
\begin{quote}\begin{description}
\item[{Parameters}] \leavevmode\begin{itemize}
\item {} 
\sphinxAtStartPar
\sphinxstyleliteralstrong{\sphinxupquote{epsilon\_theta}} (\sphinxstyleliteralemphasis{\sphinxupquote{numpy array}}) \textendash{} strain{[}\sphinxhyphen{}{]}

\item {} 
\sphinxAtStartPar
\sphinxstyleliteralstrong{\sphinxupquote{f\_t}} (\sphinxstyleliteralemphasis{\sphinxupquote{numpy array}}) \textendash{} cracking tensile strength{[}MPa{]}

\item {} 
\sphinxAtStartPar
\sphinxstyleliteralstrong{\sphinxupquote{E\_0}} (\sphinxstyleliteralemphasis{\sphinxupquote{numpy array}}) \textendash{} modulus of elasticity{[}MPa{]}

\end{itemize}

\item[{Returns}] \leavevmode
\sphinxAtStartPar
stress{[}MPa{]}

\item[{Return type}] \leavevmode
\sphinxAtStartPar
numpy array

\end{description}\end{quote}

\begin{sphinxadmonition}{note}{Note:}
\sphinxAtStartPar
TODO: modulus of elasticity reduction due to creep
\end{sphinxadmonition}

\end{fulllineitems}

\index{crack\_width\_open() (in module cracking)@\spxentry{crack\_width\_open()}\spxextra{in module cracking}}

\begin{fulllineitems}
\phantomsection\label{\detokenize{cracking:cracking.crack_width_open}}\pysiglinewithargsret{\sphinxcode{\sphinxupquote{cracking.}}\sphinxbfcode{\sphinxupquote{crack\_width\_open}}}{\emph{\DUrole{n}{a}}, \emph{\DUrole{n}{b}}, \emph{\DUrole{n}{u\_st}}, \emph{\DUrole{n}{f\_t}}, \emph{\DUrole{n}{E\_0}}}{}
\sphinxAtStartPar
calculate crack opening on the concret cover surface
\begin{quote}\begin{description}
\item[{Parameters}] \leavevmode\begin{itemize}
\item {} 
\sphinxAtStartPar
\sphinxstyleliteralstrong{\sphinxupquote{a}} (\sphinxstyleliteralemphasis{\sphinxupquote{numpy array}}) \textendash{} inner radius boundary of the rust (center of rebar to rust\sphinxhyphen{}concrete) {[}m{]}

\item {} 
\sphinxAtStartPar
\sphinxstyleliteralstrong{\sphinxupquote{b}} (\sphinxstyleliteralemphasis{\sphinxupquote{numpy array}}) \textendash{} outer radius boundary of the concrete (center of rebar to cover surface) {[}m{]}

\item {} 
\sphinxAtStartPar
\sphinxstyleliteralstrong{\sphinxupquote{u\_st}} (\sphinxstyleliteralemphasis{\sphinxupquote{numpy array}}) \textendash{} rust expansion(to original rebar surface) beyond the porous zone {[}m{]}

\item {} 
\sphinxAtStartPar
\sphinxstyleliteralstrong{\sphinxupquote{f\_t}} (\sphinxstyleliteralemphasis{\sphinxupquote{numpy array}}) \textendash{} ultimate tensile strength {[}MPa{]}

\item {} 
\sphinxAtStartPar
\sphinxstyleliteralstrong{\sphinxupquote{E\_0}} (\sphinxstyleliteralemphasis{\sphinxupquote{numpy array}}) \textendash{} modulus of elasticity {[}MPa{]}

\end{itemize}

\item[{Returns}] \leavevmode
\sphinxAtStartPar
sample of crack opening on the concret cover surface

\item[{Return type}] \leavevmode
\sphinxAtStartPar
numpy array

\end{description}\end{quote}

\end{fulllineitems}

\index{solve\_stress\_strain\_crack\_deterministic() (in module cracking)@\spxentry{solve\_stress\_strain\_crack\_deterministic()}\spxextra{in module cracking}}

\begin{fulllineitems}
\phantomsection\label{\detokenize{cracking:cracking.solve_stress_strain_crack_deterministic}}\pysiglinewithargsret{\sphinxcode{\sphinxupquote{cracking.}}\sphinxbfcode{\sphinxupquote{solve\_stress\_strain\_crack\_deterministic}}}{\emph{\DUrole{n}{pars}}, \emph{\DUrole{n}{number\_of\_points}\DUrole{o}{=}\DUrole{default_value}{100}}}{}
\sphinxAtStartPar
solve the stress and strain along the polar axis using strain\_stress\_crack\_f().
One deterministic solution is returned by the means of all input variables.
\begin{quote}\begin{description}
\item[{Parameters}] \leavevmode\begin{itemize}
\item {} 
\sphinxAtStartPar
\sphinxstyleliteralstrong{\sphinxupquote{pars}} (\sphinxstyleliteralemphasis{\sphinxupquote{Param object instance}}) \textendash{} an object instance containing material properties

\item {} 
\sphinxAtStartPar
\sphinxstyleliteralstrong{\sphinxupquote{number\_of\_points}} (\sphinxstyleliteralemphasis{\sphinxupquote{int}}\sphinxstyleliteralemphasis{\sphinxupquote{, }}\sphinxstyleliteralemphasis{\sphinxupquote{optional}}) \textendash{} number of points where the stress and strain is reported along
the polar axis, by default 100

\end{itemize}

\item[{Returns}] \leavevmode
\sphinxAtStartPar

\sphinxAtStartPar
(epsilon\_theta, sigma\_theta, rust\_thickness,
crack\_condition, R\_c, w\_open)

\sphinxAtStartPar
(strain, stress, rust thickness,
crack condition code, crack front cooridnate, open crack width)


\item[{Return type}] \leavevmode
\sphinxAtStartPar
tuple

\end{description}\end{quote}

\end{fulllineitems}

\index{solve\_stress\_strain\_crack\_stochastic() (in module cracking)@\spxentry{solve\_stress\_strain\_crack\_stochastic()}\spxextra{in module cracking}}

\begin{fulllineitems}
\phantomsection\label{\detokenize{cracking:cracking.solve_stress_strain_crack_stochastic}}\pysiglinewithargsret{\sphinxcode{\sphinxupquote{cracking.}}\sphinxbfcode{\sphinxupquote{solve\_stress\_strain\_crack\_stochastic}}}{\emph{\DUrole{n}{pars}}, \emph{\DUrole{n}{number\_of\_points}\DUrole{o}{=}\DUrole{default_value}{100}}}{}
\sphinxAtStartPar
solve the stress and strain along the polar axis using strain\_stress\_crack\_f().
the stochastic solution matrix is returned, where each row represents a deterministic solution
\begin{quote}\begin{description}
\item[{Parameters}] \leavevmode\begin{itemize}
\item {} 
\sphinxAtStartPar
\sphinxstyleliteralstrong{\sphinxupquote{pars}} (\sphinxstyleliteralemphasis{\sphinxupquote{Param object instance}}) \textendash{} an object instance containing material properties

\item {} 
\sphinxAtStartPar
\sphinxstyleliteralstrong{\sphinxupquote{number\_of\_points}} (\sphinxstyleliteralemphasis{\sphinxupquote{int}}\sphinxstyleliteralemphasis{\sphinxupquote{, }}\sphinxstyleliteralemphasis{\sphinxupquote{optional}}) \textendash{} number of points where the stress and strain is reported along
the polar axis, by default 100

\end{itemize}

\item[{Returns}] \leavevmode
\sphinxAtStartPar

\sphinxAtStartPar
(epsilon\_theta, sigma\_theta, rust\_thickness,
crack\_condition, R\_c, w\_open)

\sphinxAtStartPar
(strain, stress, rust thickness,
crack condition code, crack front cooridnate, open crack width)


\item[{Return type}] \leavevmode
\sphinxAtStartPar
tuple

\end{description}\end{quote}

\end{fulllineitems}

\index{strain\_f() (in module cracking)@\spxentry{strain\_f()}\spxextra{in module cracking}}

\begin{fulllineitems}
\phantomsection\label{\detokenize{cracking:cracking.strain_f}}\pysiglinewithargsret{\sphinxcode{\sphinxupquote{cracking.}}\sphinxbfcode{\sphinxupquote{strain\_f}}}{\emph{\DUrole{n}{r}}, \emph{\DUrole{n}{a}}, \emph{\DUrole{n}{b}}, \emph{\DUrole{n}{u\_st}}, \emph{\DUrole{n}{f\_t}}, \emph{\DUrole{n}{E\_0}}, \emph{\DUrole{n}{crack\_condition}}}{}
\sphinxAtStartPar
strain\_f returns the strain along the polar axis r, a\textless{}=r\textless{}=b,
fully vectorized with numpy funcions
\begin{quote}\begin{description}
\item[{Parameters}] \leavevmode\begin{itemize}
\item {} 
\sphinxAtStartPar
\sphinxstyleliteralstrong{\sphinxupquote{r}} (\sphinxstyleliteralemphasis{\sphinxupquote{2D numpy array}}) \textendash{} coordinate along the polar axis, a matrix with rows representing each r grid,
column number is repeated values {[}m{]}

\item {} 
\sphinxAtStartPar
\sphinxstyleliteralstrong{\sphinxupquote{a}} (\sphinxstyleliteralemphasis{\sphinxupquote{numpy array}}) \textendash{} inner radius boundary of the rust
(center of rebar to rust\sphinxhyphen{}concrete interface) {[}m{]}

\item {} 
\sphinxAtStartPar
\sphinxstyleliteralstrong{\sphinxupquote{b}} (\sphinxstyleliteralemphasis{\sphinxupquote{numpy array}}) \textendash{} outer radius boundary of the concrete
(center of rebar to cover surface) {[}m{]}

\item {} 
\sphinxAtStartPar
\sphinxstyleliteralstrong{\sphinxupquote{u\_st}} (\sphinxstyleliteralemphasis{\sphinxupquote{numpy array}}) \textendash{} rust expansion(to original rebar surface) beyond the porous zone {[}m{]}

\item {} 
\sphinxAtStartPar
\sphinxstyleliteralstrong{\sphinxupquote{f\_t}} (\sphinxstyleliteralemphasis{\sphinxupquote{array}}) \textendash{} ultimate tensile strength {[}MPa{]}

\item {} 
\sphinxAtStartPar
\sphinxstyleliteralstrong{\sphinxupquote{E\_0}} (\sphinxstyleliteralemphasis{\sphinxupquote{array}}) \textendash{} modulus of elasticity {[}MPa{]}

\item {} 
\sphinxAtStartPar
\sphinxstyleliteralstrong{\sphinxupquote{crack\_condition}} (\sphinxstyleliteralemphasis{\sphinxupquote{array}}) \textendash{} 
\sphinxAtStartPar
crack\_condition array {[}int{]}. Each element corresponds to the condition of each row of the matrix
\begin{itemize}
\item {} 
\sphinxAtStartPar
0 ‘sound cover’

\item {} 
\sphinxAtStartPar
1 ‘partially cracked’

\item {} 
\sphinxAtStartPar
2 ‘fully cracked’

\end{itemize}


\end{itemize}

\item[{Returns}] \leavevmode
\sphinxAtStartPar
strain, epsilon\_theta matrix, row is the strain along the polar axis

\item[{Return type}] \leavevmode
\sphinxAtStartPar
2D numpy array

\end{description}\end{quote}

\end{fulllineitems}

\index{strain\_stress\_crack\_f() (in module cracking)@\spxentry{strain\_stress\_crack\_f()}\spxextra{in module cracking}}

\begin{fulllineitems}
\phantomsection\label{\detokenize{cracking:cracking.strain_stress_crack_f}}\pysiglinewithargsret{\sphinxcode{\sphinxupquote{cracking.}}\sphinxbfcode{\sphinxupquote{strain\_stress\_crack\_f}}}{\emph{\DUrole{n}{r}}, \emph{\DUrole{n}{r0\_bar}}, \emph{\DUrole{n}{x\_loss}}, \emph{\DUrole{n}{cover}}, \emph{\DUrole{n}{f\_t}}, \emph{\DUrole{n}{E\_0}}, \emph{\DUrole{n}{w\_c}}, \emph{\DUrole{n}{r\_v}}, \emph{\DUrole{n}{plot}\DUrole{o}{=}\DUrole{default_value}{False}}, \emph{\DUrole{n}{ax}\DUrole{o}{=}\DUrole{default_value}{None}}}{}~\begin{description}
\item[{calculate the stress, strain, crack\_condition for the whole concrete cover}] \leavevmode
\sphinxAtStartPar
(fully vectorized with numpy matrix funcions)

\end{description}
\begin{quote}\begin{description}
\item[{Parameters}] \leavevmode\begin{itemize}
\item {} 
\sphinxAtStartPar
\sphinxstyleliteralstrong{\sphinxupquote{r}} (\sphinxstyleliteralemphasis{\sphinxupquote{2D numpy array}}) \textendash{} coordinate along the polar axis, a matrix with rows representing each r grid,
column number is repeated values {[}m{]}

\item {} 
\sphinxAtStartPar
\sphinxstyleliteralstrong{\sphinxupquote{r0\_bar}} (\sphinxstyleliteralemphasis{\sphinxupquote{numpy array}}) \textendash{} original rebar radius {[}m{]}

\item {} 
\sphinxAtStartPar
\sphinxstyleliteralstrong{\sphinxupquote{x\_loss}} (\sphinxstyleliteralemphasis{\sphinxupquote{numpy array}}) \textendash{} section loss of the steel due to corrosion

\item {} 
\sphinxAtStartPar
\sphinxstyleliteralstrong{\sphinxupquote{cover}} (\sphinxstyleliteralemphasis{\sphinxupquote{numpy array}}) \textendash{} concrete cover depth {[}m{]}

\item {} 
\sphinxAtStartPar
\sphinxstyleliteralstrong{\sphinxupquote{f\_t}} (\sphinxstyleliteralemphasis{\sphinxupquote{array}}) \textendash{} ultimate tensile strength {[}MPa{]}

\item {} 
\sphinxAtStartPar
\sphinxstyleliteralstrong{\sphinxupquote{E\_0}} (\sphinxstyleliteralemphasis{\sphinxupquote{array}}) \textendash{} modulus of elasticity {[}MPa{]}

\item {} 
\sphinxAtStartPar
\sphinxstyleliteralstrong{\sphinxupquote{w\_c}} (\sphinxstyleliteralemphasis{\sphinxupquote{float}}\sphinxstyleliteralemphasis{\sphinxupquote{, }}\sphinxstyleliteralemphasis{\sphinxupquote{array}}) \textendash{} water cement ratio

\item {} 
\sphinxAtStartPar
\sphinxstyleliteralstrong{\sphinxupquote{r\_v}} (\sphinxstyleliteralemphasis{\sphinxupquote{numpy array}}) \textendash{} expansion rate r\_v ranges from 2 to 6.5 times

\item {} 
\sphinxAtStartPar
\sphinxstyleliteralstrong{\sphinxupquote{plot}} (\sphinxstyleliteralemphasis{\sphinxupquote{bool}}\sphinxstyleliteralemphasis{\sphinxupquote{, }}\sphinxstyleliteralemphasis{\sphinxupquote{optional}}) \textendash{} if true, plot the stress and strain along r, by default False

\item {} 
\sphinxAtStartPar
\sphinxstyleliteralstrong{\sphinxupquote{ax}} (\sphinxstyleliteralemphasis{\sphinxupquote{axis instance}}) \textendash{} subplot axis, by default None

\end{itemize}

\item[{Returns}] \leavevmode
\sphinxAtStartPar

\sphinxAtStartPar
(epsilon\_theta, sigma\_theta, rust\_thickness,
crack\_condition, R\_c, w\_open)

\sphinxAtStartPar
(strain, stress, rust thickness,
crack condition code, crack front cooridnate, open crack width)


\item[{Return type}] \leavevmode
\sphinxAtStartPar
tuple

\end{description}\end{quote}

\begin{sphinxadmonition}{note}{Note:}
\sphinxAtStartPar
Vectorization:
r is a matrix. Other material property parameters(such as E) are 1\sphinxhyphen{}D arrays (to be converted to column vector in the calculation)
\end{sphinxadmonition}

\end{fulllineitems}



\section{helper\_func module}
\label{\detokenize{helper_func:module-helper_func}}\label{\detokenize{helper_func:helper-func-module}}\label{\detokenize{helper_func::doc}}\index{module@\spxentry{module}!helper\_func@\spxentry{helper\_func}}\index{helper\_func@\spxentry{helper\_func}!module@\spxentry{module}}
\sphinxAtStartPar
\sphinxstylestrong{Summary}

\sphinxAtStartPar
The helper module is designed to handle the repeated math operations that are not directly related to the mechanistic model calculation. These operations include the following
\begin{itemize}
\item {} 
\sphinxAtStartPar
distribution sampling from a distribution (uniform, beta)

\item {} 
\sphinxAtStartPar
distribution curve fitting to data with an analytical or a numerical method

\item {} 
\sphinxAtStartPar
interpolation function for data tables

\item {} 
\sphinxAtStartPar
numerical integration for probability density functions

\item {} 
\sphinxAtStartPar
reliability probability calculation

\item {} 
\sphinxAtStartPar
statistical calculation to find mean and standard distribution ignoring not\sphinxhyphen{}a\sphinxhyphen{}number (nan).

\item {} 
\sphinxAtStartPar
figure sub\sphinxhyphen{}plotting

\end{itemize}
\index{Beta\_custom() (in module helper\_func)@\spxentry{Beta\_custom()}\spxextra{in module helper\_func}}

\begin{fulllineitems}
\phantomsection\label{\detokenize{helper_func:helper_func.Beta_custom}}\pysiglinewithargsret{\sphinxcode{\sphinxupquote{helper\_func.}}\sphinxbfcode{\sphinxupquote{Beta\_custom}}}{\emph{\DUrole{n}{m}}, \emph{\DUrole{n}{s}}, \emph{\DUrole{n}{a}}, \emph{\DUrole{n}{b}}, \emph{\DUrole{n}{n\_sample}\DUrole{o}{=}\DUrole{default_value}{100000}}, \emph{\DUrole{n}{plot}\DUrole{o}{=}\DUrole{default_value}{False}}}{}
\sphinxAtStartPar
Beta\_custom draws samples from a general beta distribution described by mean, std and lower and upper   bounds
X\textasciitilde{}General Beta(a,b, loc = c, scale = d)
Z\textasciitilde{}std Beta(alpha, beta)

\sphinxAtStartPar
X = c + d*Z

\sphinxAtStartPar
E(X) = c + d * E(Z)

\sphinxAtStartPar
var(X) = d\textasciicircum{}2 * var(Z)
\begin{quote}\begin{description}
\item[{Parameters}] \leavevmode\begin{itemize}
\item {} 
\sphinxAtStartPar
\sphinxstyleliteralstrong{\sphinxupquote{m}} (\sphinxstyleliteralemphasis{\sphinxupquote{mean}}) \textendash{} 

\item {} 
\sphinxAtStartPar
\sphinxstyleliteralstrong{\sphinxupquote{s}} (\sphinxstyleliteralemphasis{\sphinxupquote{standard deviation}}) \textendash{} 

\item {} 
\sphinxAtStartPar
\sphinxstyleliteralstrong{\sphinxupquote{a}} (\sphinxstyleliteralemphasis{\sphinxupquote{lower bound}}\sphinxstyleliteralemphasis{\sphinxupquote{, }}\sphinxstyleliteralemphasis{\sphinxupquote{not shape param a}}\sphinxstyleliteralemphasis{\sphinxupquote{(}}\sphinxstyleliteralemphasis{\sphinxupquote{alpha}}\sphinxstyleliteralemphasis{\sphinxupquote{)}}) \textendash{} 

\item {} 
\sphinxAtStartPar
\sphinxstyleliteralstrong{\sphinxupquote{b}} (\sphinxstyleliteralemphasis{\sphinxupquote{upper bound}}\sphinxstyleliteralemphasis{\sphinxupquote{, }}\sphinxstyleliteralemphasis{\sphinxupquote{not shape param b}}\sphinxstyleliteralemphasis{\sphinxupquote{(}}\sphinxstyleliteralemphasis{\sphinxupquote{beta}}\sphinxstyleliteralemphasis{\sphinxupquote{)}}) \textendash{} 

\item {} 
\sphinxAtStartPar
\sphinxstyleliteralstrong{\sphinxupquote{n\_sample}} (\sphinxstyleliteralemphasis{\sphinxupquote{int}}) \textendash{} sample number

\item {} 
\sphinxAtStartPar
\sphinxstyleliteralstrong{\sphinxupquote{plot}} (\sphinxstyleliteralemphasis{\sphinxupquote{bool}}) \textendash{} default is False

\end{itemize}

\item[{Returns}] \leavevmode
\sphinxAtStartPar
sample array from the distribution

\item[{Return type}] \leavevmode
\sphinxAtStartPar
numpy array

\end{description}\end{quote}

\end{fulllineitems}

\index{Fit\_distrib() (in module helper\_func)@\spxentry{Fit\_distrib()}\spxextra{in module helper\_func}}

\begin{fulllineitems}
\phantomsection\label{\detokenize{helper_func:helper_func.Fit_distrib}}\pysiglinewithargsret{\sphinxcode{\sphinxupquote{helper\_func.}}\sphinxbfcode{\sphinxupquote{Fit\_distrib}}}{\emph{\DUrole{n}{s}}, \emph{\DUrole{n}{fit\_type}\DUrole{o}{=}\DUrole{default_value}{\textquotesingle{}kernel\textquotesingle{}}}, \emph{\DUrole{n}{plot}\DUrole{o}{=}\DUrole{default_value}{False}}, \emph{\DUrole{n}{xlabel}\DUrole{o}{=}\DUrole{default_value}{\textquotesingle{}\textquotesingle{}}}, \emph{\DUrole{n}{title}\DUrole{o}{=}\DUrole{default_value}{\textquotesingle{}\textquotesingle{}}}, \emph{\DUrole{n}{axn}\DUrole{o}{=}\DUrole{default_value}{None}}}{}
\sphinxAtStartPar
fit data to a probability distribution function(parametric or numerical)
and return a continuous random variable or a random variable represented by Gaussian kernels
parametric : normal
numerical : Gaussian kernels
\begin{quote}\begin{description}
\item[{Parameters}] \leavevmode\begin{itemize}
\item {} 
\sphinxAtStartPar
\sphinxstyleliteralstrong{\sphinxupquote{s}} (\sphinxstyleliteralemphasis{\sphinxupquote{array\sphinxhyphen{}like}}) \textendash{} sample data

\item {} 
\sphinxAtStartPar
\sphinxstyleliteralstrong{\sphinxupquote{fit\_type}} (\sphinxstyleliteralemphasis{\sphinxupquote{string}}) \textendash{} fit type keywords, ‘kernel’, ‘normal’

\item {} 
\sphinxAtStartPar
\sphinxstyleliteralstrong{\sphinxupquote{plot}} (\sphinxstyleliteralemphasis{\sphinxupquote{bool}}) \textendash{} when True, create a plot with histogram and fitted pdf curve

\end{itemize}

\item[{Returns}] \leavevmode
\sphinxAtStartPar
\begin{description}
\item[{when parametric normal is used}] \leavevmode
\sphinxAtStartPar
continuous random variable : stats.norm(loc = mu, scale = sigma)

\item[{when kernel is used}] \leavevmode
\sphinxAtStartPar
Gaussian kernel random variable : (stats.gaussian\_kde)

\end{description}


\item[{Return type}] \leavevmode
\sphinxAtStartPar
instance of random variable

\end{description}\end{quote}

\end{fulllineitems}

\index{Get\_mean() (in module helper\_func)@\spxentry{Get\_mean()}\spxextra{in module helper\_func}}

\begin{fulllineitems}
\phantomsection\label{\detokenize{helper_func:helper_func.Get_mean}}\pysiglinewithargsret{\sphinxcode{\sphinxupquote{helper\_func.}}\sphinxbfcode{\sphinxupquote{Get\_mean}}}{\emph{\DUrole{n}{x}}}{}
\sphinxAtStartPar
get mean ignoring nans

\end{fulllineitems}

\index{Get\_std() (in module helper\_func)@\spxentry{Get\_std()}\spxextra{in module helper\_func}}

\begin{fulllineitems}
\phantomsection\label{\detokenize{helper_func:helper_func.Get_std}}\pysiglinewithargsret{\sphinxcode{\sphinxupquote{helper\_func.}}\sphinxbfcode{\sphinxupquote{Get\_std}}}{\emph{\DUrole{n}{x}}}{}
\sphinxAtStartPar
get standard deviation ignoring nans

\end{fulllineitems}

\index{Hist\_custom() (in module helper\_func)@\spxentry{Hist\_custom()}\spxextra{in module helper\_func}}

\begin{fulllineitems}
\phantomsection\label{\detokenize{helper_func:helper_func.Hist_custom}}\pysiglinewithargsret{\sphinxcode{\sphinxupquote{helper\_func.}}\sphinxbfcode{\sphinxupquote{Hist\_custom}}}{\emph{\DUrole{n}{S}}}{}
\sphinxAtStartPar
plot histogram with N\_SAMPLE//100 bins ignoring nans

\end{fulllineitems}

\index{Normal\_custom() (in module helper\_func)@\spxentry{Normal\_custom()}\spxextra{in module helper\_func}}

\begin{fulllineitems}
\phantomsection\label{\detokenize{helper_func:helper_func.Normal_custom}}\pysiglinewithargsret{\sphinxcode{\sphinxupquote{helper\_func.}}\sphinxbfcode{\sphinxupquote{Normal\_custom}}}{\emph{\DUrole{n}{m}}, \emph{\DUrole{n}{s}}, \emph{\DUrole{n}{n\_sample}\DUrole{o}{=}\DUrole{default_value}{100000}}, \emph{\DUrole{n}{non\_negative}\DUrole{o}{=}\DUrole{default_value}{False}}, \emph{\DUrole{n}{plot}\DUrole{o}{=}\DUrole{default_value}{False}}}{}
\sphinxAtStartPar
Sampling from a normal distribution
\begin{quote}\begin{description}
\item[{Parameters}] \leavevmode\begin{itemize}
\item {} 
\sphinxAtStartPar
\sphinxstyleliteralstrong{\sphinxupquote{m}} (\sphinxstyleliteralemphasis{\sphinxupquote{int}}\sphinxstyleliteralemphasis{\sphinxupquote{ or }}\sphinxstyleliteralemphasis{\sphinxupquote{float}}) \textendash{} mean

\item {} 
\sphinxAtStartPar
\sphinxstyleliteralstrong{\sphinxupquote{s}} (\sphinxstyleliteralemphasis{\sphinxupquote{int}}\sphinxstyleliteralemphasis{\sphinxupquote{ or }}\sphinxstyleliteralemphasis{\sphinxupquote{float}}) \textendash{} standard deviation

\item {} 
\sphinxAtStartPar
\sphinxstyleliteralstrong{\sphinxupquote{n\_sample}} (\sphinxstyleliteralemphasis{\sphinxupquote{int}}) \textendash{} sample number, default is a Global var N\_SAMPLE

\item {} 
\sphinxAtStartPar
\sphinxstyleliteralstrong{\sphinxupquote{non\_negative}} (\sphinxstyleliteralemphasis{\sphinxupquote{bool}}) \textendash{} if true, return truncated distribution with no negatives, default is False

\item {} 
\sphinxAtStartPar
\sphinxstyleliteralstrong{\sphinxupquote{plot}} (\sphinxstyleliteralemphasis{\sphinxupquote{bool}}) \textendash{} default is False

\end{itemize}

\item[{Returns}] \leavevmode
\sphinxAtStartPar
sample array from the distribution

\item[{Return type}] \leavevmode
\sphinxAtStartPar
numpy array

\end{description}\end{quote}

\end{fulllineitems}

\index{Pf\_RS() (in module helper\_func)@\spxentry{Pf\_RS()}\spxextra{in module helper\_func}}

\begin{fulllineitems}
\phantomsection\label{\detokenize{helper_func:helper_func.Pf_RS}}\pysiglinewithargsret{\sphinxcode{\sphinxupquote{helper\_func.}}\sphinxbfcode{\sphinxupquote{Pf\_RS}}}{\emph{\DUrole{n}{R\_info}}, \emph{\DUrole{n}{S}}, \emph{\DUrole{n}{R\_distrib\_type}\DUrole{o}{=}\DUrole{default_value}{\textquotesingle{}normal\textquotesingle{}}}, \emph{\DUrole{n}{plot}\DUrole{o}{=}\DUrole{default_value}{False}}}{}
\sphinxAtStartPar
Pf\_RS calculates the probability of failure  Pf = P(R\sphinxhyphen{}S\textless{}0), given the R(resistance) and S(load)
with three three methods and use method 3 if it is checked “OK” with the other two
\begin{enumerate}
\sphinxsetlistlabels{\arabic}{enumi}{enumii}{}{.}%
\item {} 
\sphinxAtStartPar
crude monte carlo

\item {} 
\sphinxAtStartPar
numerical integral of g kernel fit

\item {} 
\sphinxAtStartPar
R S integral: \(\int\limits_{-\infty}^{\infty} F_R(x)f_S(x)dx\), reliability index(beta factor) is calculated with simple 1st order g.mean()/g.std()

\end{enumerate}
\begin{quote}\begin{description}
\item[{Parameters}] \leavevmode\begin{itemize}
\item {} 
\sphinxAtStartPar
\sphinxstyleliteralstrong{\sphinxupquote{R\_info}} (\sphinxstyleliteralemphasis{\sphinxupquote{tuple}}\sphinxstyleliteralemphasis{\sphinxupquote{, }}\sphinxstyleliteralemphasis{\sphinxupquote{numpy array}}) \textendash{} 
\sphinxAtStartPar
distribution of Resistance, e.g. cover thickness, critical chloride content, tensile strength
can be array or distribution parameters

\sphinxAtStartPar
R\_distrib\_type=’normal’ \sphinxhyphen{}\textgreater{} tuple(m,s) for normal m: mean s: standard deviation

\sphinxAtStartPar
R\_distrib\_type=’beta’ \sphinxhyphen{}\textgreater{} tuple(m,s,a,b) for (General) beta distribution
m: mean, s: standard deviation a,b : lower, upper bound

\sphinxAtStartPar
R\_distrib\_type=’array’ \sphinxhyphen{}\textgreater{} array: for not\sphinxhyphen{}determined distribution, will be treated numerically(R S integral is not applied )


\item {} 
\sphinxAtStartPar
\sphinxstyleliteralstrong{\sphinxupquote{S}} (\sphinxstyleliteralemphasis{\sphinxupquote{numpy array}}) \textendash{} distribution of load, e.g. carbonation depth, chloride content, tensile stress
the distribution type is calculated S is usually not determined, can vary a lot in different cases, therefore fitted with kernel

\item {} 
\sphinxAtStartPar
\sphinxstyleliteralstrong{\sphinxupquote{R\_distrib\_type}} (\sphinxstyleliteralemphasis{\sphinxupquote{str}}\sphinxstyleliteralemphasis{\sphinxupquote{, }}\sphinxstyleliteralemphasis{\sphinxupquote{optional}}) \textendash{} ‘normal’, ‘beta’, ‘array’, by default ‘normal’

\item {} 
\sphinxAtStartPar
\sphinxstyleliteralstrong{\sphinxupquote{plot}} (\sphinxstyleliteralemphasis{\sphinxupquote{bool}}\sphinxstyleliteralemphasis{\sphinxupquote{, }}\sphinxstyleliteralemphasis{\sphinxupquote{optional}}) \textendash{} plot distribution, by default False

\end{itemize}

\item[{Returns}] \leavevmode
\sphinxAtStartPar
(probability of failure, reliability index)

\item[{Return type}] \leavevmode
\sphinxAtStartPar
tuple

\end{description}\end{quote}

\begin{sphinxadmonition}{note}{Note:}
\sphinxAtStartPar
For R as arrays R S integral is not applied
R S integration method: \(P_f = P(R-S<=0)=\int\limits_{-\infty}^{\infty}f_S(y) \int\limits_{-\infty}^{y}f_R(x)dxdy\)
the dual numerical integration seems too computationally expensive, so consider fit R to analytical distribution in the future versions{[}TODO{]}
\end{sphinxadmonition}

\end{fulllineitems}

\index{RS\_plot() (in module helper\_func)@\spxentry{RS\_plot()}\spxextra{in module helper\_func}}

\begin{fulllineitems}
\phantomsection\label{\detokenize{helper_func:helper_func.RS_plot}}\pysiglinewithargsret{\sphinxcode{\sphinxupquote{helper\_func.}}\sphinxbfcode{\sphinxupquote{RS\_plot}}}{\emph{\DUrole{n}{model}}, \emph{\DUrole{n}{ax}\DUrole{o}{=}\DUrole{default_value}{None}}, \emph{\DUrole{n}{t\_offset}\DUrole{o}{=}\DUrole{default_value}{0}}, \emph{\DUrole{n}{amplify}\DUrole{o}{=}\DUrole{default_value}{1}}}{}
\sphinxAtStartPar
plot R S distribution vertically at a time to an axis
\begin{quote}\begin{description}
\item[{Parameters}] \leavevmode\begin{itemize}
\item {} 
\sphinxAtStartPar
\sphinxstyleliteralstrong{\sphinxupquote{model.R\_distrib}} (\sphinxstyleliteralemphasis{\sphinxupquote{scipy.stats.\_continuous\_distns}}\sphinxstyleliteralemphasis{\sphinxupquote{, }}\sphinxstyleliteralemphasis{\sphinxupquote{normal}}\sphinxstyleliteralemphasis{\sphinxupquote{ or }}\sphinxstyleliteralemphasis{\sphinxupquote{beta}}) \textendash{} calculated in Pf\_RS() through model.postproc()

\item {} 
\sphinxAtStartPar
\sphinxstyleliteralstrong{\sphinxupquote{model.S\_kde\_fit}} (\sphinxstyleliteralemphasis{\sphinxupquote{stats.gaussian\_kde}}) \textendash{} calculated in Pf\_RS() through model.postproc()
distribution of load, e.g. carbonation depth, chloride content, tensile     stress. The distrubtion type is calculated S is usually not determined, can vary a lot in different cases, therefore fitted with kernel

\item {} 
\sphinxAtStartPar
\sphinxstyleliteralstrong{\sphinxupquote{model.S}} (\sphinxstyleliteralemphasis{\sphinxupquote{numpy array}}) \textendash{} load, e.g. carbonation depth, chloride content, tensile stress

\item {} 
\sphinxAtStartPar
\sphinxstyleliteralstrong{\sphinxupquote{ax}} (\sphinxstyleliteralemphasis{\sphinxupquote{axis}}) \textendash{} 

\item {} 
\sphinxAtStartPar
\sphinxstyleliteralstrong{\sphinxupquote{t\_offset}} (\sphinxstyleliteralemphasis{\sphinxupquote{time offset to move the plot along the t\sphinxhyphen{}axis. default is zero}}) \textendash{} 

\item {} 
\sphinxAtStartPar
\sphinxstyleliteralstrong{\sphinxupquote{amplify}} (\sphinxstyleliteralemphasis{\sphinxupquote{scale the height of the pdf plot}}) \textendash{} 

\end{itemize}

\end{description}\end{quote}

\end{fulllineitems}

\index{dropna() (in module helper\_func)@\spxentry{dropna()}\spxextra{in module helper\_func}}

\begin{fulllineitems}
\phantomsection\label{\detokenize{helper_func:helper_func.dropna}}\pysiglinewithargsret{\sphinxcode{\sphinxupquote{helper\_func.}}\sphinxbfcode{\sphinxupquote{dropna}}}{\emph{\DUrole{n}{x}}}{}
\sphinxAtStartPar
removes nans

\end{fulllineitems}

\index{f\_solve\_poly2() (in module helper\_func)@\spxentry{f\_solve\_poly2()}\spxextra{in module helper\_func}}

\begin{fulllineitems}
\phantomsection\label{\detokenize{helper_func:helper_func.f_solve_poly2}}\pysiglinewithargsret{\sphinxcode{\sphinxupquote{helper\_func.}}\sphinxbfcode{\sphinxupquote{f\_solve\_poly2}}}{\emph{\DUrole{n}{a}}, \emph{\DUrole{n}{b}}, \emph{\DUrole{n}{c}}}{}
\sphinxAtStartPar
find the two roots of \(ax^2+bx+c=0\)

\end{fulllineitems}

\index{find\_mean() (in module helper\_func)@\spxentry{find\_mean()}\spxextra{in module helper\_func}}

\begin{fulllineitems}
\phantomsection\label{\detokenize{helper_func:helper_func.find_mean}}\pysiglinewithargsret{\sphinxcode{\sphinxupquote{helper\_func.}}\sphinxbfcode{\sphinxupquote{find\_mean}}}{\emph{\DUrole{n}{val}}, \emph{\DUrole{n}{s}}, \emph{\DUrole{n}{confidence\_one\_tailed}\DUrole{o}{=}\DUrole{default_value}{0.95}}}{}
\sphinxAtStartPar
return the mean value of a unknown normal distribution
based on the given value at a known one\sphinxhyphen{}tailed confidence level(default 95\%)
\begin{quote}\begin{description}
\item[{Parameters}] \leavevmode\begin{itemize}
\item {} 
\sphinxAtStartPar
\sphinxstyleliteralstrong{\sphinxupquote{val}} (\sphinxstyleliteralemphasis{\sphinxupquote{float}}) \textendash{} cut\sphinxhyphen{}off value

\item {} 
\sphinxAtStartPar
\sphinxstyleliteralstrong{\sphinxupquote{s}} (\sphinxstyleliteralemphasis{\sphinxupquote{standard deviation}}) \textendash{} 

\item {} 
\sphinxAtStartPar
\sphinxstyleliteralstrong{\sphinxupquote{confidence\_one\_tailed}} (\sphinxstyleliteralemphasis{\sphinxupquote{confidence level}}) \textendash{} 

\end{itemize}

\item[{Returns}] \leavevmode
\sphinxAtStartPar
mean value of the unknown normal distribution

\item[{Return type}] \leavevmode
\sphinxAtStartPar
float

\end{description}\end{quote}

\end{fulllineitems}

\index{find\_similar\_group() (in module helper\_func)@\spxentry{find\_similar\_group()}\spxextra{in module helper\_func}}

\begin{fulllineitems}
\phantomsection\label{\detokenize{helper_func:helper_func.find_similar_group}}\pysiglinewithargsret{\sphinxcode{\sphinxupquote{helper\_func.}}\sphinxbfcode{\sphinxupquote{find\_similar\_group}}}{\emph{\DUrole{n}{item\_list}}, \emph{\DUrole{n}{similar\_group\_size}\DUrole{o}{=}\DUrole{default_value}{2}}}{}
\sphinxAtStartPar
find\_similar\_group finds most alike values in a list
\begin{quote}\begin{description}
\item[{Parameters}] \leavevmode\begin{itemize}
\item {} 
\sphinxAtStartPar
\sphinxstyleliteralstrong{\sphinxupquote{item\_list}} (\sphinxstyleliteralemphasis{\sphinxupquote{list}}) \textendash{} a list to choose from

\item {} 
\sphinxAtStartPar
\sphinxstyleliteralstrong{\sphinxupquote{similar\_group\_size}} (\sphinxstyleliteralemphasis{\sphinxupquote{int}}\sphinxstyleliteralemphasis{\sphinxupquote{, }}\sphinxstyleliteralemphasis{\sphinxupquote{optional}}) \textendash{} number of the alike values, by default 2

\end{itemize}

\item[{Returns}] \leavevmode
\sphinxAtStartPar
a sublist with alike values

\item[{Return type}] \leavevmode
\sphinxAtStartPar
list

\end{description}\end{quote}

\end{fulllineitems}

\index{interp\_extrap\_f() (in module helper\_func)@\spxentry{interp\_extrap\_f()}\spxextra{in module helper\_func}}

\begin{fulllineitems}
\phantomsection\label{\detokenize{helper_func:helper_func.interp_extrap_f}}\pysiglinewithargsret{\sphinxcode{\sphinxupquote{helper\_func.}}\sphinxbfcode{\sphinxupquote{interp\_extrap\_f}}}{\emph{\DUrole{n}{x}}, \emph{\DUrole{n}{y}}, \emph{\DUrole{n}{x\_find}}, \emph{\DUrole{n}{plot}\DUrole{o}{=}\DUrole{default_value}{False}}}{}
\sphinxAtStartPar
interpolate or extrapolate value from an array with fitted2\sphinxhyphen{}deg or 3\sphinxhyphen{}deg polynomial
\begin{quote}\begin{description}
\item[{Parameters}] \leavevmode\begin{itemize}
\item {} 
\sphinxAtStartPar
\sphinxstyleliteralstrong{\sphinxupquote{x}} (\sphinxstyleliteralemphasis{\sphinxupquote{array\sphinxhyphen{}like}}) \textendash{} variable

\item {} 
\sphinxAtStartPar
\sphinxstyleliteralstrong{\sphinxupquote{y}} (\sphinxstyleliteralemphasis{\sphinxupquote{array\sphinxhyphen{}like}}) \textendash{} function value

\item {} 
\sphinxAtStartPar
\sphinxstyleliteralstrong{\sphinxupquote{x\_find}} (\sphinxstyleliteralemphasis{\sphinxupquote{int}}\sphinxstyleliteralemphasis{\sphinxupquote{ or }}\sphinxstyleliteralemphasis{\sphinxupquote{float}}\sphinxstyleliteralemphasis{\sphinxupquote{ or }}\sphinxstyleliteralemphasis{\sphinxupquote{array\sphinxhyphen{}like}}) \textendash{} look\sphinxhyphen{}up x

\item {} 
\sphinxAtStartPar
\sphinxstyleliteralstrong{\sphinxupquote{plot}} (\sphinxstyleliteralemphasis{\sphinxupquote{bool}}) \textendash{} plot curve fit and data points, default if false

\end{itemize}

\item[{Returns}] \leavevmode
\sphinxAtStartPar
inter/extrapolated value(s), raise warning when extrapolation is used

\item[{Return type}] \leavevmode
\sphinxAtStartPar
int or float or array\sphinxhyphen{}like

\end{description}\end{quote}

\end{fulllineitems}

\index{sample\_integral() (in module helper\_func)@\spxentry{sample\_integral()}\spxextra{in module helper\_func}}

\begin{fulllineitems}
\phantomsection\label{\detokenize{helper_func:helper_func.sample_integral}}\pysiglinewithargsret{\sphinxcode{\sphinxupquote{helper\_func.}}\sphinxbfcode{\sphinxupquote{sample\_integral}}}{\emph{\DUrole{n}{Y}}, \emph{\DUrole{n}{x}}}{}
\sphinxAtStartPar
integrate Y over x, where every Y data point is a bunch of distribution samples,
\begin{quote}\begin{description}
\item[{Parameters}] \leavevmode\begin{itemize}
\item {} 
\sphinxAtStartPar
\sphinxstyleliteralstrong{\sphinxupquote{Y}} (\sphinxstyleliteralemphasis{\sphinxupquote{numpy array}}) \textendash{} 
\sphinxAtStartPar
2D

\sphinxAtStartPar
column: y data point

\sphinxAtStartPar
row: samples for each y data point


\item {} 
\sphinxAtStartPar
\sphinxstyleliteralstrong{\sphinxupquote{x}} (\sphinxstyleliteralemphasis{\sphinxupquote{numpy array}}) \textendash{} 1D

\end{itemize}

\item[{Returns}] \leavevmode
\sphinxAtStartPar
int\_y\_x : integral of y over x for all sampled data

\item[{Return type}] \leavevmode
\sphinxAtStartPar
numpy array

\end{description}\end{quote}
\subsubsection*{Examples}

\sphinxAtStartPar
{[}y0\_sample1, y0\_sample2
\begin{quote}

\sphinxAtStartPar
y1\_sample1, y1\_sample2{]}
\end{quote}

\end{fulllineitems}



\section{test\_helper\_func module}
\label{\detokenize{test_helper_func:module-test_helper_func}}\label{\detokenize{test_helper_func:test-helper-func-module}}\label{\detokenize{test_helper_func::doc}}\index{module@\spxentry{module}!test\_helper\_func@\spxentry{test\_helper\_func}}\index{test\_helper\_func@\spxentry{test\_helper\_func}!module@\spxentry{module}}\index{TestHelperFunc (class in test\_helper\_func)@\spxentry{TestHelperFunc}\spxextra{class in test\_helper\_func}}

\begin{fulllineitems}
\phantomsection\label{\detokenize{test_helper_func:test_helper_func.TestHelperFunc}}\pysiglinewithargsret{\sphinxbfcode{\sphinxupquote{class }}\sphinxcode{\sphinxupquote{test\_helper\_func.}}\sphinxbfcode{\sphinxupquote{TestHelperFunc}}}{\emph{\DUrole{n}{methodName}\DUrole{o}{=}\DUrole{default_value}{\textquotesingle{}runTest\textquotesingle{}}}}{}
\sphinxAtStartPar
Bases: \sphinxcode{\sphinxupquote{unittest.case.TestCase}}
\index{setUp() (test\_helper\_func.TestHelperFunc method)@\spxentry{setUp()}\spxextra{test\_helper\_func.TestHelperFunc method}}

\begin{fulllineitems}
\phantomsection\label{\detokenize{test_helper_func:test_helper_func.TestHelperFunc.setUp}}\pysiglinewithargsret{\sphinxbfcode{\sphinxupquote{setUp}}}{}{}
\sphinxAtStartPar
Hook method for setting up the test fixture before exercising it.

\end{fulllineitems}

\index{tearDown() (test\_helper\_func.TestHelperFunc method)@\spxentry{tearDown()}\spxextra{test\_helper\_func.TestHelperFunc method}}

\begin{fulllineitems}
\phantomsection\label{\detokenize{test_helper_func:test_helper_func.TestHelperFunc.tearDown}}\pysiglinewithargsret{\sphinxbfcode{\sphinxupquote{tearDown}}}{}{}
\sphinxAtStartPar
Hook method for deconstructing the test fixture after testing it.

\end{fulllineitems}

\index{test\_Beta\_custom() (test\_helper\_func.TestHelperFunc method)@\spxentry{test\_Beta\_custom()}\spxextra{test\_helper\_func.TestHelperFunc method}}

\begin{fulllineitems}
\phantomsection\label{\detokenize{test_helper_func:test_helper_func.TestHelperFunc.test_Beta_custom}}\pysiglinewithargsret{\sphinxbfcode{\sphinxupquote{test\_Beta\_custom}}}{}{}
\end{fulllineitems}

\index{test\_Fit\_distrib() (test\_helper\_func.TestHelperFunc method)@\spxentry{test\_Fit\_distrib()}\spxextra{test\_helper\_func.TestHelperFunc method}}

\begin{fulllineitems}
\phantomsection\label{\detokenize{test_helper_func:test_helper_func.TestHelperFunc.test_Fit_distrib}}\pysiglinewithargsret{\sphinxbfcode{\sphinxupquote{test\_Fit\_distrib}}}{}{}
\end{fulllineitems}

\index{test\_Get\_mean() (test\_helper\_func.TestHelperFunc method)@\spxentry{test\_Get\_mean()}\spxextra{test\_helper\_func.TestHelperFunc method}}

\begin{fulllineitems}
\phantomsection\label{\detokenize{test_helper_func:test_helper_func.TestHelperFunc.test_Get_mean}}\pysiglinewithargsret{\sphinxbfcode{\sphinxupquote{test\_Get\_mean}}}{}{}
\end{fulllineitems}

\index{test\_Get\_std() (test\_helper\_func.TestHelperFunc method)@\spxentry{test\_Get\_std()}\spxextra{test\_helper\_func.TestHelperFunc method}}

\begin{fulllineitems}
\phantomsection\label{\detokenize{test_helper_func:test_helper_func.TestHelperFunc.test_Get_std}}\pysiglinewithargsret{\sphinxbfcode{\sphinxupquote{test\_Get\_std}}}{}{}
\end{fulllineitems}

\index{test\_Normal\_custom() (test\_helper\_func.TestHelperFunc method)@\spxentry{test\_Normal\_custom()}\spxextra{test\_helper\_func.TestHelperFunc method}}

\begin{fulllineitems}
\phantomsection\label{\detokenize{test_helper_func:test_helper_func.TestHelperFunc.test_Normal_custom}}\pysiglinewithargsret{\sphinxbfcode{\sphinxupquote{test\_Normal\_custom}}}{}{}
\end{fulllineitems}

\index{test\_Pf\_RS() (test\_helper\_func.TestHelperFunc method)@\spxentry{test\_Pf\_RS()}\spxextra{test\_helper\_func.TestHelperFunc method}}

\begin{fulllineitems}
\phantomsection\label{\detokenize{test_helper_func:test_helper_func.TestHelperFunc.test_Pf_RS}}\pysiglinewithargsret{\sphinxbfcode{\sphinxupquote{test\_Pf\_RS}}}{}{}
\end{fulllineitems}

\index{test\_RS\_plot() (test\_helper\_func.TestHelperFunc method)@\spxentry{test\_RS\_plot()}\spxextra{test\_helper\_func.TestHelperFunc method}}

\begin{fulllineitems}
\phantomsection\label{\detokenize{test_helper_func:test_helper_func.TestHelperFunc.test_RS_plot}}\pysiglinewithargsret{\sphinxbfcode{\sphinxupquote{test\_RS\_plot}}}{}{}
\end{fulllineitems}

\index{test\_dropna() (test\_helper\_func.TestHelperFunc method)@\spxentry{test\_dropna()}\spxextra{test\_helper\_func.TestHelperFunc method}}

\begin{fulllineitems}
\phantomsection\label{\detokenize{test_helper_func:test_helper_func.TestHelperFunc.test_dropna}}\pysiglinewithargsret{\sphinxbfcode{\sphinxupquote{test\_dropna}}}{}{}
\end{fulllineitems}

\index{test\_f\_solve\_poly2() (test\_helper\_func.TestHelperFunc method)@\spxentry{test\_f\_solve\_poly2()}\spxextra{test\_helper\_func.TestHelperFunc method}}

\begin{fulllineitems}
\phantomsection\label{\detokenize{test_helper_func:test_helper_func.TestHelperFunc.test_f_solve_poly2}}\pysiglinewithargsret{\sphinxbfcode{\sphinxupquote{test\_f\_solve\_poly2}}}{}{}
\end{fulllineitems}

\index{test\_find\_mean() (test\_helper\_func.TestHelperFunc method)@\spxentry{test\_find\_mean()}\spxextra{test\_helper\_func.TestHelperFunc method}}

\begin{fulllineitems}
\phantomsection\label{\detokenize{test_helper_func:test_helper_func.TestHelperFunc.test_find_mean}}\pysiglinewithargsret{\sphinxbfcode{\sphinxupquote{test\_find\_mean}}}{}{}
\end{fulllineitems}

\index{test\_find\_similar\_group() (test\_helper\_func.TestHelperFunc method)@\spxentry{test\_find\_similar\_group()}\spxextra{test\_helper\_func.TestHelperFunc method}}

\begin{fulllineitems}
\phantomsection\label{\detokenize{test_helper_func:test_helper_func.TestHelperFunc.test_find_similar_group}}\pysiglinewithargsret{\sphinxbfcode{\sphinxupquote{test\_find\_similar\_group}}}{}{}
\end{fulllineitems}

\index{test\_interp\_extrap\_f() (test\_helper\_func.TestHelperFunc method)@\spxentry{test\_interp\_extrap\_f()}\spxextra{test\_helper\_func.TestHelperFunc method}}

\begin{fulllineitems}
\phantomsection\label{\detokenize{test_helper_func:test_helper_func.TestHelperFunc.test_interp_extrap_f}}\pysiglinewithargsret{\sphinxbfcode{\sphinxupquote{test\_interp\_extrap\_f}}}{}{}
\end{fulllineitems}

\index{test\_sample\_integral() (test\_helper\_func.TestHelperFunc method)@\spxentry{test\_sample\_integral()}\spxextra{test\_helper\_func.TestHelperFunc method}}

\begin{fulllineitems}
\phantomsection\label{\detokenize{test_helper_func:test_helper_func.TestHelperFunc.test_sample_integral}}\pysiglinewithargsret{\sphinxbfcode{\sphinxupquote{test\_sample\_integral}}}{}{}
\end{fulllineitems}


\end{fulllineitems}



\chapter{Examples}
\label{\detokenize{examples:examples}}\label{\detokenize{examples::doc}}

\section{membrane module example}
\label{\detokenize{membrane_example:membrane-module-example}}\label{\detokenize{membrane_example::doc}}\begin{itemize}
\item {} 
\sphinxAtStartPar
Raw parameter data

\item {} 
\sphinxAtStartPar
initialize model

\item {} 
\sphinxAtStartPar
run model

\item {} 
\sphinxAtStartPar
calibrate model

\end{itemize}

{
\sphinxsetup{VerbatimColor={named}{nbsphinx-code-bg}}
\sphinxsetup{VerbatimBorderColor={named}{nbsphinx-code-border}}
\begin{sphinxVerbatim}[commandchars=\\\{\}]
\llap{\color{nbsphinxin}[18]:\,\hspace{\fboxrule}\hspace{\fboxsep}}\PYG{o}{\PYGZpc{}}\PYG{k}{matplotlib} inline
\PYG{k+kn}{import} \PYG{n+nn}{numpy} \PYG{k}{as} \PYG{n+nn}{np}
\PYG{k+kn}{from} \PYG{n+nn}{membrane} \PYG{k+kn}{import} \PYG{n}{Membrane\PYGZus{}Model}
\end{sphinxVerbatim}
}

{
\sphinxsetup{VerbatimColor={named}{nbsphinx-code-bg}}
\sphinxsetup{VerbatimBorderColor={named}{nbsphinx-code-border}}
\begin{sphinxVerbatim}[commandchars=\\\{\}]
\llap{\color{nbsphinxin}[7]:\,\hspace{\fboxrule}\hspace{\fboxsep}}\PYG{c+c1}{\PYGZsh{} Case study Raw parameter data}
\PYG{k}{class} \PYG{n+nc}{Param}\PYG{p}{:} \PYG{k}{pass}

\PYG{n}{raw\PYGZus{}pars} \PYG{o}{=} \PYG{n}{Param}\PYG{p}{(}\PYG{p}{)}

\PYG{c+c1}{\PYGZsh{} product information}
\PYG{n}{raw\PYGZus{}pars}\PYG{o}{.}\PYG{n}{life\PYGZus{}product\PYGZus{}label\PYGZus{}life} \PYG{o}{=} \PYG{l+m+mi}{10}  \PYG{c+c1}{\PYGZsh{} year, defined as 95\PYGZpc{} confident non\PYGZhy{}failure}
\PYG{n}{raw\PYGZus{}pars}\PYG{o}{.}\PYG{n}{life\PYGZus{}std} \PYG{o}{=} \PYG{l+m+mf}{0.2} \PYG{o}{*} \PYG{n}{raw\PYGZus{}pars}\PYG{o}{.}\PYG{n}{life\PYGZus{}product\PYGZus{}label\PYGZus{}life} \PYG{c+c1}{\PYGZsh{} assume if not known, calibrate later for real service conditions}
\PYG{n}{raw\PYGZus{}pars}\PYG{o}{.}\PYG{n}{life\PYGZus{}confidence} \PYG{o}{=} \PYG{l+m+mf}{0.95}

\PYG{c+c1}{\PYGZsh{} calibration data (if available)}
\PYG{c+c1}{\PYGZsh{} field survey result}
\PYG{n}{raw\PYGZus{}pars}\PYG{o}{.}\PYG{n}{membrane\PYGZus{}failure\PYGZus{}ratio\PYGZus{}field} \PYG{o}{=} \PYG{l+m+mf}{0.01}
\PYG{n}{raw\PYGZus{}pars}\PYG{o}{.}\PYG{n}{membrane\PYGZus{}age\PYGZus{}field} \PYG{o}{=} \PYG{l+m+mi}{5}  \PYG{c+c1}{\PYGZsh{} [year]}


\end{sphinxVerbatim}
}

{
\sphinxsetup{VerbatimColor={named}{nbsphinx-code-bg}}
\sphinxsetup{VerbatimBorderColor={named}{nbsphinx-code-border}}
\begin{sphinxVerbatim}[commandchars=\\\{\}]
\llap{\color{nbsphinxin}[8]:\,\hspace{\fboxrule}\hspace{\fboxsep}}\PYG{c+c1}{\PYGZsh{} initialize model}
\PYG{n}{mem\PYGZus{}model} \PYG{o}{=} \PYG{n}{Membrane\PYGZus{}Model}\PYG{p}{(}\PYG{n}{raw\PYGZus{}pars}\PYG{p}{)}

\PYG{c+c1}{\PYGZsh{} run and postproc (uncalibrated)}
\PYG{n}{mem\PYGZus{}model}\PYG{o}{.}\PYG{n}{run}\PYG{p}{(}\PYG{l+m+mi}{10}\PYG{p}{)}  \PYG{c+c1}{\PYGZsh{} 10 years}
\PYG{n}{mem\PYGZus{}model}\PYG{o}{.}\PYG{n}{postproc}\PYG{p}{(}\PYG{n}{plot}\PYG{o}{=}\PYG{k+kc}{True}\PYG{p}{)}
\end{sphinxVerbatim}
}

{

\kern-\sphinxverbatimsmallskipamount\kern-\baselineskip
\kern+\FrameHeightAdjust\kern-\fboxrule
\vspace{\nbsphinxcodecellspacing}

\sphinxsetup{VerbatimColor={named}{white}}
\sphinxsetup{VerbatimBorderColor={named}{nbsphinx-code-border}}
\begin{sphinxVerbatim}[commandchars=\\\{\}]
Pf(g = R-S < 0) from various methods
    sample count: 0.05058
    g integral: 0.051280490624611694
    R S integral: 0.05000000000001138
    beta\_factor: 1.6378434656157241
\end{sphinxVerbatim}
}

\hrule height -\fboxrule\relax
\vspace{\nbsphinxcodecellspacing}

\makeatletter\setbox\nbsphinxpromptbox\box\voidb@x\makeatother

\begin{nbsphinxfancyoutput}

\noindent\sphinxincludegraphics[width=715\sphinxpxdimen,height=208\sphinxpxdimen]{{membrane_example_3_1}.png}

\end{nbsphinxfancyoutput}

{
\sphinxsetup{VerbatimColor={named}{nbsphinx-code-bg}}
\sphinxsetup{VerbatimBorderColor={named}{nbsphinx-code-border}}
\begin{sphinxVerbatim}[commandchars=\\\{\}]
\llap{\color{nbsphinxin}[12]:\,\hspace{\fboxrule}\hspace{\fboxsep}}\PYG{c+c1}{\PYGZsh{} calibration to field data}
\PYG{n}{mem\PYGZus{}model\PYGZus{}cal} \PYG{o}{=} \PYG{n}{mem\PYGZus{}model}\PYG{o}{.}\PYG{n}{calibrate}\PYG{p}{(}\PYG{n}{raw\PYGZus{}pars}\PYG{o}{.}\PYG{n}{membrane\PYGZus{}age\PYGZus{}field}\PYG{p}{,} \PYG{n}{raw\PYGZus{}pars}\PYG{o}{.}\PYG{n}{membrane\PYGZus{}failure\PYGZus{}ratio\PYGZus{}field}\PYG{p}{)}


\end{sphinxVerbatim}
}

{

\kern-\sphinxverbatimsmallskipamount\kern-\baselineskip
\kern+\FrameHeightAdjust\kern-\fboxrule
\vspace{\nbsphinxcodecellspacing}

\sphinxsetup{VerbatimColor={named}{white}}
\sphinxsetup{VerbatimBorderColor={named}{nbsphinx-code-border}}
\begin{sphinxVerbatim}[commandchars=\\\{\}]
probability of failure:
model: 0.010000011916189768
field: 0.01
\end{sphinxVerbatim}
}

{
\sphinxsetup{VerbatimColor={named}{nbsphinx-code-bg}}
\sphinxsetup{VerbatimBorderColor={named}{nbsphinx-code-border}}
\begin{sphinxVerbatim}[commandchars=\\\{\}]
\llap{\color{nbsphinxin}[13]:\,\hspace{\fboxrule}\hspace{\fboxsep}}\PYG{c+c1}{\PYGZsh{} run and postproc (calibrated)}
\PYG{n}{mem\PYGZus{}model\PYGZus{}cal}\PYG{o}{.}\PYG{n}{run}\PYG{p}{(}\PYG{l+m+mi}{10}\PYG{p}{)}  \PYG{c+c1}{\PYGZsh{} 10 years}
\PYG{n}{mem\PYGZus{}model\PYGZus{}cal}\PYG{o}{.}\PYG{n}{postproc}\PYG{p}{(}\PYG{n}{plot}\PYG{o}{=}\PYG{k+kc}{True}\PYG{p}{)}
\end{sphinxVerbatim}
}

{

\kern-\sphinxverbatimsmallskipamount\kern-\baselineskip
\kern+\FrameHeightAdjust\kern-\fboxrule
\vspace{\nbsphinxcodecellspacing}

\sphinxsetup{VerbatimColor={named}{white}}
\sphinxsetup{VerbatimBorderColor={named}{nbsphinx-code-border}}
\begin{sphinxVerbatim}[commandchars=\\\{\}]
Pf(g = R-S < 0) from various methods
    sample count: 0.17761
    g integral: 0.1791927385519138
    R S integral: 0.17795324587799488
    beta\_factor: 0.9208084394524149
\end{sphinxVerbatim}
}

\hrule height -\fboxrule\relax
\vspace{\nbsphinxcodecellspacing}

\makeatletter\setbox\nbsphinxpromptbox\box\voidb@x\makeatother

\begin{nbsphinxfancyoutput}

\noindent\sphinxincludegraphics[width=704\sphinxpxdimen,height=208\sphinxpxdimen]{{membrane_example_5_1}.png}

\end{nbsphinxfancyoutput}

{
\sphinxsetup{VerbatimColor={named}{nbsphinx-code-bg}}
\sphinxsetup{VerbatimBorderColor={named}{nbsphinx-code-border}}
\begin{sphinxVerbatim}[commandchars=\\\{\}]
\llap{\color{nbsphinxin}[22]:\,\hspace{\fboxrule}\hspace{\fboxsep}}\PYG{c+c1}{\PYGZsh{} model with a list of time steps}
\PYG{n}{t\PYGZus{}lis} \PYG{o}{=} \PYG{n}{np}\PYG{o}{.}\PYG{n}{arange}\PYG{p}{(}\PYG{l+m+mi}{0}\PYG{p}{,}\PYG{l+m+mi}{21}\PYG{p}{,}\PYG{l+m+mi}{1}\PYG{p}{)}
\PYG{n}{pf\PYGZus{}lis}\PYG{p}{,} \PYG{n}{beta\PYGZus{}lis} \PYG{o}{=} \PYG{n}{mem\PYGZus{}model\PYGZus{}cal}\PYG{o}{.}\PYG{n}{membrane\PYGZus{}failure\PYGZus{}with\PYGZus{}year}\PYG{p}{(}\PYG{n}{year\PYGZus{}lis}\PYG{o}{=}\PYG{n}{t\PYGZus{}lis}\PYG{p}{,} \PYG{n}{plot}\PYG{o}{=}\PYG{k+kc}{True}\PYG{p}{,} \PYG{n}{amplify}\PYG{o}{=}\PYG{l+m+mi}{30}\PYG{p}{)}
\end{sphinxVerbatim}
}

\hrule height -\fboxrule\relax
\vspace{\nbsphinxcodecellspacing}

\makeatletter\setbox\nbsphinxpromptbox\box\voidb@x\makeatother

\begin{nbsphinxfancyoutput}

\noindent\sphinxincludegraphics[width=568\sphinxpxdimen,height=564\sphinxpxdimen]{{membrane_example_6_0}.png}

\end{nbsphinxfancyoutput}


\section{carbonation module example}
\label{\detokenize{carbonation_example:carbonation-module-example}}\label{\detokenize{carbonation_example::doc}}\begin{itemize}
\item {} 
\sphinxAtStartPar
Raw parameter data

\item {} 
\sphinxAtStartPar
initialize model

\item {} 
\sphinxAtStartPar
run model

\item {} 
\sphinxAtStartPar
calibrate model

\end{itemize}

{
\sphinxsetup{VerbatimColor={named}{nbsphinx-code-bg}}
\sphinxsetup{VerbatimBorderColor={named}{nbsphinx-code-border}}
\begin{sphinxVerbatim}[commandchars=\\\{\}]
\llap{\color{nbsphinxin}[10]:\,\hspace{\fboxrule}\hspace{\fboxsep}}\PYG{o}{\PYGZpc{}}\PYG{k}{matplotlib} inline
\PYG{k+kn}{import} \PYG{n+nn}{helper\PYGZus{}func} \PYG{k}{as} \PYG{n+nn}{hf}
\PYG{k+kn}{import} \PYG{n+nn}{numpy} \PYG{k}{as} \PYG{n+nn}{np}
\PYG{k+kn}{from} \PYG{n+nn}{carbonation} \PYG{k+kn}{import} \PYG{n}{Carbonation\PYGZus{}Model}\PYG{p}{,} \PYG{n}{load\PYGZus{}df\PYGZus{}R\PYGZus{}ACC}
\end{sphinxVerbatim}
}

{
\sphinxsetup{VerbatimColor={named}{nbsphinx-code-bg}}
\sphinxsetup{VerbatimBorderColor={named}{nbsphinx-code-border}}
\begin{sphinxVerbatim}[commandchars=\\\{\}]
\llap{\color{nbsphinxin}[4]:\,\hspace{\fboxrule}\hspace{\fboxsep}}\PYG{c+c1}{\PYGZsh{} Case study}

\PYG{c+c1}{\PYGZsh{} global \PYGZhy{} Raw parameters}
\PYG{k}{class} \PYG{n+nc}{Param}\PYG{p}{:} \PYG{k}{pass}

\PYG{n}{pars} \PYG{o}{=} \PYG{n}{Param}\PYG{p}{(}\PYG{p}{)}

\PYG{n}{pars}\PYG{o}{.}\PYG{n}{cover\PYGZus{}mean} \PYG{o}{=} \PYG{l+m+mi}{50}  \PYG{c+c1}{\PYGZsh{} mm}
\PYG{n}{pars}\PYG{o}{.}\PYG{n}{cover\PYGZus{}std} \PYG{o}{=} \PYG{l+m+mi}{5}
\PYG{n}{pars}\PYG{o}{.}\PYG{n}{RH\PYGZus{}real} \PYG{o}{=} \PYG{l+m+mi}{60}
\PYG{n}{pars}\PYG{o}{.}\PYG{n}{t\PYGZus{}c} \PYG{o}{=} \PYG{l+m+mi}{28}
\PYG{n}{pars}\PYG{o}{.}\PYG{n}{x\PYGZus{}c} \PYG{o}{=} \PYG{l+m+mf}{0.008}  \PYG{c+c1}{\PYGZsh{} m}
\PYG{n}{pars}\PYG{o}{.}\PYG{n}{ToW} \PYG{o}{=} \PYG{l+m+mi}{2} \PYG{o}{/} \PYG{l+m+mf}{52.}
\PYG{n}{pars}\PYG{o}{.}\PYG{n}{p\PYGZus{}SR} \PYG{o}{=} \PYG{l+m+mf}{0.0}
\PYG{n}{pars}\PYG{o}{.}\PYG{n}{C\PYGZus{}S\PYGZus{}emi} \PYG{o}{=} \PYG{l+m+mf}{0.}

\PYG{n}{pars}\PYG{o}{.}\PYG{n}{option} \PYG{o}{=} \PYG{n}{Param}\PYG{p}{(}\PYG{p}{)}
\PYG{n}{pars}\PYG{o}{.}\PYG{n}{option}\PYG{o}{.}\PYG{n}{choose} \PYG{o}{=} \PYG{k+kc}{False}
\PYG{n}{pars}\PYG{o}{.}\PYG{n}{option}\PYG{o}{.}\PYG{n}{cement\PYGZus{}type} \PYG{o}{=} \PYG{l+s+s1}{\PYGZsq{}}\PYG{l+s+s1}{CEM\PYGZus{}I\PYGZus{}42.5\PYGZus{}R+SF}\PYG{l+s+s1}{\PYGZsq{}}
\PYG{n}{pars}\PYG{o}{.}\PYG{n}{option}\PYG{o}{.}\PYG{n}{wc\PYGZus{}eqv} \PYG{o}{=} \PYG{l+m+mf}{0.6}
\PYG{n}{pars}\PYG{o}{.}\PYG{n}{option}\PYG{o}{.}\PYG{n}{df\PYGZus{}R\PYGZus{}ACC} \PYG{o}{=} \PYG{n}{load\PYGZus{}df\PYGZus{}R\PYGZus{}ACC}\PYG{p}{(}\PYG{p}{)}
\PYG{n}{pars}\PYG{o}{.}\PYG{n}{option}\PYG{o}{.}\PYG{n}{plot} \PYG{o}{=} \PYG{k+kc}{True}

\PYG{c+c1}{\PYGZsh{} initialize model}
\PYG{n}{carb\PYGZus{}model} \PYG{o}{=} \PYG{n}{Carbonation\PYGZus{}Model}\PYG{p}{(}\PYG{n}{pars}\PYG{p}{)}

\PYG{c+c1}{\PYGZsh{} run and postproc model}
\PYG{n}{carb\PYGZus{}model}\PYG{o}{.}\PYG{n}{run}\PYG{p}{(}\PYG{l+m+mi}{50}\PYG{p}{)}
\PYG{n}{carb\PYGZus{}model}\PYG{o}{.}\PYG{n}{postproc}\PYG{p}{(}\PYG{n}{plot}\PYG{o}{=}\PYG{k+kc}{True}\PYG{p}{)}
\end{sphinxVerbatim}
}

{

\kern-\sphinxverbatimsmallskipamount\kern-\baselineskip
\kern+\FrameHeightAdjust\kern-\fboxrule
\vspace{\nbsphinxcodecellspacing}

\sphinxsetup{VerbatimColor={named}{nbsphinx-stderr}}
\sphinxsetup{VerbatimBorderColor={named}{nbsphinx-code-border}}
\begin{sphinxVerbatim}[commandchars=\\\{\}]
/Users/gangli/Local Documents/Mitacs project local/Tinkrete/modules/carbonation.py:281: RuntimeWarning: divide by zero encountered in power
  W = (t\_0 / t) ** ((p\_SR * ToW) ** b\_w / 2.0)
/Users/gangli/Local Documents/Mitacs project local/Tinkrete/modules/carbonation.py:76: RuntimeWarning: invalid value encountered in sqrt
  ) ** 0.5 * pars.W\_t
Pf(g = R-S < 0) from various methods
    sample count: 0.0002001220744654239
    g integral: 0.00021125729406379793
    R S integral: 0.0002582774031271491
    beta\_factor: 3.4595900423913237
\end{sphinxVerbatim}
}

\hrule height -\fboxrule\relax
\vspace{\nbsphinxcodecellspacing}

\makeatletter\setbox\nbsphinxpromptbox\box\voidb@x\makeatother

\begin{nbsphinxfancyoutput}

\noindent\sphinxincludegraphics[width=704\sphinxpxdimen,height=208\sphinxpxdimen]{{carbonation_example_2_1}.png}

\end{nbsphinxfancyoutput}

{
\sphinxsetup{VerbatimColor={named}{nbsphinx-code-bg}}
\sphinxsetup{VerbatimBorderColor={named}{nbsphinx-code-border}}
\begin{sphinxVerbatim}[commandchars=\\\{\}]
\llap{\color{nbsphinxin}[8]:\,\hspace{\fboxrule}\hspace{\fboxsep}}\PYG{c+c1}{\PYGZsh{} calibration to field data}
\PYG{c+c1}{\PYGZsh{} field data: field carbonation after 20 years, mean=30, std=5}
\PYG{n}{carb\PYGZus{}depth\PYGZus{}field} \PYG{o}{=} \PYG{n}{hf}\PYG{o}{.}\PYG{n}{Normal\PYGZus{}custom}\PYG{p}{(}\PYG{l+m+mi}{30}\PYG{p}{,} \PYG{l+m+mi}{5}\PYG{p}{,} \PYG{n}{n\PYGZus{}sample}\PYG{o}{=}\PYG{l+m+mi}{12}\PYG{p}{)}  \PYG{c+c1}{\PYGZsh{} mm}

\PYG{n}{carb\PYGZus{}model\PYGZus{}cal} \PYG{o}{=} \PYG{n}{carb\PYGZus{}model}\PYG{o}{.}\PYG{n}{calibrate}\PYG{p}{(}\PYG{l+m+mi}{20}\PYG{p}{,} \PYG{n}{carb\PYGZus{}depth\PYGZus{}field}\PYG{p}{,} \PYG{n}{print\PYGZus{}out}\PYG{o}{=}\PYG{k+kc}{True}\PYG{p}{)}

\end{sphinxVerbatim}
}

{

\kern-\sphinxverbatimsmallskipamount\kern-\baselineskip
\kern+\FrameHeightAdjust\kern-\fboxrule
\vspace{\nbsphinxcodecellspacing}

\sphinxsetup{VerbatimColor={named}{nbsphinx-stderr}}
\sphinxsetup{VerbatimBorderColor={named}{nbsphinx-code-border}}
\begin{sphinxVerbatim}[commandchars=\\\{\}]
/Users/gangli/Local Documents/Mitacs project local/Tinkrete/modules/carbonation.py:281: RuntimeWarning: divide by zero encountered in power
  W = (t\_0 / t) ** ((p\_SR * ToW) ** b\_w / 2.0)
/Users/gangli/Local Documents/Mitacs project local/Tinkrete/modules/carbonation.py:76: RuntimeWarning: invalid value encountered in sqrt
  ) ** 0.5 * pars.W\_t
carb\_depth:
model:
mean:29.108260004178188
std:6.106447400167045
field:
mean:29.079887643042
std:4.42009388341032
\end{sphinxVerbatim}
}

{
\sphinxsetup{VerbatimColor={named}{nbsphinx-code-bg}}
\sphinxsetup{VerbatimBorderColor={named}{nbsphinx-code-border}}
\begin{sphinxVerbatim}[commandchars=\\\{\}]
\llap{\color{nbsphinxin}[15]:\,\hspace{\fboxrule}\hspace{\fboxsep}}\PYG{c+c1}{\PYGZsh{} carbonation for a list of time steps}

\PYG{n}{year\PYGZus{}lis} \PYG{o}{=} \PYG{n}{np}\PYG{o}{.}\PYG{n}{arange}\PYG{p}{(}\PYG{l+m+mi}{10}\PYG{p}{,}\PYG{l+m+mi}{150}\PYG{p}{,}\PYG{l+m+mi}{20}\PYG{p}{)}

\PYG{n}{pf\PYGZus{}lis}\PYG{p}{,} \PYG{n}{beta\PYGZus{}lis} \PYG{o}{=} \PYG{n}{carb\PYGZus{}model\PYGZus{}cal}\PYG{o}{.}\PYG{n}{carb\PYGZus{}with\PYGZus{}year}\PYG{p}{(}\PYG{n}{year\PYGZus{}lis}\PYG{o}{=}\PYG{n}{year\PYGZus{}lis}\PYG{p}{,} \PYG{n}{plot}\PYG{o}{=}\PYG{k+kc}{True}\PYG{p}{,} \PYG{n}{amplify}\PYG{o}{=}\PYG{l+m+mi}{200}\PYG{p}{)}

\end{sphinxVerbatim}
}

{

\kern-\sphinxverbatimsmallskipamount\kern-\baselineskip
\kern+\FrameHeightAdjust\kern-\fboxrule
\vspace{\nbsphinxcodecellspacing}

\sphinxsetup{VerbatimColor={named}{nbsphinx-stderr}}
\sphinxsetup{VerbatimBorderColor={named}{nbsphinx-code-border}}
\begin{sphinxVerbatim}[commandchars=\\\{\}]
/Users/gangli/Local Documents/Mitacs project local/Tinkrete/modules/carbonation.py:281: RuntimeWarning: divide by zero encountered in power
  W = (t\_0 / t) ** ((p\_SR * ToW) ** b\_w / 2.0)
/Users/gangli/Local Documents/Mitacs project local/Tinkrete/modules/carbonation.py:76: RuntimeWarning: invalid value encountered in sqrt
  ) ** 0.5 * pars.W\_t
warning: very small Pf
\end{sphinxVerbatim}
}

\hrule height -\fboxrule\relax
\vspace{\nbsphinxcodecellspacing}

\makeatletter\setbox\nbsphinxpromptbox\box\voidb@x\makeatother

\begin{nbsphinxfancyoutput}

\noindent\sphinxincludegraphics[width=568\sphinxpxdimen,height=564\sphinxpxdimen]{{carbonation_example_4_1}.png}

\end{nbsphinxfancyoutput}

{
\sphinxsetup{VerbatimColor={named}{nbsphinx-code-bg}}
\sphinxsetup{VerbatimBorderColor={named}{nbsphinx-code-border}}
\begin{sphinxVerbatim}[commandchars=\\\{\}]
\llap{\color{nbsphinxin}[179]:\,\hspace{\fboxrule}\hspace{\fboxsep}}\PYG{c+c1}{\PYGZsh{} fig.savefig(\PYGZsq{}RS\PYGZus{}time\PYGZus{}carbonation.pdf\PYGZsq{},dpi=1200)}

\end{sphinxVerbatim}
}

{
\sphinxsetup{VerbatimColor={named}{nbsphinx-code-bg}}
\sphinxsetup{VerbatimBorderColor={named}{nbsphinx-code-border}}
\begin{sphinxVerbatim}[commandchars=\\\{\}]
\llap{\color{nbsphinxin}[ ]:\,\hspace{\fboxrule}\hspace{\fboxsep}}
\end{sphinxVerbatim}
}


\section{chloride module example}
\label{\detokenize{chloride_example:chloride-module-example}}\label{\detokenize{chloride_example::doc}}
{
\sphinxsetup{VerbatimColor={named}{nbsphinx-code-bg}}
\sphinxsetup{VerbatimBorderColor={named}{nbsphinx-code-border}}
\begin{sphinxVerbatim}[commandchars=\\\{\}]
\llap{\color{nbsphinxin}[8]:\,\hspace{\fboxrule}\hspace{\fboxsep}}\PYG{o}{\PYGZpc{}}\PYG{k}{matplotlib} inline
\PYG{k+kn}{from} \PYG{n+nn}{chloride} \PYG{k+kn}{import} \PYG{n}{Chloride\PYGZus{}Model}\PYG{p}{,} \PYG{n}{load\PYGZus{}df\PYGZus{}D\PYGZus{}RCM}\PYG{p}{,} \PYG{n}{C\PYGZus{}crit\PYGZus{}param}\PYG{p}{,} \PYG{n}{C\PYGZus{}eqv\PYGZus{}to\PYGZus{}C\PYGZus{}S\PYGZus{}0}
\PYG{k+kn}{import} \PYG{n+nn}{pandas} \PYG{k}{as} \PYG{n+nn}{pd}
\end{sphinxVerbatim}
}

{
\sphinxsetup{VerbatimColor={named}{nbsphinx-code-bg}}
\sphinxsetup{VerbatimBorderColor={named}{nbsphinx-code-border}}
\begin{sphinxVerbatim}[commandchars=\\\{\}]
\llap{\color{nbsphinxin}[4]:\,\hspace{\fboxrule}\hspace{\fboxsep}}\PYG{c+c1}{\PYGZsh{} raw data}
\PYG{k}{class} \PYG{n+nc}{Param}\PYG{p}{:} \PYG{k}{pass}

\PYG{n}{pars\PYGZus{}raw} \PYG{o}{=} \PYG{n}{Param}\PYG{p}{(}\PYG{p}{)}

\PYG{n}{pars\PYGZus{}raw}\PYG{o}{.}\PYG{n}{marine} \PYG{o}{=} \PYG{k+kc}{False}

\PYG{c+c1}{\PYGZsh{} 1)marine or coastal}
\PYG{n}{pars\PYGZus{}raw}\PYG{o}{.}\PYG{n}{C\PYGZus{}0\PYGZus{}M} \PYG{o}{=} \PYG{l+m+mf}{18.980} \PYG{c+c1}{\PYGZsh{} natural chloirde content of sea water [g/l]}

\PYG{c+c1}{\PYGZsh{} 2) de\PYGZhy{}icing salt (hard to quantify)}
\PYG{n}{pars\PYGZus{}raw}\PYG{o}{.}\PYG{n}{C\PYGZus{}0\PYGZus{}R} \PYG{o}{=} \PYG{l+m+mi}{0}  \PYG{c+c1}{\PYGZsh{} average chloride content of the chloride contaminated water [g/l]}
\PYG{n}{pars\PYGZus{}raw}\PYG{o}{.}\PYG{n}{n} \PYG{o}{=} \PYG{l+m+mi}{0}      \PYG{c+c1}{\PYGZsh{} average number of salting events per year [\PYGZhy{}]}
\PYG{n}{pars\PYGZus{}raw}\PYG{o}{.}\PYG{n}{C\PYGZus{}R\PYGZus{}i} \PYG{o}{=} \PYG{l+m+mi}{0}  \PYG{c+c1}{\PYGZsh{} average amount of chloride spread within one spreading event [g/m2]}
\PYG{n}{pars\PYGZus{}raw}\PYG{o}{.}\PYG{n}{h\PYGZus{}S\PYGZus{}i} \PYG{o}{=} \PYG{l+m+mi}{1}  \PYG{c+c1}{\PYGZsh{} amount of water from rain and melted snow per spreading period [l/m2]}

\PYG{n}{pars\PYGZus{}raw}\PYG{o}{.}\PYG{n}{C\PYGZus{}eqv\PYGZus{}to\PYGZus{}C\PYGZus{}S\PYGZus{}0} \PYG{o}{=} \PYG{n}{C\PYGZus{}eqv\PYGZus{}to\PYGZus{}C\PYGZus{}S\PYGZus{}0} \PYG{c+c1}{\PYGZsh{} imported correlation function for chloride content from soluiton to concrete}

\PYG{n}{pars\PYGZus{}raw}\PYG{o}{.}\PYG{n}{exposure\PYGZus{}condition} \PYG{o}{=} \PYG{l+s+s1}{\PYGZsq{}}\PYG{l+s+s1}{splash}\PYG{l+s+s1}{\PYGZsq{}}
\PYG{n}{pars\PYGZus{}raw}\PYG{o}{.}\PYG{n}{exposure\PYGZus{}condition\PYGZus{}geom\PYGZus{}sensitive} \PYG{o}{=} \PYG{k+kc}{True}
\PYG{n}{pars\PYGZus{}raw}\PYG{o}{.}\PYG{n}{T\PYGZus{}real} \PYG{o}{=} \PYG{l+m+mi}{273} \PYG{o}{+} \PYG{l+m+mi}{25}  \PYG{c+c1}{\PYGZsh{} averaged ambient temperature[K]}

\PYG{n}{pars\PYGZus{}raw}\PYG{o}{.}\PYG{n}{x\PYGZus{}a} \PYG{o}{=} \PYG{l+m+mf}{10.}
\PYG{n}{pars\PYGZus{}raw}\PYG{o}{.}\PYG{n}{x\PYGZus{}h} \PYG{o}{=} \PYG{l+m+mf}{10.}
\PYG{n}{pars\PYGZus{}raw}\PYG{o}{.}\PYG{n}{D\PYGZus{}RCM\PYGZus{}test} \PYG{o}{=} \PYG{l+s+s1}{\PYGZsq{}}\PYG{l+s+s1}{N/A}\PYG{l+s+s1}{\PYGZsq{}}
\PYG{n}{pars\PYGZus{}raw}\PYG{o}{.}\PYG{n}{concrete\PYGZus{}type} \PYG{o}{=} \PYG{l+s+s1}{\PYGZsq{}}\PYG{l+s+s1}{Portland cement concrete}\PYG{l+s+s1}{\PYGZsq{}}
\PYG{n}{pars\PYGZus{}raw}\PYG{o}{.}\PYG{n}{cement\PYGZus{}concrete\PYGZus{}ratio} \PYG{o}{=} \PYG{l+m+mf}{300.}\PYG{o}{/}\PYG{l+m+mf}{2400.}
\PYG{n}{pars\PYGZus{}raw}\PYG{o}{.}\PYG{n}{C\PYGZus{}max\PYGZus{}user\PYGZus{}input} \PYG{o}{=} \PYG{k+kc}{None}
\PYG{n}{pars\PYGZus{}raw}\PYG{o}{.}\PYG{n}{C\PYGZus{}max\PYGZus{}option} \PYG{o}{=} \PYG{l+s+s1}{\PYGZsq{}}\PYG{l+s+s1}{empirical}\PYG{l+s+s1}{\PYGZsq{}}
\PYG{n}{pars\PYGZus{}raw}\PYG{o}{.}\PYG{n}{C\PYGZus{}0} \PYG{o}{=} \PYG{l+m+mi}{0}

\PYG{n}{pars\PYGZus{}raw}\PYG{o}{.}\PYG{n}{C\PYGZus{}crit\PYGZus{}distrib\PYGZus{}param} \PYG{o}{=} \PYG{n}{C\PYGZus{}crit\PYGZus{}param}\PYG{p}{(}\PYG{p}{)}  \PYG{c+c1}{\PYGZsh{} critical chloride content import from Chloride module 0.6 wt.\PYGZpc{} cement (mean value)}

\PYG{c+c1}{\PYGZsh{} more options}
\PYG{n}{pars\PYGZus{}raw}\PYG{o}{.}\PYG{n}{option} \PYG{o}{=} \PYG{n}{Param}\PYG{p}{(}\PYG{p}{)}
\PYG{n}{pars\PYGZus{}raw}\PYG{o}{.}\PYG{n}{option}\PYG{o}{.}\PYG{n}{choose} \PYG{o}{=} \PYG{k+kc}{True}
\PYG{n}{pars\PYGZus{}raw}\PYG{o}{.}\PYG{n}{option}\PYG{o}{.}\PYG{n}{cement\PYGZus{}type} \PYG{o}{=} \PYG{l+s+s1}{\PYGZsq{}}\PYG{l+s+s1}{CEM\PYGZus{}I\PYGZus{}42.5\PYGZus{}R+SF}\PYG{l+s+s1}{\PYGZsq{}}
\PYG{n}{pars\PYGZus{}raw}\PYG{o}{.}\PYG{n}{option}\PYG{o}{.}\PYG{n}{wc\PYGZus{}eqv} \PYG{o}{=} \PYG{l+m+mf}{0.4}    \PYG{c+c1}{\PYGZsh{} equivalent water/binder ratio}
\PYG{n}{pars\PYGZus{}raw}\PYG{o}{.}\PYG{n}{option}\PYG{o}{.}\PYG{n}{df\PYGZus{}D\PYGZus{}RCM\PYGZus{}0} \PYG{o}{=} \PYG{n}{load\PYGZus{}df\PYGZus{}D\PYGZus{}RCM}\PYG{p}{(}\PYG{p}{)}
\end{sphinxVerbatim}
}

{
\sphinxsetup{VerbatimColor={named}{nbsphinx-code-bg}}
\sphinxsetup{VerbatimBorderColor={named}{nbsphinx-code-border}}
\begin{sphinxVerbatim}[commandchars=\\\{\}]
\llap{\color{nbsphinxin}[6]:\,\hspace{\fboxrule}\hspace{\fboxsep}}\PYG{c+c1}{\PYGZsh{} initialize model}
\PYG{n}{model\PYGZus{}cl} \PYG{o}{=} \PYG{n}{Chloride\PYGZus{}Model}\PYG{p}{(}\PYG{n}{pars\PYGZus{}raw}\PYG{p}{)}

\PYG{c+c1}{\PYGZsh{} run for 40 mm and 10 year}
\PYG{n}{model\PYGZus{}cl}\PYG{o}{.}\PYG{n}{run}\PYG{p}{(}\PYG{n}{x} \PYG{o}{=} \PYG{l+m+mi}{40}\PYG{p}{,} \PYG{n}{t} \PYG{o}{=} \PYG{l+m+mi}{10}\PYG{p}{)}

\PYG{c+c1}{\PYGZsh{} postproc}
\PYG{n}{model\PYGZus{}cl}\PYG{o}{.}\PYG{n}{postproc}\PYG{p}{(}\PYG{n}{plot}\PYG{o}{=}\PYG{k+kc}{True}\PYG{p}{)}
\end{sphinxVerbatim}
}

{

\kern-\sphinxverbatimsmallskipamount\kern-\baselineskip
\kern+\FrameHeightAdjust\kern-\fboxrule
\vspace{\nbsphinxcodecellspacing}

\sphinxsetup{VerbatimColor={named}{nbsphinx-stderr}}
\sphinxsetup{VerbatimBorderColor={named}{nbsphinx-code-border}}
\begin{sphinxVerbatim}[commandchars=\\\{\}]
/Users/gangli/anaconda3/lib/python3.7/site-packages/scipy/optimize/minpack.py:808: OptimizeWarning: Covariance of the parameters could not be estimated
  category=OptimizeWarning)
Pf(g = R-S < 0) from various methods
    sample count: 0.5239
    g integral: 0.5264259274316304
    R S integral: 0.5264361482672922
    beta\_factor: -0.31087440419437906
\end{sphinxVerbatim}
}

\hrule height -\fboxrule\relax
\vspace{\nbsphinxcodecellspacing}

\makeatletter\setbox\nbsphinxpromptbox\box\voidb@x\makeatother

\begin{nbsphinxfancyoutput}

\noindent\sphinxincludegraphics[width=704\sphinxpxdimen,height=208\sphinxpxdimen]{{chloride_example_3_1}.png}

\end{nbsphinxfancyoutput}

{
\sphinxsetup{VerbatimColor={named}{nbsphinx-code-bg}}
\sphinxsetup{VerbatimBorderColor={named}{nbsphinx-code-border}}
\begin{sphinxVerbatim}[commandchars=\\\{\}]
\llap{\color{nbsphinxin}[9]:\,\hspace{\fboxrule}\hspace{\fboxsep}}\PYG{c+c1}{\PYGZsh{} Calibration}
\PYG{c+c1}{\PYGZsh{} field data at three depth}
\PYG{n}{chloride\PYGZus{}content\PYGZus{}field} \PYG{o}{=} \PYG{n}{pd}\PYG{o}{.}\PYG{n}{DataFrame}\PYG{p}{(}\PYG{p}{)}
\PYG{n}{chloride\PYGZus{}content\PYGZus{}field}\PYG{p}{[}\PYG{l+s+s1}{\PYGZsq{}}\PYG{l+s+s1}{depth}\PYG{l+s+s1}{\PYGZsq{}}\PYG{p}{]} \PYG{o}{=} \PYG{p}{[}\PYG{l+m+mf}{12.5}\PYG{p}{,} \PYG{l+m+mi}{50}\PYG{p}{,} \PYG{l+m+mi}{100}\PYG{p}{]}  \PYG{c+c1}{\PYGZsh{} [mm]}
\PYG{n}{chloride\PYGZus{}content\PYGZus{}field}\PYG{p}{[}\PYG{l+s+s1}{\PYGZsq{}}\PYG{l+s+s1}{cl}\PYG{l+s+s1}{\PYGZsq{}}\PYG{p}{]} \PYG{o}{=} \PYG{n}{np}\PYG{o}{.}\PYG{n}{array}\PYG{p}{(}\PYG{p}{[}\PYG{l+m+mf}{0.226}\PYG{p}{,} \PYG{l+m+mf}{0.04}\PYG{p}{,} \PYG{l+m+mf}{0.014}\PYG{p}{]}\PYG{p}{)} \PYG{o}{/} \PYG{n}{pars\PYGZus{}raw}\PYG{o}{.}\PYG{n}{cement\PYGZus{}concrete\PYGZus{}ratio}  \PYG{c+c1}{\PYGZsh{} chloride\PYGZus{}content[wt.\PYGZhy{}\PYGZpc{}/cement]}
\PYG{n+nb}{print}\PYG{p}{(}\PYG{n}{chloride\PYGZus{}content\PYGZus{}field}\PYG{p}{)}
\end{sphinxVerbatim}
}

{

\kern-\sphinxverbatimsmallskipamount\kern-\baselineskip
\kern+\FrameHeightAdjust\kern-\fboxrule
\vspace{\nbsphinxcodecellspacing}

\sphinxsetup{VerbatimColor={named}{white}}
\sphinxsetup{VerbatimBorderColor={named}{nbsphinx-code-border}}
\begin{sphinxVerbatim}[commandchars=\\\{\}]
   depth     cl
0   12.5  1.808
1   50.0  0.320
2  100.0  0.112
\end{sphinxVerbatim}
}

{
\sphinxsetup{VerbatimColor={named}{nbsphinx-code-bg}}
\sphinxsetup{VerbatimBorderColor={named}{nbsphinx-code-border}}
\begin{sphinxVerbatim}[commandchars=\\\{\}]
\llap{\color{nbsphinxin}[11]:\,\hspace{\fboxrule}\hspace{\fboxsep}}\PYG{c+c1}{\PYGZsh{}calibrate model to the field chloride content}
\PYG{n}{model\PYGZus{}cl\PYGZus{}cal} \PYG{o}{=} \PYG{n}{model\PYGZus{}cl}\PYG{o}{.}\PYG{n}{calibrate}\PYG{p}{(}\PYG{l+m+mi}{40}\PYG{p}{,} \PYG{n}{chloride\PYGZus{}content\PYGZus{}field}\PYG{p}{,}\PYG{n}{print\PYGZus{}proc}\PYG{o}{=}\PYG{k+kc}{False}\PYG{p}{,} \PYG{n}{plot}\PYG{o}{=}\PYG{k+kc}{True}\PYG{p}{)}
\end{sphinxVerbatim}
}

{

\kern-\sphinxverbatimsmallskipamount\kern-\baselineskip
\kern+\FrameHeightAdjust\kern-\fboxrule
\vspace{\nbsphinxcodecellspacing}

\sphinxsetup{VerbatimColor={named}{white}}
\sphinxsetup{VerbatimBorderColor={named}{nbsphinx-code-border}}
\begin{sphinxVerbatim}[commandchars=\\\{\}]
2.9516601562500007e-13
1.0129394531250003e-12
/Users/gangli/Local Documents/Mitacs project local/Tinkrete/modules/chloride.py:66: RuntimeWarning: invalid value encountered in sqrt
  1 - erf((x - pars.dx) / (2 * (pars.D\_app * t) ** 0.5))
2.574462890625e-12
\end{sphinxVerbatim}
}

\hrule height -\fboxrule\relax
\vspace{\nbsphinxcodecellspacing}

\makeatletter\setbox\nbsphinxpromptbox\box\voidb@x\makeatother

\begin{nbsphinxfancyoutput}

\noindent\sphinxincludegraphics[width=372\sphinxpxdimen,height=248\sphinxpxdimen]{{chloride_example_5_1}.png}

\end{nbsphinxfancyoutput}

{
\sphinxsetup{VerbatimColor={named}{nbsphinx-code-bg}}
\sphinxsetup{VerbatimBorderColor={named}{nbsphinx-code-border}}
\begin{sphinxVerbatim}[commandchars=\\\{\}]
\llap{\color{nbsphinxin}[13]:\,\hspace{\fboxrule}\hspace{\fboxsep}}\PYG{c+c1}{\PYGZsh{} run the calibrated model for 40 mm and 10 year}
\PYG{n}{model\PYGZus{}cl\PYGZus{}cal}\PYG{o}{.}\PYG{n}{run}\PYG{p}{(}\PYG{n}{x} \PYG{o}{=} \PYG{l+m+mi}{40}\PYG{p}{,} \PYG{n}{t} \PYG{o}{=} \PYG{l+m+mi}{10}\PYG{p}{)}
\PYG{n}{model\PYGZus{}cl\PYGZus{}cal}\PYG{o}{.}\PYG{n}{postproc}\PYG{p}{(}\PYG{n}{plot}\PYG{o}{=}\PYG{k+kc}{True}\PYG{p}{)}
\PYG{c+c1}{\PYGZsh{} plt.savefig(\PYGZsq{}chloride\PYGZus{}at\PYGZus{}rebar\PYGZus{}40year.pdf\PYGZsq{},dpi=1200)}
\end{sphinxVerbatim}
}

{

\kern-\sphinxverbatimsmallskipamount\kern-\baselineskip
\kern+\FrameHeightAdjust\kern-\fboxrule
\vspace{\nbsphinxcodecellspacing}

\sphinxsetup{VerbatimColor={named}{white}}
\sphinxsetup{VerbatimBorderColor={named}{nbsphinx-code-border}}
\begin{sphinxVerbatim}[commandchars=\\\{\}]
Pf(g = R-S < 0) from various methods
    sample count: 0.11808
    g integral: 0.11947147378471051
    R S integral: 0.1195360365046989
    beta\_factor: 0.5079907001618054
\end{sphinxVerbatim}
}

\hrule height -\fboxrule\relax
\vspace{\nbsphinxcodecellspacing}

\makeatletter\setbox\nbsphinxpromptbox\box\voidb@x\makeatother

\begin{nbsphinxfancyoutput}

\noindent\sphinxincludegraphics[width=704\sphinxpxdimen,height=208\sphinxpxdimen]{{chloride_example_6_1}.png}

\end{nbsphinxfancyoutput}

{
\sphinxsetup{VerbatimColor={named}{nbsphinx-code-bg}}
\sphinxsetup{VerbatimBorderColor={named}{nbsphinx-code-border}}
\begin{sphinxVerbatim}[commandchars=\\\{\}]
\llap{\color{nbsphinxin}[19]:\,\hspace{\fboxrule}\hspace{\fboxsep}}\PYG{c+c1}{\PYGZsh{} run model for a list of time steps}
\PYG{n}{t\PYGZus{}lis} \PYG{o}{=} \PYG{n}{np}\PYG{o}{.}\PYG{n}{arange}\PYG{p}{(}\PYG{l+m+mi}{5}\PYG{p}{,}\PYG{l+m+mi}{50}\PYG{p}{,}\PYG{l+m+mi}{5}\PYG{p}{)}
\PYG{n}{cover} \PYG{o}{=} \PYG{l+m+mi}{50}
\PYG{n}{pf\PYGZus{}lis}\PYG{p}{,} \PYG{n}{beta\PYGZus{}lis} \PYG{o}{=} \PYG{n}{model\PYGZus{}cl\PYGZus{}cal}\PYG{o}{.}\PYG{n}{chloride\PYGZus{}with\PYGZus{}year}\PYG{p}{(}\PYG{n}{depth}\PYG{o}{=}\PYG{n}{cover}\PYG{p}{,} \PYG{n}{year\PYGZus{}lis}\PYG{o}{=}\PYG{n}{t\PYGZus{}lis}\PYG{p}{,}\PYG{n}{amplify}\PYG{o}{=}\PYG{l+m+mi}{1}\PYG{p}{)}
\PYG{c+c1}{\PYGZsh{} fig.savefig(\PYGZsq{}RS\PYGZus{}time\PYGZus{}chloride.pdf\PYGZsq{},dpi=1200)}

\end{sphinxVerbatim}
}

{

\kern-\sphinxverbatimsmallskipamount\kern-\baselineskip
\kern+\FrameHeightAdjust\kern-\fboxrule
\vspace{\nbsphinxcodecellspacing}

\sphinxsetup{VerbatimColor={named}{nbsphinx-stderr}}
\sphinxsetup{VerbatimBorderColor={named}{nbsphinx-code-border}}
\begin{sphinxVerbatim}[commandchars=\\\{\}]
/Users/gangli/Local Documents/Mitacs project local/Tinkrete/modules/chloride.py:66: RuntimeWarning: invalid value encountered in sqrt
  1 - erf((x - pars.dx) / (2 * (pars.D\_app * t) ** 0.5))
/Users/gangli/Local Documents/Mitacs project local/Tinkrete/modules/helper\_func.py:440: IntegrationWarning: The maximum number of subdivisions (50) has been achieved.
  If increasing the limit yields no improvement it is advised to analyze
  the integrand in order to determine the difficulties.  If the position of a
  local difficulty can be determined (singularity, discontinuity) one will
  probably gain from splitting up the interval and calling the integrator
  on the subranges.  Perhaps a special-purpose integrator should be used.
  lambda x: R\_distrib.cdf(x) * S\_kde\_fit(x)[0], 0, S\_dropna.max()
\end{sphinxVerbatim}
}

\hrule height -\fboxrule\relax
\vspace{\nbsphinxcodecellspacing}

\makeatletter\setbox\nbsphinxpromptbox\box\voidb@x\makeatother

\begin{nbsphinxfancyoutput}

\noindent\sphinxincludegraphics[width=568\sphinxpxdimen,height=564\sphinxpxdimen]{{chloride_example_7_1}.png}

\end{nbsphinxfancyoutput}

{
\sphinxsetup{VerbatimColor={named}{nbsphinx-code-bg}}
\sphinxsetup{VerbatimBorderColor={named}{nbsphinx-code-border}}
\begin{sphinxVerbatim}[commandchars=\\\{\}]
\llap{\color{nbsphinxin}[ ]:\,\hspace{\fboxrule}\hspace{\fboxsep}}
\end{sphinxVerbatim}
}


\section{corrosion module example}
\label{\detokenize{corrosion_example:corrosion-module-example}}\label{\detokenize{corrosion_example::doc}}\begin{itemize}
\item {} 
\sphinxAtStartPar
Input Raw data

\item {} 
\sphinxAtStartPar
moisture

\item {} 
\sphinxAtStartPar
temperature

\item {} 
\sphinxAtStartPar
corrosion state determined by chloride and carbonation from other modules

\item {} 
\sphinxAtStartPar
Output

\item {} 
\sphinxAtStartPar
icorr and corrosion rate

\item {} 
\sphinxAtStartPar
accumulated sectionloss with time

\end{itemize}

{
\sphinxsetup{VerbatimColor={named}{nbsphinx-code-bg}}
\sphinxsetup{VerbatimBorderColor={named}{nbsphinx-code-border}}
\begin{sphinxVerbatim}[commandchars=\\\{\}]
\llap{\color{nbsphinxin}[1]:\,\hspace{\fboxrule}\hspace{\fboxsep}}\PYG{o}{\PYGZpc{}}\PYG{k}{matplotlib} inline
\PYG{k+kn}{import} \PYG{n+nn}{numpy} \PYG{k}{as} \PYG{n+nn}{np}
\PYG{k+kn}{from} \PYG{n+nn}{corrosion} \PYG{k+kn}{import} \PYG{n}{Corrosion\PYGZus{}Model}\PYG{p}{,} \PYG{n}{Section\PYGZus{}loss\PYGZus{}Model}
\PYG{k+kn}{import} \PYG{n+nn}{helper\PYGZus{}func} \PYG{k}{as} \PYG{n+nn}{hf}
\PYG{k+kn}{import} \PYG{n+nn}{matplotlib}\PYG{n+nn}{.}\PYG{n+nn}{pyplot} \PYG{k}{as} \PYG{n+nn}{plt}
\end{sphinxVerbatim}
}

{
\sphinxsetup{VerbatimColor={named}{nbsphinx-code-bg}}
\sphinxsetup{VerbatimBorderColor={named}{nbsphinx-code-border}}
\begin{sphinxVerbatim}[commandchars=\\\{\}]
\llap{\color{nbsphinxin}[2]:\,\hspace{\fboxrule}\hspace{\fboxsep}}
\PYG{k}{class} \PYG{n+nc}{Param}\PYG{p}{:} \PYG{k}{pass}
\PYG{n}{raw\PYGZus{}pars} \PYG{o}{=} \PYG{n}{Param}\PYG{p}{(}\PYG{p}{)}

\PYG{c+c1}{\PYGZsh{} geometry and age}
\PYG{n}{raw\PYGZus{}pars}\PYG{o}{.}\PYG{n}{d} \PYG{o}{=} \PYG{l+m+mf}{0.04}  \PYG{c+c1}{\PYGZsh{} cover depth [m]}
\PYG{n}{raw\PYGZus{}pars}\PYG{o}{.}\PYG{n}{t} \PYG{o}{=} \PYG{l+m+mi}{3650}  \PYG{c+c1}{\PYGZsh{} age[day]}

\PYG{c+c1}{\PYGZsh{} concrete composition}
\PYG{n}{raw\PYGZus{}pars}\PYG{o}{.}\PYG{n}{cement\PYGZus{}type} \PYG{o}{=} \PYG{l+s+s1}{\PYGZsq{}}\PYG{l+s+s1}{Type I}\PYG{l+s+s1}{\PYGZsq{}}
\PYG{n}{raw\PYGZus{}pars}\PYG{o}{.}\PYG{n}{concrete\PYGZus{}density} \PYG{o}{=} \PYG{l+m+mi}{2400} \PYG{c+c1}{\PYGZsh{}kg/m\PYGZca{}3}
\PYG{n}{raw\PYGZus{}pars}\PYG{o}{.}\PYG{n}{a\PYGZus{}c} \PYG{o}{=} \PYG{l+m+mi}{2}        \PYG{c+c1}{\PYGZsh{} aggregate(fine and coarse)/cement ratio}
\PYG{n}{raw\PYGZus{}pars}\PYG{o}{.}\PYG{n}{w\PYGZus{}c} \PYG{o}{=} \PYG{l+m+mf}{0.5}      \PYG{c+c1}{\PYGZsh{} water/cement ratio}
\PYG{n}{raw\PYGZus{}pars}\PYG{o}{.}\PYG{n}{rho\PYGZus{}c}\PYG{o}{=} \PYG{l+m+mf}{3.1e3}   \PYG{c+c1}{\PYGZsh{} density of cement particle [kg/m\PYGZca{}3]}
\PYG{n}{raw\PYGZus{}pars}\PYG{o}{.}\PYG{n}{rho\PYGZus{}a}\PYG{o}{=} \PYG{l+m+mf}{2600.}   \PYG{c+c1}{\PYGZsh{} density of aggregate particle(fine and coarse) range 2400\PYGZhy{}2900 [kg/m\PYGZca{}3]}


\PYG{c+c1}{\PYGZsh{} concrete condition}
\PYG{n}{raw\PYGZus{}pars}\PYG{o}{.}\PYG{n}{epsilon} \PYG{o}{=} \PYG{l+m+mf}{0.25}     \PYG{c+c1}{\PYGZsh{} porosity of concrete}
\PYG{n}{raw\PYGZus{}pars}\PYG{o}{.}\PYG{n}{theta\PYGZus{}water} \PYG{o}{=} \PYG{l+m+mf}{0.12} \PYG{c+c1}{\PYGZsh{} volumetric water content}
\PYG{n}{raw\PYGZus{}pars}\PYG{o}{.}\PYG{n}{T} \PYG{o}{=} \PYG{l+m+mf}{273.15}\PYG{o}{+}\PYG{l+m+mi}{25}      \PYG{c+c1}{\PYGZsh{} temperature [K]}


\end{sphinxVerbatim}
}

{
\sphinxsetup{VerbatimColor={named}{nbsphinx-code-bg}}
\sphinxsetup{VerbatimBorderColor={named}{nbsphinx-code-border}}
\begin{sphinxVerbatim}[commandchars=\\\{\}]
\llap{\color{nbsphinxin}[3]:\,\hspace{\fboxrule}\hspace{\fboxsep}}\PYG{c+c1}{\PYGZsh{} initialize and run model}
\PYG{n}{model\PYGZus{}corr} \PYG{o}{=} \PYG{n}{Corrosion\PYGZus{}Model}\PYG{p}{(}\PYG{n}{raw\PYGZus{}pars}\PYG{p}{)}
\PYG{n}{model\PYGZus{}corr}\PYG{o}{.}\PYG{n}{run}\PYG{p}{(}\PYG{p}{)}

\PYG{c+c1}{\PYGZsh{} result}
\PYG{n}{model\PYGZus{}corr}\PYG{o}{.}\PYG{n}{icorr}

\PYG{c+c1}{\PYGZsh{} icorr}
\PYG{n+nb}{print}\PYG{p}{(}\PYG{l+s+sa}{f}\PYG{l+s+s2}{\PYGZdq{}}\PYG{l+s+s2}{icorr [A/m\PYGZca{}2]: }\PYG{l+s+s2}{\PYGZob{}}\PYG{l+s+s2}{model\PYGZus{}corr.icorr.mean()\PYGZcb{}}\PYG{l+s+s2}{\PYGZdq{}}\PYG{p}{)}
\PYG{c+c1}{\PYGZsh{} section loss}
\PYG{n}{model\PYGZus{}corr}\PYG{o}{.}\PYG{n}{x\PYGZus{}loss\PYGZus{}rate}
\PYG{n+nb}{print}\PYG{p}{(}\PYG{l+s+sa}{f}\PYG{l+s+s2}{\PYGZdq{}}\PYG{l+s+s2}{section loss rate [mm/year]: }\PYG{l+s+s2}{\PYGZob{}}\PYG{l+s+s2}{model\PYGZus{}corr.x\PYGZus{}loss\PYGZus{}rate.mean()\PYGZcb{}}\PYG{l+s+s2}{\PYGZdq{}}\PYG{p}{)}
\end{sphinxVerbatim}
}

{

\kern-\sphinxverbatimsmallskipamount\kern-\baselineskip
\kern+\FrameHeightAdjust\kern-\fboxrule
\vspace{\nbsphinxcodecellspacing}

\sphinxsetup{VerbatimColor={named}{white}}
\sphinxsetup{VerbatimBorderColor={named}{nbsphinx-code-border}}
\begin{sphinxVerbatim}[commandchars=\\\{\}]
icorr [A/m\^{}2]: 0.006407370834095256
section loss rate [mm/year]: 0.007420513570849189
\end{sphinxVerbatim}
}
\begin{itemize}
\item {} 
\sphinxAtStartPar
Accumulated section loss with the increasing probability of active corrosion

\end{itemize}

{
\sphinxsetup{VerbatimColor={named}{nbsphinx-code-bg}}
\sphinxsetup{VerbatimBorderColor={named}{nbsphinx-code-border}}
\begin{sphinxVerbatim}[commandchars=\\\{\}]
\llap{\color{nbsphinxin}[10]:\,\hspace{\fboxrule}\hspace{\fboxsep}}\PYG{c+c1}{\PYGZsh{} time steps}
\PYG{n}{t\PYGZus{}lis} \PYG{o}{=} \PYG{n}{np}\PYG{o}{.}\PYG{n}{linspace}\PYG{p}{(}\PYG{l+m+mi}{0}\PYG{p}{,} \PYG{l+m+mi}{365}\PYG{o}{*}\PYG{l+m+mi}{100} \PYG{p}{,} \PYG{l+m+mi}{100}\PYG{p}{)}

\PYG{c+c1}{\PYGZsh{} Given probability of active corrosion with time, and the section loss  (determined by membrane, carbonation, chloride module)}
\PYG{c+c1}{\PYGZsh{} dummy data used for this example}
\PYG{n}{pf\PYGZus{}lis} \PYG{o}{=} \PYG{n}{np}\PYG{o}{.}\PYG{n}{linspace}\PYG{p}{(}\PYG{l+m+mi}{0}\PYG{p}{,}\PYG{l+m+mi}{1}\PYG{p}{,}\PYG{n+nb}{len}\PYG{p}{(}\PYG{n}{t\PYGZus{}lis}\PYG{p}{)}\PYG{p}{)}\PYG{o}{*}\PYG{o}{*}\PYG{l+m+mi}{5}
\PYG{n}{plt}\PYG{o}{.}\PYG{n}{plot}\PYG{p}{(}\PYG{n}{t\PYGZus{}lis} \PYG{o}{/} \PYG{l+m+mi}{365}\PYG{p}{,} \PYG{n}{pf\PYGZus{}lis}\PYG{p}{)}
\PYG{n}{plt}\PYG{o}{.}\PYG{n}{title}\PYG{p}{(}\PYG{l+s+s1}{\PYGZsq{}}\PYG{l+s+s1}{dummy data Pf vs time}\PYG{l+s+s1}{\PYGZsq{}}\PYG{p}{)}
\PYG{n}{plt}\PYG{o}{.}\PYG{n}{xlabel}\PYG{p}{(}\PYG{l+s+s1}{\PYGZsq{}}\PYG{l+s+s1}{Time[year]}\PYG{l+s+s1}{\PYGZsq{}}\PYG{p}{)}
\PYG{n}{plt}\PYG{o}{.}\PYG{n}{ylabel}\PYG{p}{(}\PYG{l+s+s1}{\PYGZsq{}}\PYG{l+s+s1}{probability of active corrosion}\PYG{l+s+s1}{\PYGZsq{}}\PYG{p}{)}

\end{sphinxVerbatim}
}

{

\kern-\sphinxverbatimsmallskipamount\kern-\baselineskip
\kern+\FrameHeightAdjust\kern-\fboxrule
\vspace{\nbsphinxcodecellspacing}

\sphinxsetup{VerbatimColor={named}{white}}
\sphinxsetup{VerbatimBorderColor={named}{nbsphinx-code-border}}
\begin{sphinxVerbatim}[commandchars=\\\{\}]
\llap{\color{nbsphinxout}[10]:\,\hspace{\fboxrule}\hspace{\fboxsep}}Text(0, 0.5, 'probability of active corrosion')
\end{sphinxVerbatim}
}

\hrule height -\fboxrule\relax
\vspace{\nbsphinxcodecellspacing}

\makeatletter\setbox\nbsphinxpromptbox\box\voidb@x\makeatother

\begin{nbsphinxfancyoutput}

\noindent\sphinxincludegraphics[width=386\sphinxpxdimen,height=278\sphinxpxdimen]{{corrosion_example_5_1}.png}

\end{nbsphinxfancyoutput}

{
\sphinxsetup{VerbatimColor={named}{nbsphinx-code-bg}}
\sphinxsetup{VerbatimBorderColor={named}{nbsphinx-code-border}}
\begin{sphinxVerbatim}[commandchars=\\\{\}]
\llap{\color{nbsphinxin}[16]:\,\hspace{\fboxrule}\hspace{\fboxsep}}\PYG{c+c1}{\PYGZsh{} prepare Param object for section loss object}
\PYG{n}{pars\PYGZus{}sl} \PYG{o}{=} \PYG{n}{Param}\PYG{p}{(}\PYG{p}{)}
\PYG{n}{pars\PYGZus{}sl}\PYG{o}{.}\PYG{n}{x\PYGZus{}loss\PYGZus{}rate} \PYG{o}{=} \PYG{n}{model\PYGZus{}corr}\PYG{o}{.}\PYG{n}{x\PYGZus{}loss\PYGZus{}rate}\PYG{o}{.}\PYG{n}{mean}\PYG{p}{(}\PYG{p}{)}     \PYG{c+c1}{\PYGZsh{} mm/year mean section loss rate from the corrosion model}
\PYG{n}{pars\PYGZus{}sl}\PYG{o}{.}\PYG{n}{p\PYGZus{}active\PYGZus{}t\PYGZus{}curve} \PYG{o}{=} \PYG{p}{(}\PYG{n}{pf\PYGZus{}lis}\PYG{p}{,} \PYG{n}{t\PYGZus{}lis}\PYG{p}{)}              \PYG{c+c1}{\PYGZsh{} use dummy data for this example}

\PYG{c+c1}{\PYGZsh{} critical section loss from the external structural analysis}
\PYG{n}{pars\PYGZus{}sl}\PYG{o}{.}\PYG{n}{x\PYGZus{}loss\PYGZus{}limit\PYGZus{}mean} \PYG{o}{=} \PYG{l+m+mf}{0.5}         \PYG{c+c1}{\PYGZsh{} mm}
\PYG{n}{pars\PYGZus{}sl}\PYG{o}{.}\PYG{n}{x\PYGZus{}loss\PYGZus{}limit\PYGZus{}std} \PYG{o}{=} \PYG{l+m+mf}{0.5} \PYG{o}{*} \PYG{l+m+mf}{0.005}  \PYG{c+c1}{\PYGZsh{} mm}

\PYG{c+c1}{\PYGZsh{} initialize section loss model object}
\PYG{n}{model\PYGZus{}sl} \PYG{o}{=} \PYG{n}{Section\PYGZus{}loss\PYGZus{}Model}\PYG{p}{(}\PYG{n}{pars\PYGZus{}sl}\PYG{p}{)}

\PYG{c+c1}{\PYGZsh{} run model for one time step, 80 year}
\PYG{n}{model\PYGZus{}sl}\PYG{o}{.}\PYG{n}{run}\PYG{p}{(}\PYG{n}{t\PYGZus{}end} \PYG{o}{=} \PYG{l+m+mi}{50}\PYG{p}{)}
\PYG{n}{model\PYGZus{}sl}\PYG{o}{.}\PYG{n}{postproc}\PYG{p}{(}\PYG{n}{plot}\PYG{o}{=}\PYG{k+kc}{True}\PYG{p}{)}

\end{sphinxVerbatim}
}

{

\kern-\sphinxverbatimsmallskipamount\kern-\baselineskip
\kern+\FrameHeightAdjust\kern-\fboxrule
\vspace{\nbsphinxcodecellspacing}

\sphinxsetup{VerbatimColor={named}{white}}
\sphinxsetup{VerbatimBorderColor={named}{nbsphinx-code-border}}
\begin{sphinxVerbatim}[commandchars=\\\{\}]
warning: very small Pf
Pf(g = R-S < 0) from various methods
    sample count: 0.0
    g integral: -5.000000269139826e-06
    R S integral: 0.0
    beta\_factor: 41.75751987706608
\end{sphinxVerbatim}
}

\hrule height -\fboxrule\relax
\vspace{\nbsphinxcodecellspacing}

\makeatletter\setbox\nbsphinxpromptbox\box\voidb@x\makeatother

\begin{nbsphinxfancyoutput}

\noindent\sphinxincludegraphics[width=704\sphinxpxdimen,height=208\sphinxpxdimen]{{corrosion_example_6_1}.png}

\end{nbsphinxfancyoutput}

{
\sphinxsetup{VerbatimColor={named}{nbsphinx-code-bg}}
\sphinxsetup{VerbatimBorderColor={named}{nbsphinx-code-border}}
\begin{sphinxVerbatim}[commandchars=\\\{\}]
\llap{\color{nbsphinxin}[27]:\,\hspace{\fboxrule}\hspace{\fboxsep}}\PYG{c+c1}{\PYGZsh{} run the model through a list of year steps}
\PYG{n}{pf\PYGZus{}sl}\PYG{p}{,} \PYG{n}{beta\PYGZus{}sl} \PYG{o}{=} \PYG{n}{model\PYGZus{}sl}\PYG{o}{.}\PYG{n}{section\PYGZus{}loss\PYGZus{}with\PYGZus{}year}\PYG{p}{(}\PYG{n}{year\PYGZus{}lis}\PYG{o}{=}\PYG{n}{np}\PYG{o}{.}\PYG{n}{arange}\PYG{p}{(}\PYG{l+m+mi}{65}\PYG{p}{,}\PYG{l+m+mi}{70}\PYG{p}{,}\PYG{l+m+mf}{0.2}\PYG{p}{)}\PYG{p}{,} \PYG{n}{amplify}\PYG{o}{=}\PYG{l+m+mf}{5e\PYGZhy{}4}\PYG{p}{)}

\end{sphinxVerbatim}
}

{

\kern-\sphinxverbatimsmallskipamount\kern-\baselineskip
\kern+\FrameHeightAdjust\kern-\fboxrule
\vspace{\nbsphinxcodecellspacing}

\sphinxsetup{VerbatimColor={named}{white}}
\sphinxsetup{VerbatimBorderColor={named}{nbsphinx-code-border}}
\begin{sphinxVerbatim}[commandchars=\\\{\}]
warning: very small Pf
warning: very small Pf
warning: very small Pf
warning: very small Pf
warning: very small Pf
warning: very small Pf
\end{sphinxVerbatim}
}

\hrule height -\fboxrule\relax
\vspace{\nbsphinxcodecellspacing}

\makeatletter\setbox\nbsphinxpromptbox\box\voidb@x\makeatother

\begin{nbsphinxfancyoutput}

\noindent\sphinxincludegraphics[width=568\sphinxpxdimen,height=564\sphinxpxdimen]{{corrosion_example_7_1}.png}

\end{nbsphinxfancyoutput}

{
\sphinxsetup{VerbatimColor={named}{nbsphinx-code-bg}}
\sphinxsetup{VerbatimBorderColor={named}{nbsphinx-code-border}}
\begin{sphinxVerbatim}[commandchars=\\\{\}]
\llap{\color{nbsphinxin}[ ]:\,\hspace{\fboxrule}\hspace{\fboxsep}}

\end{sphinxVerbatim}
}


\section{cracking model example}
\label{\detokenize{cracking_example:cracking-model-example}}\label{\detokenize{cracking_example::doc}}
{
\sphinxsetup{VerbatimColor={named}{nbsphinx-code-bg}}
\sphinxsetup{VerbatimBorderColor={named}{nbsphinx-code-border}}
\begin{sphinxVerbatim}[commandchars=\\\{\}]
\llap{\color{nbsphinxin}[39]:\,\hspace{\fboxrule}\hspace{\fboxsep}}\PYG{o}{\PYGZpc{}}\PYG{k}{matplotlib} inline
\PYG{c+c1}{\PYGZsh{} \PYGZpc{}load\PYGZus{}ext autoreload}
\PYG{c+c1}{\PYGZsh{} \PYGZpc{}autoreload 2}

\PYG{k+kn}{import} \PYG{n+nn}{helper\PYGZus{}func} \PYG{k}{as} \PYG{n+nn}{hf}
\PYG{k+kn}{from} \PYG{n+nn}{cracking} \PYG{k+kn}{import} \PYG{n}{Cracking\PYGZus{}Model}
\end{sphinxVerbatim}
}

{
\sphinxsetup{VerbatimColor={named}{nbsphinx-code-bg}}
\sphinxsetup{VerbatimBorderColor={named}{nbsphinx-code-border}}
\begin{sphinxVerbatim}[commandchars=\\\{\}]
\llap{\color{nbsphinxin}[40]:\,\hspace{\fboxrule}\hspace{\fboxsep}}\PYG{c+c1}{\PYGZsh{} raw data}
\PYG{k}{class} \PYG{n+nc}{Param}\PYG{p}{:} \PYG{k}{pass}
\PYG{n}{raw\PYGZus{}pars} \PYG{o}{=} \PYG{n}{Param}\PYG{p}{(}\PYG{p}{)}

\PYG{c+c1}{\PYGZsh{} material properties}
\PYG{n}{r0\PYGZus{}bar\PYGZus{}mean} \PYG{o}{=} \PYG{l+m+mf}{5e\PYGZhy{}3}          \PYG{c+c1}{\PYGZsh{} rebar diameter [m]}
\PYG{n}{f\PYGZus{}t\PYGZus{}mean}\PYG{o}{=}\PYG{l+m+mf}{5.}                 \PYG{c+c1}{\PYGZsh{} concrete ultimate tensile strength[MPa]}
\PYG{n}{E\PYGZus{}0\PYGZus{}mean}\PYG{o}{=}\PYG{l+m+mf}{32e3}               \PYG{c+c1}{\PYGZsh{} concrete modulus of elesticity [Mpa]}

\PYG{n}{x\PYGZus{}loss\PYGZus{}mean} \PYG{o}{=} \PYG{l+m+mf}{12.5e\PYGZhy{}6}\PYG{o}{*}\PYG{l+m+mf}{0.6}   \PYG{c+c1}{\PYGZsh{} rebar section loss, mean [m]}
\PYG{n}{cover\PYGZus{}mean} \PYG{o}{=} \PYG{l+m+mf}{4e\PYGZhy{}2}           \PYG{c+c1}{\PYGZsh{} cover thickness, mean [m]}

\PYG{n}{raw\PYGZus{}pars}\PYG{o}{.}\PYG{n}{r0\PYGZus{}bar} \PYG{o}{=} \PYG{n}{Normal\PYGZus{}custom}\PYG{p}{(}\PYG{n}{r0\PYGZus{}bar\PYGZus{}mean}\PYG{p}{,} \PYG{l+m+mf}{0.1}\PYG{o}{*}\PYG{n}{r0\PYGZus{}bar\PYGZus{}mean}\PYG{p}{,} \PYG{n}{non\PYGZus{}negative}\PYG{o}{=}\PYG{k+kc}{True}\PYG{p}{)}
\PYG{n}{raw\PYGZus{}pars}\PYG{o}{.}\PYG{n}{x\PYGZus{}loss} \PYG{o}{=} \PYG{n}{Normal\PYGZus{}custom}\PYG{p}{(}\PYG{n}{x\PYGZus{}loss\PYGZus{}mean}\PYG{p}{,} \PYG{l+m+mf}{0.1}\PYG{o}{*}\PYG{n}{x\PYGZus{}loss\PYGZus{}mean}\PYG{p}{,} \PYG{n}{non\PYGZus{}negative}\PYG{o}{=}\PYG{k+kc}{True}\PYG{p}{)}  \PYG{c+c1}{\PYGZsh{} or from the corrosion model solution}
\PYG{n}{raw\PYGZus{}pars}\PYG{o}{.}\PYG{n}{cover} \PYG{o}{=} \PYG{n}{Normal\PYGZus{}custom}\PYG{p}{(}\PYG{n}{cover\PYGZus{}mean}\PYG{p}{,} \PYG{l+m+mf}{0.1}\PYG{o}{*}\PYG{n}{cover\PYGZus{}mean}\PYG{p}{,} \PYG{n}{non\PYGZus{}negative}\PYG{o}{=}\PYG{k+kc}{True}\PYG{p}{)}
\PYG{n}{raw\PYGZus{}pars}\PYG{o}{.}\PYG{n}{f\PYGZus{}t} \PYG{o}{=} \PYG{n}{Normal\PYGZus{}custom}\PYG{p}{(}\PYG{n}{f\PYGZus{}t\PYGZus{}mean}\PYG{p}{,} \PYG{l+m+mf}{0.1}\PYG{o}{*}\PYG{n}{f\PYGZus{}t\PYGZus{}mean}\PYG{p}{,} \PYG{n}{non\PYGZus{}negative}\PYG{o}{=}\PYG{k+kc}{True}\PYG{p}{)}
\PYG{n}{raw\PYGZus{}pars}\PYG{o}{.}\PYG{n}{E\PYGZus{}0} \PYG{o}{=} \PYG{n}{Normal\PYGZus{}custom}\PYG{p}{(}\PYG{n}{E\PYGZus{}0\PYGZus{}mean}\PYG{p}{,} \PYG{l+m+mf}{0.1}\PYG{o}{*}\PYG{n}{E\PYGZus{}0\PYGZus{}mean}\PYG{p}{,} \PYG{n}{non\PYGZus{}negative}\PYG{o}{=}\PYG{k+kc}{True}\PYG{p}{)}
\PYG{n}{raw\PYGZus{}pars}\PYG{o}{.}\PYG{n}{w\PYGZus{}c} \PYG{o}{=} \PYG{n}{Normal\PYGZus{}custom}\PYG{p}{(}\PYG{l+m+mf}{0.5}\PYG{p}{,} \PYG{l+m+mf}{0.1}\PYG{o}{*}\PYG{l+m+mf}{0.6}\PYG{p}{,} \PYG{n}{non\PYGZus{}negative}\PYG{o}{=}\PYG{k+kc}{True}\PYG{p}{)}
\PYG{n}{raw\PYGZus{}pars}\PYG{o}{.}\PYG{n}{r\PYGZus{}v} \PYG{o}{=} \PYG{n}{Beta\PYGZus{}custom}\PYG{p}{(}\PYG{l+m+mf}{2.96}\PYG{p}{,} \PYG{l+m+mf}{2.96}\PYG{o}{*}\PYG{l+m+mf}{0.05}\PYG{p}{,} \PYG{l+m+mf}{3.3}\PYG{p}{,} \PYG{l+m+mf}{2.6}\PYG{p}{)}  \PYG{c+c1}{\PYGZsh{} rust volumetric expansion rate  2.96 lower 2.6  upper: 3.3}


\end{sphinxVerbatim}
}

{
\sphinxsetup{VerbatimColor={named}{nbsphinx-code-bg}}
\sphinxsetup{VerbatimBorderColor={named}{nbsphinx-code-border}}
\begin{sphinxVerbatim}[commandchars=\\\{\}]
\llap{\color{nbsphinxin}[41]:\,\hspace{\fboxrule}\hspace{\fboxsep}}\PYG{c+c1}{\PYGZsh{} initialize model}
\PYG{n}{model\PYGZus{}crack} \PYG{o}{=} \PYG{n}{Cracking\PYGZus{}Model}\PYG{p}{(}\PYG{n}{raw\PYGZus{}pars}\PYG{p}{)}
\PYG{c+c1}{\PYGZsh{} run model in deterministic mode to check the stress and strain diagram}
\PYG{n}{model\PYGZus{}crack}\PYG{o}{.}\PYG{n}{run}\PYG{p}{(}\PYG{n}{stochastic}\PYG{o}{=}\PYG{k+kc}{False}\PYG{p}{)}
\end{sphinxVerbatim}
}

{

\kern-\sphinxverbatimsmallskipamount\kern-\baselineskip
\kern+\FrameHeightAdjust\kern-\fboxrule
\vspace{\nbsphinxcodecellspacing}

\sphinxsetup{VerbatimColor={named}{white}}
\sphinxsetup{VerbatimBorderColor={named}{nbsphinx-code-border}}
\begin{sphinxVerbatim}[commandchars=\\\{\}]
deterministic
/Users/gangli/Local Documents/Mitacs project local/Tinkrete/modules/cracking.py:256: RuntimeWarning: invalid value encountered in greater\_equal
  sol = solve\_stress\_strain\_crack\_stochastic(self.pars)  \# no plot
/Users/gangli/Local Documents/Mitacs project local/Tinkrete/modules/cracking.py:257: RuntimeWarning: invalid value encountered in less\_equal
  else:
/Users/gangli/Local Documents/Mitacs project local/Tinkrete/modules/cracking.py:273: RuntimeWarning: invalid value encountered in less
/Users/gangli/Local Documents/Mitacs project local/Tinkrete/modules/cracking.py:70: RuntimeWarning: invalid value encountered in less\_equal
  return sigma\_theta
/Users/gangli/Local Documents/Mitacs project local/Tinkrete/modules/cracking.py:73: RuntimeWarning: invalid value encountered in greater
  def crack\_width\_open(a, b, u\_st, f\_t, E\_0):
/Users/gangli/Local Documents/Mitacs project local/Tinkrete/modules/cracking.py:73: RuntimeWarning: invalid value encountered in less\_equal
  def crack\_width\_open(a, b, u\_st, f\_t, E\_0):
/Users/gangli/Local Documents/Mitacs project local/Tinkrete/modules/cracking.py:78: RuntimeWarning: invalid value encountered in greater
  inner radius boundary of the rust (center of rebar to rust-concrete) [m]
/Users/gangli/Local Documents/Mitacs project local/Tinkrete/modules/cracking.py:78: RuntimeWarning: invalid value encountered in less\_equal
  inner radius boundary of the rust (center of rebar to rust-concrete) [m]
\end{sphinxVerbatim}
}

\hrule height -\fboxrule\relax
\vspace{\nbsphinxcodecellspacing}

\makeatletter\setbox\nbsphinxpromptbox\box\voidb@x\makeatother

\begin{nbsphinxfancyoutput}

\noindent\sphinxincludegraphics[width=424\sphinxpxdimen,height=280\sphinxpxdimen]{{cracking_example_3_1}.png}

\end{nbsphinxfancyoutput}

{
\sphinxsetup{VerbatimColor={named}{nbsphinx-code-bg}}
\sphinxsetup{VerbatimBorderColor={named}{nbsphinx-code-border}}
\begin{sphinxVerbatim}[commandchars=\\\{\}]
\llap{\color{nbsphinxin}[42]:\,\hspace{\fboxrule}\hspace{\fboxsep}}\PYG{c+c1}{\PYGZsh{} run model in stochastic mode}
\PYG{n}{model\PYGZus{}crack}\PYG{o}{.}\PYG{n}{run}\PYG{p}{(}\PYG{n}{stochastic}\PYG{o}{=}\PYG{k+kc}{True}\PYG{p}{)}
\PYG{n}{model\PYGZus{}crack}\PYG{o}{.}\PYG{n}{postproc}\PYG{p}{(}\PYG{p}{)}

\PYG{n+nb}{print}\PYG{p}{(}\PYG{n}{model\PYGZus{}crack}\PYG{o}{.}\PYG{n}{crack\PYGZus{}visible\PYGZus{}rate\PYGZus{}count}\PYG{p}{)}
\PYG{n+nb}{print}\PYG{p}{(}\PYG{n}{model\PYGZus{}crack}\PYG{o}{.}\PYG{n}{R\PYGZus{}c} \PYG{o}{\PYGZhy{}} \PYG{n}{model\PYGZus{}crack}\PYG{o}{.}\PYG{n}{pars}\PYG{o}{.}\PYG{n}{r0\PYGZus{}bar}\PYG{p}{)} \PYG{c+c1}{\PYGZsh{}/ M.pars.cover}
\PYG{n+nb}{print}\PYG{p}{(}\PYG{n}{model\PYGZus{}crack}\PYG{o}{.}\PYG{n}{pars}\PYG{o}{.}\PYG{n}{cover}\PYG{p}{)}
\end{sphinxVerbatim}
}

{

\kern-\sphinxverbatimsmallskipamount\kern-\baselineskip
\kern+\FrameHeightAdjust\kern-\fboxrule
\vspace{\nbsphinxcodecellspacing}

\sphinxsetup{VerbatimColor={named}{nbsphinx-stderr}}
\sphinxsetup{VerbatimBorderColor={named}{nbsphinx-code-border}}
\begin{sphinxVerbatim}[commandchars=\\\{\}]
/Users/gangli/Local Documents/Mitacs project local/Tinkrete/modules/cracking.py:166: RuntimeWarning: divide by zero encountered in true\_divide

/Users/gangli/Local Documents/Mitacs project local/Tinkrete/modules/cracking.py:168: RuntimeWarning: divide by zero encountered in true\_divide

/Users/gangli/Local Documents/Mitacs project local/Tinkrete/modules/cracking.py:256: RuntimeWarning: invalid value encountered in greater\_equal
  sol = solve\_stress\_strain\_crack\_stochastic(self.pars)  \# no plot
/Users/gangli/Local Documents/Mitacs project local/Tinkrete/modules/cracking.py:257: RuntimeWarning: invalid value encountered in less\_equal
  else:
/Users/gangli/Local Documents/Mitacs project local/Tinkrete/modules/cracking.py:267: RuntimeWarning: divide by zero encountered in true\_divide
  crack\_length\_over\_cover[np.isnan(crack\_length\_over\_cover)] = 0.0  \# crack length=0 for no crack
/Users/gangli/Local Documents/Mitacs project local/Tinkrete/modules/cracking.py:273: RuntimeWarning: invalid value encountered in less
/Users/gangli/Local Documents/Mitacs project local/Tinkrete/modules/cracking.py:70: RuntimeWarning: invalid value encountered in less\_equal
  return sigma\_theta
/Users/gangli/Local Documents/Mitacs project local/Tinkrete/modules/cracking.py:73: RuntimeWarning: invalid value encountered in greater
  def crack\_width\_open(a, b, u\_st, f\_t, E\_0):
/Users/gangli/Local Documents/Mitacs project local/Tinkrete/modules/cracking.py:73: RuntimeWarning: invalid value encountered in less\_equal
  def crack\_width\_open(a, b, u\_st, f\_t, E\_0):
/Users/gangli/Local Documents/Mitacs project local/Tinkrete/modules/cracking.py:78: RuntimeWarning: invalid value encountered in greater
  inner radius boundary of the rust (center of rebar to rust-concrete) [m]
/Users/gangli/Local Documents/Mitacs project local/Tinkrete/modules/cracking.py:78: RuntimeWarning: invalid value encountered in less\_equal
  inner radius boundary of the rust (center of rebar to rust-concrete) [m]
0.0
[0.01133316 0.00386504        nan {\ldots} 0.00987846 0.00390584 0.00404105]
[0.04600693 0.03706798 0.04092084 {\ldots} 0.04372599 0.03084513 0.03971957]
\end{sphinxVerbatim}
}

{
\sphinxsetup{VerbatimColor={named}{nbsphinx-code-bg}}
\sphinxsetup{VerbatimBorderColor={named}{nbsphinx-code-border}}
\begin{sphinxVerbatim}[commandchars=\\\{\}]
\llap{\color{nbsphinxin}[43]:\,\hspace{\fboxrule}\hspace{\fboxsep}}\PYG{n}{plt}\PYG{o}{.}\PYG{n}{figure}\PYG{p}{(}\PYG{p}{)}
\PYG{n}{hf}\PYG{o}{.}\PYG{n}{Hist\PYGZus{}custom}\PYG{p}{(}\PYG{n}{model\PYGZus{}crack}\PYG{o}{.}\PYG{n}{crack\PYGZus{}condition}\PYG{p}{)}
\end{sphinxVerbatim}
}

{

\kern-\sphinxverbatimsmallskipamount\kern-\baselineskip
\kern+\FrameHeightAdjust\kern-\fboxrule
\vspace{\nbsphinxcodecellspacing}

\sphinxsetup{VerbatimColor={named}{white}}
\sphinxsetup{VerbatimBorderColor={named}{nbsphinx-code-border}}
\begin{sphinxVerbatim}[commandchars=\\\{\}]
\llap{\color{nbsphinxout}[43]:\,\hspace{\fboxrule}\hspace{\fboxsep}}(array([29667.,     0.,     0.,     0.,     0.,     0.,     0.,     0.,
            0., 70333.]),
 array([0. , 0.1, 0.2, 0.3, 0.4, 0.5, 0.6, 0.7, 0.8, 0.9, 1. ]),
 <a list of 10 Patch objects>)
\end{sphinxVerbatim}
}

\hrule height -\fboxrule\relax
\vspace{\nbsphinxcodecellspacing}

\makeatletter\setbox\nbsphinxpromptbox\box\voidb@x\makeatother

\begin{nbsphinxfancyoutput}

\noindent\sphinxincludegraphics[width=387\sphinxpxdimen,height=248\sphinxpxdimen]{{cracking_example_5_1}.png}

\end{nbsphinxfancyoutput}

{
\sphinxsetup{VerbatimColor={named}{nbsphinx-code-bg}}
\sphinxsetup{VerbatimBorderColor={named}{nbsphinx-code-border}}
\begin{sphinxVerbatim}[commandchars=\\\{\}]
\llap{\color{nbsphinxin}[52]:\,\hspace{\fboxrule}\hspace{\fboxsep}}\PYG{c+c1}{\PYGZsh{} histgram of the relative crack length though the cover}
\PYG{n}{hf}\PYG{o}{.}\PYG{n}{Hist\PYGZus{}custom}\PYG{p}{(}\PYG{n}{model\PYGZus{}crack}\PYG{o}{.}\PYG{n}{crack\PYGZus{}length\PYGZus{}over\PYGZus{}cover}\PYG{p}{[}\PYG{n}{model\PYGZus{}crack}\PYG{o}{.}\PYG{n}{crack\PYGZus{}length\PYGZus{}over\PYGZus{}cover} \PYG{o}{!=} \PYG{l+m+mi}{0}\PYG{p}{]}\PYG{p}{)} \PYG{c+c1}{\PYGZsh{} eliminate the uncracked case}
\PYG{n}{plt}\PYG{o}{.}\PYG{n}{xlabel}\PYG{p}{(}\PYG{l+s+s1}{\PYGZsq{}}\PYG{l+s+s1}{crack length/ cover}\PYG{l+s+s1}{\PYGZsq{}}\PYG{p}{)}
\end{sphinxVerbatim}
}

{

\kern-\sphinxverbatimsmallskipamount\kern-\baselineskip
\kern+\FrameHeightAdjust\kern-\fboxrule
\vspace{\nbsphinxcodecellspacing}

\sphinxsetup{VerbatimColor={named}{white}}
\sphinxsetup{VerbatimBorderColor={named}{nbsphinx-code-border}}
\begin{sphinxVerbatim}[commandchars=\\\{\}]
\llap{\color{nbsphinxout}[52]:\,\hspace{\fboxrule}\hspace{\fboxsep}}Text(0.5, 0, 'crack length/ cover')
\end{sphinxVerbatim}
}

\hrule height -\fboxrule\relax
\vspace{\nbsphinxcodecellspacing}

\makeatletter\setbox\nbsphinxpromptbox\box\voidb@x\makeatother

\begin{nbsphinxfancyoutput}

\noindent\sphinxincludegraphics[width=362\sphinxpxdimen,height=262\sphinxpxdimen]{{cracking_example_6_1}.png}

\end{nbsphinxfancyoutput}

{
\sphinxsetup{VerbatimColor={named}{nbsphinx-code-bg}}
\sphinxsetup{VerbatimBorderColor={named}{nbsphinx-code-border}}
\begin{sphinxVerbatim}[commandchars=\\\{\}]
\llap{\color{nbsphinxin}[62]:\,\hspace{\fboxrule}\hspace{\fboxsep}}
\end{sphinxVerbatim}
}


\chapter{Indices and tables}
\label{\detokenize{index:indices-and-tables}}\begin{itemize}
\item {} 
\sphinxAtStartPar
\DUrole{xref,std,std-ref}{genindex}

\item {} 
\sphinxAtStartPar
\DUrole{xref,std,std-ref}{modindex}

\item {} 
\sphinxAtStartPar
\DUrole{xref,std,std-ref}{search}

\end{itemize}


\renewcommand{\indexname}{Python Module Index}
\begin{sphinxtheindex}
\let\bigletter\sphinxstyleindexlettergroup
\bigletter{c}
\item\relax\sphinxstyleindexentry{carbonation}\sphinxstyleindexpageref{carbonation:\detokenize{module-carbonation}}
\item\relax\sphinxstyleindexentry{chloride}\sphinxstyleindexpageref{chloride:\detokenize{module-chloride}}
\item\relax\sphinxstyleindexentry{corrosion}\sphinxstyleindexpageref{corrosion:\detokenize{module-corrosion}}
\item\relax\sphinxstyleindexentry{cracking}\sphinxstyleindexpageref{cracking:\detokenize{module-cracking}}
\indexspace
\bigletter{h}
\item\relax\sphinxstyleindexentry{helper\_func}\sphinxstyleindexpageref{helper_func:\detokenize{module-helper_func}}
\indexspace
\bigletter{m}
\item\relax\sphinxstyleindexentry{membrane}\sphinxstyleindexpageref{membrane:\detokenize{module-membrane}}
\indexspace
\bigletter{t}
\item\relax\sphinxstyleindexentry{test\_helper\_func}\sphinxstyleindexpageref{test_helper_func:\detokenize{module-test_helper_func}}
\end{sphinxtheindex}

\renewcommand{\indexname}{Index}
\printindex
\end{document}